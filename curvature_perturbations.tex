%============================================================================
\section{Matching to comoving curvature perturbations}
%============================================================================


With our convention for the metric perturbations in Newtonian gauge\todotag{TBC}
\begin{align}\label{eq:newtonian_gauge_metric}
\begin{aligned}
    &g_{\mu\nu}=a^2\begin{pmatrix}
        -(1+2\Phi) & 0 \\
        0 & (1-2\Psi)\delta_{ij}
    \end{pmatrix}\equiv a^2(\eta_{\mu\nu}+h_{\mu\nu})\\[5pt]
    &\quad \Rightarrow \quad h^{00}=h_{00}=-2\Phi,\quad h^{ii}=h_{ii}=-6\Psi\,.
\end{aligned}
\end{align}
%
We can thus plug this into \eqref{eq:rho_p_fluctuations_via_b_y} to explicitly express $\delta\rho$ and $\delta p$, we get
\begin{align}
\begin{aligned}
    \delta\rho &= (y^2F_{yy})\left(\tfrac12 h^{00}+ \dot{\pi}^0\right) + (-bF_b+byF_{by})\left(-\tfrac12 h^{ii} +(\partial_j \pi^j)\right) + O(2)\\
    &= (y^2F_{yy})\left(-\Phi+ \dot{\pi}^0\right) + (-bF_b+byF_{by})\left(3\Psi +(\partial_j \pi^j)\right) + O(2)
    \\[3pt]
    \delta p &= (yF_y-byF_{by})\left(\tfrac12 h^{00}+ \dot{\pi}^0\right) - (b^2 F_{bb})\left(-\tfrac12 h^{ii} +(\partial_j \pi^j)\right)+ O(2)\\
    &= (yF_y-byF_{by})\left(-\Phi+ \dot{\pi}^0\right) - (b^2 F_{bb})\left(3\Psi +(\partial_j \pi^j)\right)+ O(2)\,.
\end{aligned}
\end{align}
Introducing the phonon potential $\partial_\ell\pi_L=\pi^\ell$ we have $\partial_j \pi^j = \nabla^2 \pi_L$ and $[\pi_L]=E^{-2}$.
We also recall the 4-velocity expansion
\begin{align}
\begin{aligned}
    &au^\mu=\delta_0^\mu \bigg(1 \!+\! \Big[\tfrac12 h^{00} -(\partial_j \pi^j)\Big]\!\bigg) +\tfrac{1}{2}\epsilon_{i j k}\,\epsilon^{\mu \alpha j k}\,\partial_\alpha\pi^i +O(2)\,.
\end{aligned}
\end{align}
On the other hand, the velocity of the fluid is $au^\mu =a(1+\dots, v^\ell)$ where $v^\ell$ is the peculiar velocity.
We also denote the velocity divergence $\theta=\nabla\cdot\vec{v}$ and the velocity scalar potential $v$ by demanding\footnote{Indices raised/lowered with $\eta^{\mu\nu}=(-+++)$} $v^\ell=\partial_\ell v$ i.e. $\nabla^2 v=\theta$.
Matching the two expressions we find
\begin{align}\label{eq:peculiar_velocity_via_phonons}
    v^\ell = \frac{1}{2}\epsilon_{i j k}\,\epsilon^{\ell \alpha j k}\,\partial_\alpha\pi^i = -\dot{\pi}^\ell\quad \Rightarrow \quad v = -\dot{\pi}_L\quad \text{where we define $\pi_L$ by}\quad \partial_\ell\pi_L=\pi^\ell
\end{align}
Dimensions are respected since $[\pi^\alpha]=E^{-1}$ and $[v^\ell]=E^0$ so that $[v]=E^{-1}$ and $[\pi_L]=E^{-2}$. 

The gauge-invariant comoving curvature perturbation $\mathcal{R}$, i.e. the 3-curvature of comoving spatial slices, is defined in terms of metric and matter perturbations
\begin{align}
    \mathcal{R}\equiv C-\tfrac13\nabla^2E+\mathcal{H}(B+v) \overset{\text{newt. gauge}}{=} -\Psi +\mathcal{H}v = -\Psi.
\end{align}
since in Newtonian gauge $C=-\Psi$ and $B=E=0$.
From \label{eq:peculiar_velocity_via_phonons} we can identify the comoving curvature perturbations in terms of metric potentials and phonon potential
\begin{equation}
    \mathcal{R} = -\Psi +\mathcal{H}v = -\Psi - \mathcal{H}\,\dot{\pi}_L\,.
\end{equation}


In other works, another gauge invariant quantity $\zeta$ is used in place of the comoving curvature $\mathcal{R}$.
In Newtonian gauge it is defined by\footnote{The second equality holds for separately conserved em tensors}. 
\begin{align}
    \zeta\overset{\text{newt gauge}}{=} -\Psi-\mathcal{H}\frac{\delta \rho}{\dot{\rho}} = -\Psi +\tfrac13 \frac{\bar{\rho}\delta }{\bar{\rho}+\bar{p}}.
\end{align}
A linear combination of the 00 and 0i Einstein equation gives the gauge-invariant Poisson equation
\begin{align}
    \nabla^2\Psi= 4\pi G a^2 \Big(\underbrace{\bar{\rho}\delta-3\mathcal{H}(\bar{\rho}+\bar{p})v}_{=\bar{\rho}\Delta}\Big)
\end{align}
Therefore
\begin{align}
    \zeta-\mathcal{R}=-\mathcal{H}\frac{\delta \rho}{\dot{\rho}}-\mathcal{H}v=\bar{\rho}\delta-3\mathcal{H}(\bar{\rho}+\bar{p})v = \frac{-k^2}{4\pi G a^2}\Psi \,\,\to 0\quad\text{as}\,\,k\to0
\end{align}
confirming the two quantities indeed coincide on large scales.


In Newtonian gauge \eqref{eq:newtonian_gauge_metric}, the quadratic lagrangian becomes 
{\small
\begin{align}
\begin{aligned}
    S^{(2)}=
    &\int d^4x\,\, a^4\,\bigg\{
    \tfrac12 (yF_y-bF_b) (\dot{\pi}^i)^2+ \tfrac12 b^2 F_{bb} (\partial_j \pi^j)^2\bigg\} -\tfrac12 a^4 y^2F_{yy} (\dot{\pi}^0)^2\\
    &\quad + a^4\,\,\bigg\{-3\Psi(\partial_j \pi^j)\left(\bar{b}F_b - \bar{b}^2 F_{bb}\right) 
    -\Phi(\partial_j \pi^j)\!\left(-2\bar{b}F_b + \bar{b}\bar{y}F_{by}\right) \bigg\}\,,
\end{aligned}
\end{align}}
where
{\small
\begin{align}
\begin{aligned}
    -\tfrac12 a^4y^2F_{yy}\big(\dot{\pi}^0\big)^2
    &=-\tfrac{1}{2}\tfrac{a^4}{y^2F_{yy}}\bigg[(byF_{by}-yF_y) (\partial_j\pi^j)+\bigg(\tfrac{1}{a^4}\!\!\int(\partial_j \pi^j)\tfrac{d}{d\tau}(a^4yF_y)\bigg)\bigg]^2\\
    &\quad+\tfrac{a^4}{y^2F_{yy}}\Big[\big(2{y}F_y - byF_{by}\big) 6\Psi+ \left({y}^2 F_{yy} - {y}F_y\right) 2\Phi\Big]\\
    &\qquad\qquad\cdot \bigg[(byF_{by}-yF_y) (\partial_j\pi^j)+\bigg(\tfrac{1}{a^4}\!\!\int(\partial_j \pi^j)\tfrac{d}{d\tau}(a^4yF_y)\bigg)\bigg]\,.
\end{aligned}
\end{align}}
Taking $\Phi=\Psi$ in the large scale limit and simplifying terms we finally get\todotag{write}
\begin{align}
    ...
\end{align}
The equation of motion for phonons will be something like
\begin{align}
    \ddot{\pi}_L=-\underbrace{c_s^2 k^2\pi_L}_{\to0}+\text{friction}
\end{align}
giving that $\ddot{\pi}_L\to 0$ as $k\to0$.
For the comoving curvature then
\begin{align}
    \dot{\mathcal{R}}=-\dot{\Psi}+\dot{\mathcal{H}}\,\dot{\pi}_L+\mathcal{H}\,\ddot{\pi}_L
\end{align}
and, upon using the 0i Einstein equation for $\dot{\Psi}+\mathcal{H}\Psi=...$, the vanishing of $\ddot{\pi}$ from the EoM of phonons might be the fastest way to predict that $\mathcal{R}$ is conserved on superhorizon scales.\todotag{Finish this argument}
