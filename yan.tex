\documentclass[a4paper,11pt]{article}
%\pdfoutput=1 

\usepackage{graphicx}  % needed for figures
\usepackage{dcolumn}   % needed for some tables
\usepackage{bm,relsize}        % for math
\usepackage{amssymb, amsmath}
\usepackage{textcomp}
\usepackage{wasysym}
\usepackage{slashed}
\usepackage{caption, subcaption}
\usepackage{multirow}
\usepackage{gensymb}
\usepackage{subcaption}
%\usepackage[table]{xcolor}
\usepackage{colortbl}
\usepackage{booktabs} % opzionale ma utile per tabelle più eleganti
\usepackage{tabularx}
\definecolor{headergray}{gray}{0.9}
\definecolor{rowgray}{gray}{0.97}


\usepackage[nolist, nohyperlinks]{acronym}


\usepackage{lipsum, color}
\usepackage[dvipsnames,svgnames,table]{xcolor}
\usepackage{braket}
\usepackage{jheppub} 
\usepackage{booktabs}
\usepackage{pdflscape}
\usepackage[utf8]{inputenc} 

\usepackage{booktabs}
\usepackage{tabularx}

\usepackage{mathrsfs}
\usepackage{bm,amssymb,slashed,graphicx,multirow,soul,mathtools,xspace,array,tikz,amsmath, gensymb}
\usetikzlibrary{patterns}
\usepackage{siunitx} 
\usepackage{float}   
\usepackage{cancel}
\allowdisplaybreaks
\usepackage{ bbold }
%\usepackage{subfigure}
\usepackage{caption,subcaption}
\usepackage{hyperref}
\usepackage{colortbl}
\usepackage{tcolorbox}

\usepackage{comment}

\usepackage[capitalise, english]{cleveref}

\definecolor{nicered}{rgb}{0.7,0.1,0.1}
\definecolor{nicegreen}{rgb}{0.1,0.5,0.1}
\definecolor{violet}{rgb}{0.7,0.3,0.3}
\hypersetup{colorlinks,citecolor= nicegreen,linkcolor= nicered}

\setcounter{tocdepth}{2}

\newcommand{\lp}{\left(}
\newcommand{\rp}{\right)}
\newcommand{\ov}{\overline}
\newcommand{\ds}{\displaystyle}
\newcommand{\g}{\gamma}
\newcommand{\be}{\begin{equation}}
\newcommand{\ee}{\end{equation}}

\newcommand{\eventname}{KM3-230213A}
\newcommand{\kmn}{KM3NeT}
\newcommand{\ic}{IceCube}

\newcommand{\nn}{\nonumber}
\newcommand{\TeV}{\si{\tera\electronvolt}}
\newcommand{\GeV}{\si{\giga\electronvolt}}
\newcommand{\PeV}{\si{\peta\electronvolt}}
\newcommand{\Br}{\text{Br}}

\newcommand {\vek}[1]{\mathbf{#1}}
\newcommand {\E}[1]{\times 10^{#1}}	% scientific exponent notation
\newcommand {\e}[1]{\mathrm{~#1}}       % units
\newcommand{\re}[0]{\mathrm{Re}\,}
\newcommand{\im}[0]{\mathrm{Im}\,}
\newcommand{\mc}[1]{\mathcal{#1}}
\newcommand{\dd}[0]{\mathrm{d}}
\newcommand{\eq}[1]{\begin{equation} #1 \end{equation}}
\newcommand{\beq}{\begin{equation} }
\newcommand{\eeq}{\end{equation}} 
\newcommand{\bi}{\begin{itemize} }
\newcommand{\ei}{\end{itemize} }
\newcommand{\EeV}{\mathrm{EeV}}
\newcommand{\TT}{\mathcal{T}}

\newcommand{\deriv}[2]{\frac{\partial #1}{\partial #2}}
\newcommand{\totalderiv}[2]{\frac{d #1}{d #2}}

\definecolor{Red}{rgb}{1.,0.,0.}
\definecolor{Grn}{rgb}{0.,0.75,0.}
\definecolor{Blu}{rgb}{0.,0.,1.}
\definecolor{Pink}{rgb}{1,0.08,0.58}
\newcommand{\Red}[1]{{\color{nicered}{#1}}}
\newcommand{\Grn}[1]{{\color{nicegreen}{#1}}}
\newcommand{\Blu}[1]{{\color{niceblue}{#1}}}        
   


\DeclareMathOperator{\diag}{diag}   
\let\Re\relax
\DeclareMathOperator{\Re}{Re}
\let\Im\relax
\DeclareMathOperator{\Im}{Im}
\DeclareMathOperator{\Tr}{Tr}
\newcommand{\lrpartial}{\negthickspace\stackrel{\leftrightarrow}{\partial}\negthickspace{}}
\newcommand{\lrPartial}{\negthickspace\stackrel{\leftrightarrow}{D}\negthickspace{}}

\usepackage[T1]{fontenc} % if needed

\usepackage{amsmath,amssymb,epsfig,color,slashed}
\allowdisplaybreaks  

\setcounter{MaxMatrixCols}{20}

\newcommand*{\I}{\mathrm{i}}

\newcommand\scalemath[2]{\scalebox{#1}{\mbox{\ensuremath{\displaystyle #2}}}}



\definecolor{verdino}{rgb}{0.66, 0.89, 0.63}

\bibliographystyle{JHEP}

\newcommand{\DR}[1]{{\color{blue}[DR: #1]}}
\newcommand{\CA}[1]{{\color{nicered}[CA: #1]}}



%============= PREAMBLE BY ANDREA ==================
%====================================================

%\usepackage{comment} % for \begin{comment} environment %already loaded above
%\usepackage{xspace} %already loaded above
\usepackage{todonotes} % for TODO tags, switch tags off with [disable] flag

\definecolor{AnswerColor}{RGB}{100,255,100} % green
\definecolor{NoteColor}{RGB}{255,255,170} % pale yellow
\definecolor{QuestionColor}{RGB}{255,220,150} % light orange
\definecolor{TodoColor}{RGB}{255,170,170}   % pale red

%-------------------- tags: answer, note, question, todo ----------------------
\newcommand{\notetag}[1]{
  {\setlength{\fboxsep}{1pt}   \todo[backgroundcolor=NoteColor, bordercolor=black, linecolor=black, size=\scriptsize]{#1}}}
\newcommand{\questiontag}[1]{
  {\setlength{\fboxsep}{1pt}   \todo[backgroundcolor=QuestionColor, bordercolor=black, linecolor=black, size=\scriptsize]{#1}}}
\newcommand{\todotag}[1]{
  {\setlength{\fboxsep}{1pt}  \todo[backgroundcolor=TodoColor, bordercolor=black, linecolor=black, size=\scriptsize]{#1}}}
\newcommand{\answertag}[1]{
  {\setlength{\fboxsep}{1pt}  \todo[backgroundcolor=AnswerColor, bordercolor=black, linecolor=black, size=\scriptsize]{#1}}}

\newcommand{\noteAC}[1]{
  {\setlength{\fboxsep}{1pt}\todo[backgroundcolor=NoteColor, bordercolor=black, linecolor=black, size=\scriptsize]{AC: #1}}}
\newcommand{\questionAC}[1]{
  {\setlength{\fboxsep}{1pt}\todo[backgroundcolor=QuestionColor, bordercolor=black, linecolor=black, size=\scriptsize]{AC: #1}}}
\newcommand{\todoAC}[1]{
  {\setlength{\fboxsep}{1pt}\todo[backgroundcolor=TodoColor, bordercolor=black, linecolor=black, size=\scriptsize]{AC: #1}}}
\newcommand{\answerAC}[1]{
  {\setlength{\fboxsep}{1pt}\todo[backgroundcolor=AnswerColor, bordercolor=black, linecolor=black, size=\scriptsize]{AC: #1}}}

%--------------- Colored cancel lines to strike through terms----------
\newcommand{\redcancel}[1]{\renewcommand{\CancelColor}{\color{red}}\cancel{#1}}
\newcommand{\bluecancel}[1]{\renewcommand{\CancelColor}{\color{blue}}\cancel{#1}}
\newcommand{\greencancel}[1]{\renewcommand{\CancelColor}{\color{green}}\cancel{#1}}
\newcommand{\graycancel}[1]{\renewcommand{\CancelColor}{\color{gray}}\cancel{#1}}
\newcommand{\purplecancel}[1]{\renewcommand{\CancelColor}{\color{purple}}\cancel{#1}}
%=================================================



\begin{document} 

\begin{acronym}
\acro{IC}{IceCube}
\end{acronym}

%\preprint{}



%%%%%%%%%%%%% FRONT %%%%%%%%%%%%%%%%%%%%%%%%%%%%%%%%%

\title{Accelerating the Universe with non-barotropic fluids}

\author[a]{Andrea Clini,}
\author[a]{Michele Redi,}
\author[b]{Diego Redigolo,}
\author[b]{Lorenzo La Penna,}

\affiliation[a]{Dipartimento di Fisica “Aldo Pontremoli”, Universit\`a degli Studi di Milano,
Via Celoria 16, 20133 Milan, Italy}
\affiliation[b]{INFN Sezione di Firenze, Via G. Sansone 1, I-50019 Sesto Fiorentino, Italy}
        
\date{\today} 


\abstract{Long-lasting accelerated expansion cannot be supported by perfect barotropic fluid where pressure fluctuations are fully determined by the energy density ones. The manifestation of this obstruction is that the adiabatic sound speed is fully determined by the equation of state so that requiring a negative equation of state to enforce acceleration unavoidably lead to gradient instabilities. We show that non-barotropic perfect fluids allow for accelerated expansion by decoupling the speed of sound from the equation of state similarly to what happens in generalized quintessence models. fluid stability requires: $w>-1$ and $0<c_s^2<1$. The former seems intimately connected with the absence of dissipation at least at leading order in the derivative expansion. Non-barotropic perfect fluids are somewhat non-generic from the ultraviolet perspective }

\maketitle


\section{Introduction}
Which fluid allows for accelerated expansion? Assuming a single fluid dominates the energy density of the Universe we can relate the acceleratio of the universe to the equation of state of the fluid that drives it
\begin{equation}
    \frac{\ddot{a}}{a}=-\frac{4\pi G_N\rho}{3}(1+3w)\,,
\end{equation}
where $w=p/\rho$ is the equation of state of the so-called dark energy. This equation tells us the well known thing that having accelerated expansion implies $w<-1/3$. We also need to check that the accelerating background is stable which amounts to exclude the presence of  instabilities in the fluctuations around it. 

Moreover, given the little we know about the nature of dark matter we might want to look for a overarching framework describing the behavior of both dark matter and dark energy in terms of the properties that control their interaction with Einstein gravity. This exercise has of course also phenomenological relevance because through their gravitational imprints in the large scale structure of the Universe we could be able to observe the properties of these misterious fluids.  

Very much in the spirit of Ref.~\cite{Hu:1998kj} we would like to be guided by symmetries to construct this general fluid. Contrary to older approaches we will consistently make use of the lagrangian effective field theory framework to describe fluids. 


%%%%%%%%%%%%%%%%%%%%% BODY %%%%%%%%%%%%%%%%%%%%%%%%%%%%%%%%%


%==============================================
\section{Dark Energy as a single fluid}
%=============================================
Here we summarize some key results about fluid in the literature.


%---------------------------------------------------------
\subsection{Barotropic perfect fluid can't be dark energy}
%-------------------------------------------------------
A relativistic barotropic perfect fluid is described by three scalar fields $\phi^I(x)$ with $I=1,2,3$ which label the fluid elements~\cite{Dubovsky:2005xd}. The lagrangian should be invariant under volume-preserving diffeos
\begin{equation}
\phi^I\to f^I(\phi)\quad ,\quad \det\frac{\partial f}{\partial \phi^I}=1\ .   
\end{equation}
We define
\begin{equation}
    B^{IJ}\equiv g^{\mu\nu}\partial_\mu\phi^I\partial_\nu\phi^J\quad,\quad b=\sqrt{\det B_{IJ}}\, ,   \label{eq:b}
\end{equation}
which is invariant under volume-preserving diffeos. The physical interpretaion of $b$ can be understood in terms of the conservation of the comoving volume current $b=\sqrt{-J^\mu J_\mu}$, where 
\begin{equation}
    J^\mu_n=\frac{1}{6\sqrt{-g}}\epsilon^{\mu\nu\rho\sigma}\epsilon_{IJK}\partial_\nu\phi^I\partial_\rho\phi^I\partial_\sigma\phi^K\quad \rm{s.t.}\quad  \nabla_\mu J^\mu=0  \,.  
\end{equation}
This allows to define the flow velocity orthogonal to the constant $\Phi^I$ surfaces
\begin{equation}
J^\mu_n=b u^\mu\quad \rm{s.t}\quad u^\mu u_\mu=-1\ ,
\end{equation}
where the definition of the current makes it manifest that $b$ is the comoving number density.
In the barotropic case $b$ also coincides with the entropy density, and it is conserved in the absence of dissipation. The entropy per particle $\sigma=s/n$ is exactly constant in this case, and the entropy is directly proportional to the number density $s\propto n$.

The action is then specified by a single function
\begin{equation}
S=\int d^4x\sqrt{-g} F(b)\ .    
\end{equation}
The function $F(b)$ is in one to one correspondence with the equation of state, which also fixes the behavior of fluctuations. Explicitly we can derive the background thermodynamic by writing the stress energy tensor and matching it to the one of a perfect fluid 
\begin{equation}
    T_{\mu\nu}=(p+\rho)u_\mu u_\nu+p g_{\mu\nu}=F_{IJ} \partial_\mu\phi^I \partial_\nu\phi^I-g_{\mu\nu} F\ ,
\end{equation}
This results in
\begin{equation}
    \rho=-F\,,\quad p=F-bF_b\,, \quad \rho+p=-bF_b\,,
\end{equation}
and the equation of state 
\begin{equation}
    w\equiv\frac{p}{\rho}=-1+\frac{b F_b}{F}\ .
\end{equation}
Notice that at the level of the background the barotropic perfect fluid is fully specified by a single function which specifies its equation of state.  

Explicitly we can write the lagrangian at quadratic order in the fluctuation around a fluid background configuration $\phi^I=x^i+\pi^I$
\begin{equation}
S_2^{\rm{baro}}=\int d^4x\sqrt{-g}\,\,\left(\frac{p+\rho}{2}\right)\left[\dot\pi^2-c_s^2(\nabla\pi)^2\right]\ ,
\end{equation}
where\footnote{Coupling a barotropic fluid to FRW the conservation of the entropy current implies $\nabla_\mu J^{\mu}=\dot{b}+3Hb=0$.} 
\begin{equation}
 c_s^2=\frac{b F_{bb}}{ F_b}=\frac{d p}{d\rho}=\frac{\dot{p}}{\dot{\rho}}=w-\frac{\dot{w}}{3H(1+w)}\,,  \label{eq:adiabatic}
\end{equation}
and it is fully determined by the equation of state as expected for a barotropic fluid. 
Notice that the absence of ghosts, and gradient instabilities and superluminal modes imply respectively 
\begin{equation}
w>-1\,\qquad 0<c_s^2\leq1\ .
\end{equation}
Moreover, using $c_s^2(w)$ derived from \eqref{eq:adiabatic} we can rewrite the absence of gradient instabilities as a further constraint on the equation of state 
\begin{equation}
w-\frac{\dot{w}}{3H(1+w)}>0\ , \label{eq:nogradbaro}
\end{equation}
which parametrizing $\dot w= \beta_w H w$ becomes 
\begin{equation}
 w\left(1-\frac{\beta_w}{3(1+w)}\right)>0\ .   
\end{equation}
Notice that, in order to have $-1<w<-1/3$ and hence accelerated expansion, stability requires $\beta_w>3(1+w)\sim\mathcal{O}(1)$ which then implies that $w$ must evolve rapidly, making a long accelerating phase impossible. Indeed going back to Eq.~\eqref{eq:nogradbaro} and assuming constant equation of state we get $w>0$ which is incompatible with accelerated expansion. 


As usual from the conservation of the stress tensor in a perturbed FRW background we can derive the equation controlling the behavior of the fluctuations of energy density and pressure: energy conservation gives the continuity equation and momentum conservation the Euler equation. Crucially for a barotropic fluid 
\begin{equation}
\delta p=c_s^2\delta\rho\,,
\end{equation}
the above conservation equations heavily simplify and eventually result in the second order equation for the density contrast $\delta=\delta\rho/\rho$:\todotag{rewrite according to convention in appendix}
\begin{equation}
\ddot{\delta\rho}+3H(1+c_s^2))\dot{\delta\rho}+\left[\frac{c_s^2k^2}{a^2}-4\pi G(1+w)\rho\right]\delta\rho=(1+w)\left[\ddot{\phi}+6H\dot{\phi}\right]\ .
\end{equation}
Now for $k^2/a^2\gg H^2$ and $\dot{\phi}\approx 0$ the equation reduces to 
\begin{equation}
\ddot{\delta\rho}+3H(1+c_s^2)\dot{\delta\rho}+\left[\frac{c_s^2k^2}{a^2}-4\pi G(1+w)(1+3c_s^2)\rho+3H\dot c_s^2\right]\delta\rho=0\ .
\end{equation}
and we recognize the usual competition between pressure support (for $c_s^2>0$) and gravitational instability. The barotropicity of the fluid implies that the adiabatic sound speed completely fixes the pressure support. 

\subsection{Non-barotropic perfect fluid}
To describe a fluid with entropy we can follow \cite{Dubovsky:2011sj,Ballesteros:2016kdx} and just add an extra scalar $\Phi^0$ to the construction above which enjoys a shift symmetry. 
\begin{equation}
\Phi^0\to\Phi^0+c\label{eq:shifttime}
\end{equation}
which allow us to define a new invariant 
\begin{equation}
y=u^\mu\partial_\mu\Phi^0
\end{equation}
which plays the role of the temperature or the chemical potential. This quantity will control the entropy per particle independently on the particle number density which is always controlled by $b$ defined in Eq.~\eqref{eq:b}.

The action is now described by a single function of two scalar quantities
\begin{equation}
S=\int d^4 x\sqrt{-g} F(b,y)\ .
\end{equation}
From this we can derive the stress energy tensor
\begin{equation}
T_{\mu\nu}=(-b F_b+y F_y )\,u_\mu u_\nu
           + (F - b F_b)\,g_{\mu\nu}\ ,
\end{equation}
which matching to the perfect fluid expression gives
\begin{equation}\label{eq:rho_pressure_exp_nonbaro}
\rho = - F+yF_y , \qquad p = F- b F_b \ .
\end{equation}
so that 
\begin{equation}
w=\frac{ F- b F_b}{- F+yF_y}\,. 
\end{equation}

Notice that in this case the presence of $y$ makes the entropy per particle not fixed by the equation of state but still conserved along the flow lines as expected in the absence of dissipation. In the EFT language the entropy per particle current is nothing else than the Noether current associated to the shift symmetry in Eq.~\eqref{eq:shifttime}
\begin{equation}
J_s^{\mu}=F_y u_\mu\quad \rm{s.t.}\quad \nabla_\mu J^\mu_s=0\ .
\end{equation}
The entropy per particle is not constant in this case, and it can be identified as
\begin{align}
    \sigma=\frac{s}{n}=\frac{F_y}{b}\ .
\end{align}
The adiabatic sound speed is then by definition
\begin{equation}
c_b^2:=\frac{dp}{d\rho}_{\mid\sigma}=w+\rho\frac{dw}{d\rho}_{\mid\sigma}=\frac{-b^2F_{bb} F_{yy}+(F_y-bF_{yb})^2}{F_{yy}\,(yF_y-bF_b)} = \frac{b\,p_b \,y\rho_y + (y\,p_y)^2}{y \rho_y (\rho+p)}\,.
\end{equation}
Finally, identifying the system temperature with $y=T$, then \eqref{eq:rho_pressure_exp_nonbaro} implies the function $F$ is minus the Helmholtz free energy density 
\begin{equation}
    F=-(\rho-Ts)\equiv-\mathfrak{a}\,.
\end{equation}

Explicitly we can write the lagrangian at quadratic order in the fluctuation around a fluid background configuration $\phi^0=t+\pi^0$ and $\phi^I=x^i+\pi^I$
\begin{equation}
S_2=\int d^4x\sqrt{-g}\left(\frac{p+\rho}{2}\right)\left[\dot\pi^2-\mathcal{C}_s^2(\nabla\pi)^2-2\mathcal{M}\dot{\pi}_0 (\nabla\!\!\cdot\!\pi)+\mathcal{A} \dot{\pi}_0^2\right]\ , 
\end{equation}
the sound speed is now 
\begin{equation}
\mathcal{C} = -\frac{b^2F_{bb}}{(yF_y-bF_b)}\,.
\end{equation}
The mixing of the entropy mode with the phonons is given by 
\begin{equation}
\mathcal{M} = \frac{(yF_y-byF_{by})}{(yF_y-bF_b)}.
\end{equation}
and the inertia of the entropy mode is
\begin{equation}
\mathcal{A}= \frac{y^2 F_{yy}}{y F_y-b F_b}\,.
\end{equation}
Crucially in this lagrangian only $p+\rho>0$ is necessary to avoid ghost instabilities.
In case $\rho+p>0$, thermodynamic stability also demands $\mathcal{A}>0$, however at this level the speed of sound $c_s^2$ can be negative.


The entropy mode is non propagating and can be integrated out as a constraint to get 
\begin{equation}
S_2=\int d^4xd^4x\sqrt{-g}\left(\frac{p+\rho}{2}\right)\left[\dot\pi^2-c_s^2(\nabla\pi)^2\right]\ , 
\end{equation}
where
\begin{equation}
    c_s^2:=\mathcal{C}_s^2+\frac{\mathcal{M}^2}{\mathcal{A}}\,.\;\label{eq:csnonbaro} 
\end{equation}
It is immediate to check the speed of sound $c_s^2$, defined as the coefficient of the fluctuations, always matches the thermodynamical definition of \emph{adiabatic sound speed} $c_b^2:= \frac{\partial p}{\partial \rho}_{\mid \sigma}$.
The entropy contribution to the speed of sound is always positive in \emph{physical} situations since
\begin{equation}
    \mathcal{A}= \frac{T\, \frac{\partial\rho }{\partial T}}{\rho+p}\geq 0 \quad\text{provided}\quad
    \begin{cases}
        \rho+p\geq 0 \quad(\text{no ghosts}),\\
        \frac{\partial\rho }{\partial T} \geq 0 \quad(\text{positive heat capacity}).
    \end{cases} 
\end{equation}
Now $c_s^2$ is actually an independent function of $c_a^2$. So that requiring the absence of gradient instabilities does not forbid accelerated expansion which is only constrained by the absence of ghosts. As a consequnce the non-barotropic fluid EFT can span the whole space of 
\begin{equation}
w>-1\quad 0<c_s^2\leq1\ .    
\end{equation}

In order to understand non barotropic fluids it is important to write the pressure fluctuations. These can be written in two ways 
\begin{align}
\delta p &=c_s^2\delta\rho+\Gamma\delta \sigma\ ,
\end{align}
where we defined $c_s^2=\partial p/\partial \rho\vert_\sigma$ is exactly the as the speed of sound appearing in the phonon Lagrangian defined in Eq.~\eqref{eq:csnonbaro}. In addition to the speed of sound the pressure fluctuation receive contribution from the fluctuation of the entropy mode $\Gamma=\partial p/\partial \sigma\vert_\rho$. 
Now we can write the continuity and Euler equation (see appendix).

Since the entropy is conserved along the flow we can write the equation in comoving gauge $(\delta p=0)$ and eliminate the non-propagating entropy fluctuation to get an expression for the sound speed similar to the one in 2.34. {\bf I would like to see the equations for the density constrast with $\Gamma$} {\bf I would like to do the gauging of the fluid with gravity showing that the phonons are identified with the adiabatic curvature mode a' la Weinberg and hence phonon scattering satisfies soft theorems as well as the adibatic curvature mode as it is well known. Don't know if it in this language.}


\subsection{Scalar fluids}
Here I will match the scalar fluids to the fluid EFTs developed so far. This will allow us to show in which sense the fluid EFT remains more general.

\paragraph{Quintessence} is typically defined as a scalar with a canonical kinetic term $X=-g^{\mu\nu}\partial_\mu\phi\partial_\nu\phi$, so that $\partial_\mu\phi=\sqrt{X}u_\mu$ with $u_\mu u^\mu=-1$, and an arbitrary potential $V(\phi)$
\begin{equation}
S=\int d^4x\sqrt{-g}\left[\tfrac{1}{2}X-V(\phi)\right]
\end{equation}
I can map this scalar theory into the non-barotropic fluid EFT by identifying...


%======================================================================
\section{Adding dark matter}
%====================================================================

Let us now consider the lagrangian for a two fluid system where the two fluids interact only through gravity 
\begin{equation}
S_2=\sum_{A=1,2}\int d^4x\sqrt{-g}\left(\frac{p_A+\rho_A}{2}\right)\left[\dot\pi_A^2-\mathcal{C}_{s,A}^2(\nabla\pi_A)^2+2\mathcal{M}_A\dot{\pi}_{0,A} (\nabla\!\!\cdot\!\pi_A)+\mathcal{A}_A \dot{\pi}_{0,A}^2\right]\ , 
\end{equation}   
In this setup I can define the sum of the two fluids and their difference. Let us define 
\begin{align}
&\pi_{\rm{tot}}=\frac{(\rho_A+p_A)\pi_A+(\rho_B+p_B)\pi_B}{\rho_A+p_A+\rho_B+p_B}\\
&S\sim\frac{\rho^B_y}{f^B_{yy}} \pi^0_A-\frac{\rho_y^A}{f^A_{yy}} \pi^0_B
\end{align}
The first is the adiabatic mode again while the second (or a similar expression) should encode the relative entropy fluctuations which are now propagating! {\bf this is the idea but it should be done with more care not sure I did the diagonalization correctly}
\section{Introducing bulk viscosity (to be continued)}
The simple example one could think of to construct a dissipative system is to start with two scalar fluids in the UV
\begin{equation}
S=\int d^4x\left[\frac{1}{2}(\partial\phi)^2+\frac{1}{2}(\partial\chi)^2+\frac{m_\chi^2}{2}\chi^2+ y \chi \psi^2+ M_\psi\psi^2+ \epsilon\partial\phi \partial\chi\right]    
\end{equation}
Where we wrote an interaction between the two scalars that respect the shift symmetry and we assume the mass hierarchy $m_\chi>2 M_\psi$. As we will see the presence of fermions is required to give to the heavy scalar $\chi$ a finite width.  
\begin{equation}
\Gamma_\chi\approx\frac{y^2}{8\pi}m_\chi\ ,
\end{equation}
Since we want the evolution of the scalar fluids we need to go to the SK formalism~\cite{Crossley:2015evo,Liu:2018kfw} putting the theory above on a SK time contour. In SK every field is doubled because $\phi_{\pm}, \chi_{\pm},\psi_{\pm}$ encode the time on the forward/backward branch. This is unavoidable to distinguish retarded and advanced response. The SK actions is
\begin{equation}
S_{\rm{SK}}=S[\phi_+,\chi_+]-S[\phi_-,\chi_-]    
\end{equation}
Now I can rotate to the SK variables
\begin{align}
&\phi_{p,m}=\frac{\phi_+\pm \phi_-}{2}
&\chi_{p,m}=\frac{\chi_+\pm \chi_-}{2}
\end{align}
The SK action for the heavy sector is
\begin{equation}
\begin{split}
S^{\rm{heavy}}_{\rm SK}
=
&\int d^4x\left\{\chi_a (\Box - M^2) \chi_r
+
\bar{\psi}_a (i\slashed{\partial}-m_\psi)\psi_r
+
\bar{\psi}_r (i\slashed{\partial}-m_\psi)\psi_a
]\right.
\\
&\left.+y[
\chi_a \bar{\psi}_r \psi_r
+\chi_r(\bar{\psi}_a \psi_r + \bar{\psi}_r \psi_a)\right\}
\end{split}
\end{equation}


%%%%%%%%%%%%%%%%%% APPENDIX %%%%%%%%%%%%%%%%%%%%%%%%%%%%%%%%%

\appendix 
%--------------------------------------
%=======================================
\section{Notations, conventions \& cosmology}
%=======================================
%---------------------------------------

We mainly follow the standard notation from \cite{MaBertschinger_CosmologicalPerturbationTheorySynchronousConformalNewtonianGauges_1995,Baumann_Cosmology_2022}.
The metric has signature $-+++$.
The conformal time is denoted by $\tau$ and differentiation with respect to it is denoted by $\dot{f}$.
The proper time of a comoving observer is denoted by $t$, related to conformal time by $dt=a \,d\tau$, and differentiation with respect to it is denoted by $f'$.

We parametrize the general perturbation to a homogeneous isotropic FLRW metric as
{\small\begin{equation} \label{eq:metric_general_form_svt}
-\mathrm{d} s^{2}:=g_{\mu\nu}dx^\mu dx^\nu=a^{2}(\tau)\left[-(1+2 A) \,\mathrm{d} \tau^{2}+2 B_{i} \mathrm{~d} x^{i} \mathrm{~d} \tau+\left((1+2C)\delta_{ij}+2E_{ij}\right) \mathrm{d} x^{i} \,\mathrm{~d} x^{j}\right],
\end{equation}}
with SVT decomposition
{\small\begin{equation*}
\underbrace{A,\, C}_{\text {scalar }}, \quad
B_{i}=\underbrace{\partial_{i} {B}}_{\text {scalar }}+\underbrace{\hat{B}_{i}}_{\text {vector }},\quad E_{i j}=\underbrace{\partial_{\langle i} \partial_{j\rangle} {E}}_{\text {scalar }}+\underbrace{\partial_{(i} \hat{E}_{j)}}_{\text {vector }}+\underbrace{\tilde{E}_{i j}}_{\text {tensor }},
\end{equation*}}
where we use the shorthand
{\small\begin{align}\label{eq:convention_svt_decomposition}
\partial_{\langle i} \partial_{j\rangle} {E} \equiv\left(\partial_{i} \partial_{j}-\frac{1}{3} \delta_{i j} \nabla^{2}\right) \tilde{E},\qquad
\partial_{(i} \hat{E}_{j)}\equiv \frac{1}{2}\left(\partial_{i} \hat{E}_{j}+\partial_{j} \hat{E}_{i}\right).
\end{align}}
%
The energy-momentum tensor of a perturbed perfect fluid is
\begin{align}\label{eq:em_tensor_fluid_perturbed}
    T^\mu_\nu= u^\mu u_\nu (\rho+p) + p \delta^\mu_\nu + \Sigma^\mu_\nu
\end{align}
The 4-velocity $u^\mu$ and the peculiar velocity $v^i$, split into scalar and vector part, are\footnote{Fourier transform convention $f(k)=\int dx e^{-ikx}f(x)$.}
{\small\begin{align}\label{eq:velocity_decomposition}
\begin{aligned}
    u^\mu=\frac{1}{a}\left(1-A,\,v^i\right),\quad u_\mu=a\left(-(1+A),\,v_i\right),\quad v^i=\delta^{ij}v_j,\\[2pt]
    v_i=\partial_i v + \hat{v}_i, \quad \theta= \partial_i v^i = ik_i v^i = -k^2 v, \quad \hat{v}^i=(\nabla\times \omega)^i
\end{aligned}\end{align}}
The (symmetric) anisotropic stress $\Sigma^\mu_\nu$ is traceless and orthogonal to the 4-velocity, and the remaining DoFs are again decomposed into scalar, vector and tensor part
{\small\begin{align}\label{eq:anisotropic_stress_decomposition}
\begin{aligned}
    &\Sigma^\mu_\mu=0,\quad \Sigma^\mu_\nu u_\mu=\Sigma^\mu_\nu u^\nu=0, \quad \Rightarrow\quad \Sigma^0_0=\Sigma^0_i=\Sigma^i_0=0, \quad \Sigma^i_i=0, \,,\\[3pt]
    &\Sigma^i_j=\Sigma^i_j =\Sigma_{ij},\quad
    \Sigma_{i j}=\underbrace{\partial_{\langle i} \partial_{j\rangle} \Sigma}_{\text {scalar }}+\underbrace{\partial_{(i} \hat{\Sigma}_{j)}}_{\text {vector }}+\underbrace{\tilde{\Sigma}_{i j}}_{\text {tensor }}.
\end{aligned}
\end{align}}
The energy-momentum tensor is finally decomposed as
\begin{align}\label{eq:energy_momentum_tensor_decomposition}
\begin{aligned}
    T_{0}^{0} & =-(\bar{\rho}+\delta \rho), \\
    T_{i}^{0} & =(\bar{\rho}+\bar{p}) \,v_{i}=-T_{\,0}^{i}, \\
    T_{j}^{i} & =(\bar{p}+\delta p) \delta_{j}^{i}+\Sigma_{j}^{i}, \quad \Sigma_{i}^{i}=0.
\end{aligned}
\end{align}


%===========================================================================
\subsection{Formulas in different gauges}
%============================================================================


%----------------------------------------------------------------------
\subsubsection{Conformal Newtonian Gauge}
%----------------------------------------------------------------------

In this gauge, vector and tensor modes are excluded from the beginning $\hat{B}=\hat{E}=\bar{E}=0$, and the residual gauge freedom is used to kill the non-diagonal scalar terms $\tilde{B}=\tilde{E}=0$.
The metric is conventionally\footnote{Beware of different conventions in the literature, swapping $\phi \leftrightarrow \psi$ or their signs. Here we adopt the convention of Ma and BertschingerBaumann MaBertschinger, Baumann \cite{Baumann_Cosmology_2022} and\texttt{CLASS} \cite{Lesgourgues_CosmicLinearAnisotropySolvingSystemCLASSIOverview_2011}, so that $\psi\approx \phi$ in the absence of anisotropic stress.
Dodelson \cite{DodelsonSchmidt_ModernCosmology_2025} uses $\phi\mapsto -\phi$. Weinberg \cite{Weinberg_AdiabaticModesCosmology_2003,Weinberg_Cosmology_2008} and Pajer swap $\phi\leftrightarrow \psi$.} written by calling $A=\psi$, $C=-\phi$
\begin{align}\label{eq:newtonian_gauge_metric}
\begin{aligned}
    &g_{\mu\nu}=a^2\begin{pmatrix}
        -(1+2\psi) & 0 \\
        0 & (1-2\phi)\delta_{ij}
    \end{pmatrix}\equiv a^2(\eta_{\mu\nu}+h_{\mu\nu})\\[5pt]
    &\quad \Rightarrow \quad h^{00}=h_{00}=-2\psi,\quad h^{ii}=h_{ii}=-6\phi\,.
\end{aligned}
\end{align}
%
In the Newtonian limit $\psi$ plays the role of the gravitational potential, while $\phi$ is a local perturbation to the scale factor.
In the absence of anisotropic stress, Einstein equations imply  $\psi\simeq \phi$.
%
The Christoffel symbols in this gauge
\begin{align}\label{eq:christoffel_newtonian_gauge}
    \begin{aligned}[t]
    \Gamma_{00}^{0} & =\mathcal{H}+\dot\psi,  \\
    \Gamma_{0 i}^{0} & =\partial_{i} \psi,  \\
    \Gamma_{00}^{i} & = \partial_{i} \psi,
    \end{aligned}
    \qquad
    \begin{aligned}[t]
    \Gamma_{i j}^{0} & =\mathcal{H} \delta_{i j}-\left[\dot\phi+2 \mathcal{H}(\phi+\psi)\right] \delta_{i j}, \\
    \Gamma_{j 0}^{i} & =\mathcal{H} \delta_{ij}-\dot\phi \delta_{ij}, \\
    \Gamma_{j k}^{i} & = -\partial_j \phi\, \delta_{ki} - \partial_k \phi\,\delta_{ji} + \partial_i \phi\,\delta_{jk}.
    \end{aligned}
\end{align}
The perturbed Einstein equations in this gauge, with (i,j)-component already split int trace and traceless part and $s$ denotes different species, are
\begin{align}
\label{eq:EE_00_poisson}
k^{2} \phi+3 \mathcal{H}(\dot{\phi}+\mathcal{H} \psi)&=-4 \pi G a^{2} \sum_{s} \bar{\rho}_{s} \,\delta_{s},
\\
%
k^{2}(\dot{\phi}+\mathcal{H} \psi)&=4 \pi G a^{2} \sum_{s}\left(\bar{\rho}_{s}+\bar{P}_{s}\right) \theta_{s},  \label{eq:EE_0j}
\\
%
\ddot{\phi}+\mathcal{H}(2 \dot{\phi}+\dot{\psi})+\left(2 \frac{\ddot{a}}{a}-\mathcal{H}^{2}\right) \psi+\frac{k^{2}}{3}(\phi-\psi)&=4 \pi G a^{2} \sum_{s} \delta P_{s}, \label{eq:EE_ij_diagonal}
\\
%
k^{2}(\phi-\psi)&=12 \pi G a^{2} \sum_{s}\left(\hat{k}^{j} \hat{k}_{i}-\frac{1}{3}\delta_{i}^{j}\right) \Sigma_{s\,j}^{i}. \label{eq:EE_ij_off_diagonal}
\end{align}
A linear combination of the 00 and 0i Einstein equation gives the gauge-invariant Poisson equation
\begin{align}\label{eq:poisson_gauge_invariant}
    \nabla^2\phi= 4\pi G a^2 \Big(\underbrace{\bar{\rho}\delta-3\mathcal{H}(\bar{\rho}+\bar{p})v}_{=\bar{\rho}\Delta}\Big), \quad \text{where}\quad \Delta:=\delta +\frac{\dot{\bar{\rho}}}{\bar{\rho}}v.
\end{align}
% 
The continuity and Euler equations $\nabla_\mu T^\mu_\nu=0$ for a \emph{separately conserved} species are
{\small\begin{align}
    &\dot{\bar{\rho}} +3\mathcal{H}(\bar{\rho}+\bar{p})=0\,, \label{eq:backround_continuity}\\
    &\dot{\delta\rho}= -3\mathcal{H}(\delta\rho+\delta P) + (\bar{\rho}+\bar{P})\big(3\dot{\phi}-\partial_j v^j \big)\,,\label{eq:perturbed_continuity_drho}\\
    &\dot{v}_j+ \mathcal{H}v_j-3\mathcal{H}\frac{\dot{\bar{p}}}{\dot{\bar{\rho}}}v_j+\frac{\partial_j\delta p}{(\bar{\rho}+\bar{p})}+\partial_j\psi{\color{blue}+\frac{\partial_\ell \Sigma^{\ell}_{j}}{(\bar{\rho}+\bar{p})}}=0\,,\label{eq:euler_vector_form}
\end{align}}
Introducing the density contrast $\delta=\delta\rho/\bar{\rho}$, the velocity divergence \eqref{eq:velocity_decomposition} and the scalar anisotropic stress \eqref{eq:anisotropic_stress_decomposition}, and taking the divergence of the Euler equation, we get
\begin{align}
    &\dot{\delta}= 3\mathcal{H}\left(\frac{\bar{p}}{\bar{\rho}}-\frac{\delta p}{\delta\rho}\right)\delta + \left(1+\frac{\bar{p}}{\bar{\rho}}\right)3\dot{\phi} - \left(1+\frac{\bar{p}}{\bar{\rho}}\right)\theta\, \label{eq:continuity_overdensity_form}\\
    &\dot{\theta}+ \mathcal{H}\theta-3\mathcal{H}\frac{\dot{\bar{p}}}{\dot{\bar{\rho}}}\theta+\frac{\nabla^2\delta P}{(\bar{\rho}+\bar{P})}+\nabla^2\psi\,\,{\color{blue}+\frac{2}{3}\frac{\nabla^4 \Sigma}{(\bar{\rho}+\bar{p})}}=0\,.\label{eq:euler_divergence_form}
\end{align}
These are combined into a 2nd order equation for overdensities\todotag{Check 2nd order eq are correct}
\begin{align}
\begin{aligned}
    \ddot{\delta\rho}=&-3\big(\dot{\mathcal{H}}+4\mathcal{H}^2\big)\left(1+\frac{\delta p}{\delta\rho}\right)\delta\rho
    -\mathcal{H}(7\dot{\delta\rho}+3\dot{\delta p})+\nabla^2\delta P +(\bar{\rho}+\bar{P})\nabla^2\psi\\
    &+3(\bar{\rho}+\bar{p})\left[\left(\mathcal{H}+\frac{\dot{\bar{p}}}{\bar{\rho}+\bar{p}}\right)\dot{\phi}+\ddot{\phi}\right] {\color{blue}+ \frac{2}{3}\nabla^4\Sigma}\,.
\end{aligned}
\end{align}
Equivalently, in terms of the overdensity $\delta$ we have\todotag{Check 2nd order eq are correct}
\begin{align}
\begin{aligned}
    \ddot{\delta}=&-\dot{\delta}\mathcal{H}\left(1-6\frac{\bar{p}}{\bar{\rho}}\right)
    +\delta\left[3(\dot{\mathcal{H}}+4\mathcal{H}^2)\left(\frac{\bar{p}}{\bar{\rho}}-\frac{\delta p}{\delta\rho}\right)+3\mathcal{H}\frac{\dot{\bar{p}}}{\bar{\rho}}\right]
    +\frac{\nabla^2\delta p}{\bar{\rho}}\\
    & +\left(1+\frac{\bar{p}}{\bar{\rho}}\right)\nabla^2\psi
    +3\left(1+\frac{\bar{p}}{\bar{\rho}}\right)\left[\left(\mathcal{H}+\frac{\dot{\bar{p}}}{\bar{\rho}+\bar{p}}\right)\dot{\phi}+\ddot{\phi}\right] {\color{blue}+ \frac{2}{3\bar{\rho}}\nabla^4\Sigma}\,.
\end{aligned}
\end{align}


%----------------------------------------------------------------------
\subsubsection{Synchronous Gauge}
%----------------------------------------------------------------------

If ever needed we collect here the relevant expressions for the synchronous gauge\dots
In particular recall in this gauge there is some residual gauge freedom that can be fixed by demanding the peculiar velocity of a \emph{cold} ($w\simeq 0$) species to vanish.\todotag{Finish this section}


%----------------------------------------------------------------------
\subsubsection{ADM Gauge}
%----------------------------------------------------------------------

In order to explicitate the comoving curvature perurbations and later match them to phonons, it is useful to work in the ADM gauge, where the metric is written as

% ADM in conformal time
\begin{align}
ds^2
=
a^2(\tau)\Big[
-\mathcal{N}^2(\tau,\mathbf{x})\,d\tau^2
+
h_{ij}(\tau,\mathbf{x})\bigl(d x^i+\mathcal{N}^i(\tau,\mathbf{x})\,d\tau\bigr)\bigl(d x^j+\mathcal{N}^j(\tau,\mathbf{x})\,d\tau\bigr)
\Big].
\end{align}

% Comoving gauge parameterization of the spatial metric
\begin{align}
h_{ij}(\tau,\mathbf{x})
=
e^{2\zeta(\tau,\mathbf{x})}\,\big[\exp(\gamma(\tau,\mathbf{x}))\big]_{ij},
\qquad
\gamma^i{}_i=0,
\qquad
\partial_i\gamma^i{}_j=0.
\end{align}

% Matrix exponential definition
\begin{align}
\big[\exp(\gamma)\big]_{ij}
\equiv
\delta_{ij}
+
\gamma_{ij}
+
\frac{1}{2}\,\gamma_{ik}\gamma_{kj}
+
\frac{1}{3!}\,\gamma_{ik}\gamma_{k\ell}\gamma_{\ell j}
+\cdots .
\end{align}

% Tensorless limit
\begin{align}
\gamma_{ij}=0
\quad\Longrightarrow\quad
ds^2_{\rm spatial}=a^2(\tau)\,e^{2\zeta(\tau,\mathbf{x})}\,\delta_{ij}\,dx^i dx^j.
\end{align}

% Your Newtonian-gauge convention
\begin{align}
ds^2
=
a^2(\tau)\Big[-(1+2\psi)\,d\tau^2+(1-2\phi)\,\delta_{ij}\,dx^i dx^j\Big].
\end{align}

% Linear relation (comoving slicing)
\begin{align}
\zeta=-\phi.
\end{align}

% More general (still linear) Newtonian-gauge expression
\begin{align}
\zeta
=
-\phi
-
\mathcal{H}\,v,
\qquad
\mathcal{H}\equiv \frac{1}{a(\tau)}\,\frac{d a(\tau)}{d\tau},
\qquad
T^{0}{}_{i}=-(\rho+P)\,\partial_i v.
\end{align}



%==============================================================================
\subsection{Comving curvature perturbations}
%===============================================================================

The gauge-invariant comoving curvature perturbation $\mathcal{R}$, i.e. the 3-curvature of comoving spatial slices, is defined in terms of metric and matter perturbations
\begin{align}\label{eq:comoving_curvature_perturbations_newtonian}
    \mathcal{R}\equiv C-\tfrac13\nabla^2E+\mathcal{H}(B+v) \overset{\text{newt. gauge}}{=} -\phi +\mathcal{H}v,
\end{align}
since in Newtonian gauge $C=-\phi$ and $B=E=0$.
Beware another gauge invariant quantity $\zeta$ is often used in place of the comoving curvature $\mathcal{R}$.
In Newtonian gauge it is defined by\footnote{The second equality holds for separately conserved em tensors}. 
\begin{align}
    \zeta\overset{\text{newt gauge}}{=} -\phi-\mathcal{H}\frac{\delta \rho}{\dot{\rho}} = -\phi +\tfrac13 \frac{\delta }{1+(\bar{p}/\bar{\rho})}.
\end{align}
From the gauge-invariant Poisson equation \eqref{eq:poisson_gauge_invariant} we can relate the two quantities
{\small
\begin{align}
    \zeta-\mathcal{R}=-\mathcal{H}\frac{\delta \rho}{\dot{\rho}}-\mathcal{H}v
    = -\mathcal{H}\frac{\bar{\rho}}{\dot{\bar{\rho}}}\Big(\delta +\frac{\dot{\bar{\rho}}}{\bar{\rho}}v\Big)
    = \tfrac13 \frac{\bar{\rho}\Delta}{\bar{\rho}+\bar{p}}
    =-\frac{k^2}{12\pi G a^2 (\bar{\rho}+\bar{p})}\phi
    \,\to 0\quad\text{as}\,\,k\to0,
\end{align}}
confirming the two quantities indeed coincide on large scales.



\newpage
%---------------------------------------------------------------------
%====================================================================
\section{Summary of Fluid EFT}
%====================================================================
%-----------------------------------------------------------------
%
In this section we summarize the conventions and formulas of the fluid EFT for our convenience while drafting, regardless of what will actually appear in the main text.\notetag{This sec will be shortened at the end.}
First we present the fomrmulas valid at any order and then the perturbative expansion.


%====================================================================
\subsection{Effective larangian \& thermodynamic interpretation}
%====================================================================
%
The mapping from the spacetime manifold $\mathcal{M}^{(4)}$ to the internal fluid space $\mathcal{F}^{(3)}$ is given by four scalar fields\footnote{Note we use $\varphi$ for fluid coordinates, not to confuse them with Bardeen or Newtonian potentials.}  
{\small\begin{align}\label{eq:internal_coord_fluid}
    \varphi^{\alpha}:(\mathcal{M},g)\to \mathbb{R}\times \mathcal{F},\quad \alpha=0,1,2,3.
\end{align}}
Note we are free to choose the coordinates in the 3-dimensional internal space and to shift $\varphi^0$ by a constant without using any gauge freedom of the spacetime manifold itself.
In particular, we might choose $\varphi^\alpha(x)=x^\alpha$ at some fixed initial time $x^0=\tau_{in}$.

The 4-velocity of the fluid is identified by requiring the spatial coordinates of the fluid be indeed comoving with it, that is $\partial_\mu \varphi^i=0$ for $i=1,2,3$.
After normalizing the 4-velocity we get
\begin{align}\label{eq:definition_4_velocity}
    u^\mu=& \frac{1}{\sqrt{-g}\, 3!\,b} \epsilon^{\mu \alpha \beta \gamma} \epsilon_{ijk} \, \partial_\alpha \varphi^i \partial_\beta \varphi^j \partial_\gamma \varphi^k\, \propto\,\,  \star(d\varphi^1\wedge d\varphi^2\wedge d \varphi^3), \quad u_\mu u^\mu=-1\,.
\end{align}
We can identify 2 adimensional quantities
\begin{align}\label{eq:definition_b_y}
    b=\sqrt{\det B}\quad \text{for}\quad
    B^{ij}= g^{\mu\nu} \partial_\mu \varphi^i \partial_\nu \varphi^j,\quad y= u^\mu \partial_\mu \varphi^0, \qquad [b]=[y]=[u^\mu]=E^0,
\end{align}
that respect Lorentz invariance in the spacetime manifold and the internal symmetries of the fluid, namely 
{\small\begin{align}\label{eq:symmetries_fluid_internal}
    &\varphi^i \mapsto \tilde{\varphi}^i\big(\{\varphi^j\}_j\big)\,\,\text{with}\,\, \det\Big(\frac{\partial \tilde{\varphi}^i}{\partial\varphi^j}\Big)=1\,\,\quad \text{(volume preserving 3d diffeo invariance)},\\
    &\varphi^0 \mapsto \varphi^0 + c\quad\qquad\quad\qquad\qquad\qquad\qquad \text{(shift symmetry in $\varphi^0$)}.
\end{align}}
Up to a \emph{dimensional} scale $\Lambda$ these variables are naturally identified with the number density $b\equiv n$ and the temperature $y\equiv T$ in the fluid EFT as explained in the next section.

The lagrangian is a generic function of the invariants
\begin{align}\label{eq:eft_action_exact}
    S=\int d^4x \sqrt{-g} F(b,y)\,.
\end{align}
The conserved Noether currents are the number density current (from volume-preserving diffeo invariance in the $\varphi^i$) and the entropy current\footnote{\label{fn:thermodynamic_interpretation}See next subsection for the explanation of this thermodynamic interpretation.} (from shift symmetry in $\varphi^0$)
\begin{align}\label{eq:noether_currents_nonbaro}
    J_n^\mu = b\, u^\mu\,,\qquad J_s^\mu = F_y \,u^\mu\,.
\end{align}
Varying the action wrt the metric, the energy-momentum tensor takes the perfect fluid form\footnote{Subscripts mean $X_b=\partial_b X_{\mid y}$ and $X_y=\partial_y X_{\mid b}$ i.e. main variables $(b,y)$, unless otherwise stated.}
\begin{align}\label{eq:em_tensor_nonbaro}
    T_{\mu\nu} = u_\mu u_\nu (yF_y - bF_b) + g_{\mu\nu} (F-bF_b) = u_\mu u_\nu(\rho+p) + g_{\mu\nu}\, p\,.
\end{align}
Energy density, pressure and equation of state parameter are then identified as
\begin{align}\label{eq:thermo_rho_p_nonbaro}
    \rho=-F+yF_y,\quad p=F-bF_b,\quad w=\frac{F-bF_b}{yF_y-F}.
\end{align}
The entropy density $s$, comoving entropy $\sigma$, temperature $T$, chemical potential $\mu$ and number density $n$ are identified as\footnotemark[\getrefnumber{fn:thermodynamic_interpretation}]
\begin{align}\label{eq:thermo_other_variables_nonbaro}
    s=F_y,\quad \sigma:=\frac{s}{n}=\frac{F_y}{b},\quad T=y,\quad \mu:=F_b, \quad n=b, \quad F=-\rho+Ts=-\mathfrak{a}.
\end{align}
Notably this identifies $F(y,b)$ with minus the Helmoltz free-energy density {\small $\mathfrak{a}=\frac{U-TS}{V}$.}
Adiabatic sound speed $c_a^2$ and non-adiabatic pressure coefficient $\Gamma$ are then by definition
{\small\begin{align}\label{eq:adiabatic_cs_nonadiabatic_pressure}
\begin{aligned}
    &c_a^2=\frac{\partial p}{\partial \rho}_{|\sigma}=\frac{-y^2F_{yy}b^2F_{bb}+(yF_y-bF_b)^2}{y^2F_{yy}(yF_y-bF_b)}, \\
    &\Gamma:=\frac{\partial p}{\partial \sigma}_{\mid\rho}= \frac{b^2F_{bb}y^2F_{yy}-(yF_y-byF_{by})(bF_b-byF_{yb})}{\tfrac{1}{by}y^2F_{yy}(yF_y-bF_b)}.
\end{aligned}
\end{align}}
Using $F=-\mathfrak{a}$ and $s=F_y=-\partial_T\mathfrak{a}$, we rewrite the adiabatic sound speed as
\begin{align}
    c_a^2:=
    &=\frac{-b^2F_{bb} F_{yy}+(F_y-bF_{yb})^2}{F_{yy}\,(yF_y-bF_b)} = \frac{n^2\,\partial_n^2\mathfrak{a}\,\partial_Ts + (s-n\partial_ns)^2}{\partial_Ts\,\,(\rho+p)}\,,
\end{align}
which makes it manifest that $c_a^2\geq0$ provided no ghosts $\rho+P>0$, the second principle of thermodynamics $\partial_Ts\geq 0$ and thermodynamic stability $\partial_n^2\mathfrak{a}\geq 0$ hold.

The adiabatic sound speed \eqref{eq:adiabatic_cs_nonadiabatic_pressure} always coincides with the coefficient $c_s^2$ of the gradient term in the quadratic action for phonons, after integrating out the non-dynamical entropy mode $\pi^0$,
\begin{align}
    c_s^2\equiv c_a^2\,,
\end{align}
which shows that absence of ghosts, thermodynamic stability and gradient instabilities are closely related.
For flat spacetimes this is proved in \eqref{eq:cs_equiv_cb_flat_spacetime} below.\todotag{prove it in curved space}

Finally, in the barotropic case we can still define $y=\partial_\mu\varphi^0u^\mu$, but $F=F(b)$ does not depend on $y$ and the above formulas reduce to
{\small\begin{align}\label{eq:thermo_variables_barotropic}
    \rho=-F,\quad p=F-bF_b,\quad w=\frac{bF_b-F}{F},\quad n=b,\quad \mu=F_b,\quad T=y\,\quad s\equiv0,\quad \frac{d p}{d \rho}=c_a^2=\frac{b^2F_{bb}}{bF_b}.
\end{align}}



%-----------------------------------------------------------------------
\subsubsection{Justification of the thermodynamic interpretation}
%----------------------------------------------------------------------
In this section we justify the thermodynamic interpretation of the fluid EFT variables and formulas.
As argued above, the identification of the 4-velocity \eqref{eq:definition_4_velocity} follows unavoidably from the requirement the spatial coordinates $\varphi^i$ be indeed comoving with the fluid.

The next point is the unequivocal identification of $b$ with the number density $n$.
Indeed, in order for the mapping \eqref{eq:internal_coord_fluid} from spacetime to the internal fluid space to make sense, it is necessary that the restriction $\varphi^{I}:\Sigma_\tau^{(3)}\to \mathcal{F}^{(3)}$ is a diffeomorphism for any spacelike hypersurface $\Sigma_\tau^{(3)}\subset \mathcal{M}^{(4)}$.
The fluid coordinates $\{\varphi^i\}_{i=1,2,3}$ can thus be used as coordinates for the hypersurface $\Sigma_\tau^{(3)}$, and $g^{\mu\nu}\partial_\nu\varphi^i$ are 3 independent vectors spanning its tangent space.
In these coordinates, the metric $g_{|\Sigma}$ induced on spatial slices reads
\begin{align}
    (g^\varphi_{\Sigma})_{ij} = \partial_\mu \varphi^i\partial_\nu \varphi^j g^{\mu\nu}\equiv B^{ij}.
\end{align}
The volume factor $\sqrt{g^\varphi_\Sigma}=\sqrt{\det B}=b$, i.e. the jacobian of the transformation from spatial coordinates to internal fluid coordinates, is then precisely the density of the fluid elements wrt the original spacetime coordinates $b\equiv n$.

To complete the thermodynamic interpretation we need some final imput, which comes from matching the EFT energy-momentum tensor to that of a perfect fluid \eqref{eq:em_tensor_nonbaro}, thereby identifying the energy density and pressure \eqref{eq:thermo_rho_p_nonbaro}.
Finally we simply impose the first law of thermodynamics
\begin{align}\label{eq:first_law_thermo}
    T s+ \mu n = \rho+p \overset{!}{=} yF_y-bF_b.
\end{align}
Since we have already shown $n=b$, we are lead to identify $\mu=-F_b$, and $s=F_y$ and $y=T$.
This is consistent with all the thermodynamical relations, for example we find back $\mu$ imposing entropy is conserved
\begin{align}\label{eq:chemical_potential_and_entropy_consistency}
    \begin{array}{c}
     0\overset{!}{=}ds = F_{yb} db + F_{yy}dy\\[3pt]
    \displaystyle\Rightarrow\,\, \frac{dy}{db}_{|s}=-\frac{F_{yb}}{F_{yy}}
    \end{array}
    \quad\Rightarrow\quad \mu:=\frac{\partial\rho}{\partial n}_{|s}= \underbrace{\frac{\partial\rho}{\partial b}_{|y}}_{=-F_b+yF_{by}}+ \underbrace{\frac{\partial\rho}{\partial y}_{|b}}_{=yF_{yy}}\,\,\frac{dy}{db}_{|s}=-F_b.
\end{align}
In fact, after enforcing $n=b$ and the expressions for $\rho$ and $p$, equating $s=F_y,\,T=y$ is the \emph{only} consistent identification.
Indeed taking $n=b$ and $y=s$ predicts wrong conjugate variables and fails the 1st law of thermodynamics \eqref{eq:first_law_thermo} since
\begin{align}\begin{aligned}
    &n=b,\,\,s= y \,\,\Rightarrow\,\, \begin{cases}
        \mu:=\frac{\partial\rho}{\partial n}_{|s}= \frac{\partial\rho}{\partial b}_{|y}=-F_b+yF_{by}\\
         T:=\frac{\partial\rho}{\partial s}_{|n}= \frac{\partial\rho}{\partial y}_{|b}=yF_{yy}
    \end{cases}\\[4pt]
    &\,\,\Rightarrow\,\, \mu n +T s= -bF_b+byF_{by}+y^2F_{yy}\neq yF_y-bF_b=\rho+p.
\end{aligned}\end{align}




%==============================================================
\subsection{Quadratic expansion \& matching phonons to fluid perturbations}
%============================================================
%
The above formulas are exact to any order in the fluid coordinates.o
We now expand to quadratic order around unperturbed\footnote{Beware this choice of fluid coordinates does not use any gauge freedom of the spacetime manifold!} fluid coordinates $\bar{\varphi}^\alpha\equiv x^\alpha$ and FLRW spacetime.
The general perturbed FLRW metric is written wrt conformal time $\tau$ as
{\small \begin{align}\label{eq:general_perturbed_FLRW_metric}
\begin{aligned}
    &g_{\mu\nu}=a^2\Big(\eta_{\mu\nu}+h_{\mu\nu}\Big)\,,\quad g^{\mu\nu}=a^{-2}\Big(\eta^{\mu\nu}-h^{\mu\nu}\Big)\quad \text{with} \quad h^{\mu\nu}=\eta^{\mu\alpha}\eta^{\nu\beta}h_{\alpha\beta},\\
    &\quad \sqrt{|g|}= a^4\big(1+\tfrac12 h^{\mu\nu}\eta_{\mu\nu} +O(h^2)\big)\ = a^4\Big[1+\tfrac12 (h^{ii}-h^{00}) +O(h^2)\Big]\,.
\end{aligned}
\end{align}}
The fluid variables are written for simplicity with rescaling factors $\gamma$ and $\lambda$ that will be used to reabsord unphysical constants in the compuations
{\small\begin{align}\label{eq:perturbed_fluid_coordinates}
    \varphi^0=\gamma\Big(\tau+\pi^0\Big), \quad \varphi^i=\lambda\Big(x^i+\pi^i\Big)\,, \quad [\varphi^\alpha]=[x^\alpha]=[\pi^\alpha]=E^{-1},\quad [\gamma]=[\lambda]=E^0.
\end{align}}
We have the expansions for the invariants $b,y$ and the 4-velocity
{\small\begin{align}\label{eq:expansions_b_y_u}
\begin{aligned}
    &b= \bar{b}(1+\delta b),\quad \bar{b}= \frac{\lambda^3}{a^3},\quad
    \delta b^{(1)} = -\tfrac12 h^{ii} +(\partial_j \pi^j),\\[3pt]
    &\delta b^{(2)} = - h^{0 i} \dot{\pi}^i - \tfrac12  h^{ii} (\partial_j \pi^j) - \tfrac12 (\dot{\pi}^i)^2 + \tfrac12 (\partial_j \pi^j)^2 -\tfrac12 \partial_i\pi^j\partial_j\pi^i\,,
    \\[3pt]
    &y= \bar{y}(1+\delta y),\quad \bar{y}= \frac{\gamma}{a},\quad \delta y^{(1)} = \tfrac12 h^{00}+ \dot{\pi}^0,\\[3pt]
    &\delta y^{(2)} = \tfrac12 h^{00}\dot{\pi}^0+ h^{0 i} \dot{\pi}^i  + \tfrac12 (\dot{\pi}^i)^2 -\partial_j\pi^0\dot{\pi}^j\,\\[3pt]
    &au^\mu=\delta_0^\mu \bigg(1 \!+\! \Big[\tfrac12 h^{00} -(\partial_j \pi^j)\Big]\! +\! \Big[ h^{0 i} \dot{\pi}^i -\tfrac12 h^{00} (\partial_j \pi^j) + \tfrac12 (\dot{\pi}^i)^2 + \tfrac12 (\partial_j \pi^j)^2 +\tfrac12 \partial_i\pi^j\partial_j\pi^i\Big]\bigg)
    \\
    &\quad\qquad+\tfrac{1}{2}\left[1 + \tfrac12 h^{00}-(\partial_\ell\pi^\ell)\right]
     \epsilon_{i j k}\,\epsilon^{\mu \alpha j k}\,\partial_\alpha\pi^i 
     +\frac{1}{2} \epsilon_{ijk}\, \epsilon^{\mu i \beta\gamma}\partial_\beta\pi^j\,\partial_\gamma\pi^k
    + O(\pi^3, h^2)
\end{aligned}\end{align}}
%
In turn we recast the fluctuations in density $\rho=-F+yF_y$ and pressure $p=F-bF_b$ as
\begin{align}\label{eq:rho_p_fluctuations_via_b_y}
\begin{aligned}
    \delta\rho &= y^2F_{yy}\delta y + (-bF_b+byF_{by})\delta b
    \\
    &= (y^2F_{yy})\left(\tfrac12 h^{00}+ \dot{\pi}^0\right) + (-bF_b+byF_{by})\left(-\tfrac12 h^{ii} +(\partial_j \pi^j)\right) + O(2)
    \\[3pt]
    \delta p &= (yF_y-byF_{by})\delta y - (b^2 F_{bb})\delta b
    \\
    &= (yF_y-byF_{by})\left(\tfrac12 h^{00}+ \dot{\pi}^0\right) - (b^2 F_{bb})\left(-\tfrac12 h^{ii} +(\partial_j \pi^j)\right)+ O(2)\,.
\end{aligned}
\end{align}
We also introduce the phonon potential to rewrite longitudinal (scalar) phonons
\begin{align}\label{eq:phonon_potential}
    \pi_L=\nabla^{-2}\partial_\ell\pi^\ell, \quad \pi^\ell_{\text{||}}=\partial_\ell \pi_L,\quad  [\pi_L]=E^{-2}.
\end{align}
Matching the velocity expansion \eqref{eq:expansions_b_y_u} in the fluid EFT to the general expression \eqref{eq:velocity_decomposition} in pertrubed FLRW spacetime $au^\mu =a(1+.., v^\ell)$, we identify the peculiar velocity and the velocity potential/divergence in terms of phonons as
\begin{align}\label{eq:peculiar_velocity_fluid_vs_phonons}
    v^\ell = \frac{1}{2}\epsilon_{i j k}\,\epsilon^{\ell \alpha j k}\,\partial_\alpha\pi^i = -\dot{\pi}^\ell\quad \Rightarrow \quad \theta=-\partial_\ell\dot{\pi}^\ell= -\nabla^2\dot{\pi}_L,\,\, \,\,v = -\dot{\pi}_L\,.
\end{align}



%==========================================================================
%\subsection{Quadratic action for phonons}
%==========================================================================
%
%=======================================================================
\subsection{Quadratic action for phonons}
%======================================================================
%
In this section we use the expansions \eqref{eq:expansions_b_y_u} to expand the action \eqref{eq:eft_action_exact} up to second order in phonon fields $\pi^\alpha \pi^\beta$ and mixed order in metric perturbations $h_{\mu\nu}\,\pi^\alpha$.
To get some intuition we first present the flat-spacetime case, since formulas are cleaner and  ultimately exact in the limit $\rho/ M_{Pl}^{4}\to0$.

%-----------------------------------------------------------------
\subsubsection{Quadratic action in flat spacetime}
%-------------------------------------------------------
%
We refer to the full computations in curved spacetime below not to repeat the same steps.
In flat spacetime $a=1$ and $h_{\mu\nu}=0$, only the first terms in the action \eqref{eq:final_quadratic_action_curved_spacetime} survive, and we can integrate by parts in both space and time the term $\partial_j \pi^0\dot{\pi}^j = -\dot{\pi}^0(\partial_j \pi^j)$ to get
{\small\begin{align}
\begin{aligned}
    S^{(2)}&= \int d^4x \frac{(yF_y-bF_b)}{2}\bigg\{(\dot{\pi}^i)^2 - (\partial_j\pi^j)^2\Big[-\frac{b^2F_{bb}}{(yF_y-bF_b)}\Big]\\
    &\qquad\qquad\qquad\qquad\qquad+(\dot{\pi}^0)^2 \frac{y^2F_{yy}}{(yF_y-bF_b)}
    - 2\dot{\pi}^0 (\partial_j \pi^j) \frac{(yF_y-byF_{by})}{(yF_y-bF_b)}
    \bigg\}\\
    &=\int d^4x \frac{(\rho+p)}{2}\bigg\{(\dot{\pi}^i)^2 - \mathcal{C} (\partial_j\pi^j)^2 
    + (\dot{\pi}^0)^2 \mathcal{A}
    - 2\mathcal{M}\dot{\pi}^0 (\partial_j \pi^j) 
    \bigg\}
\end{aligned}
\end{align}}
for $\rho= -F + y F_y$, $p= F - b F_b$ and 
{\small\begin{align}
\begin{aligned}
    &\mathcal{C} = -\frac{b^2F_{bb}}{(yF_y-bF_b)},\quad 
    \mathcal{A} = \frac{y^2F_{yy}}{(yF_y-bF_b)} = \frac{T\, \frac{\partial\rho }{\partial T}}{\rho+p},\quad
    \mathcal{M} = \frac{(yF_y-byF_{by})}{(yF_y-bF_b)}.
\end{aligned}
\end{align}}
The entropy field $\pi^0$ is thus non-dynamical and the resulting constraint equation gives
\begin{align}
    \dot{\pi}^0 = \frac{\mathcal{M}}{\mathcal{A}} (\partial_j \pi^j) + \text{const.}
\end{align}
The constant is not physical and is reabsorbed in the factor $\gamma$ of the original definition $\varphi^0:=\gamma(t+\pi^0)$.
After solving the constraint, the action for the coordinate $\pi^i$ becomes
{\small
\begin{align}
    S^{(2)}&= \int d^4x \frac{(\rho+p)}{2}\Big[(\dot{\pi}^i)^2 - c_s^2 (\partial_j\pi^j)^2 \Big], \quad \text{for}\quad c_s^2 = \mathcal{C} + \frac{\mathcal{M}^2}{\mathcal{A}}.
\end{align}
}
We note the entropy contribution to the speed of sound is always positive in \emph{physical} situations.
Indeed $\mathcal{M}^2$ is a true square and, recalling $y\equiv T$, we have
\begin{equation}
    \mathcal{A}= \frac{T\, \frac{\partial\rho }{\partial T}}{\rho+p}\geq 0 \quad\text{provided}\quad
    \begin{cases}
        \rho+p\geq 0 \quad(\text{no ghosts}),\\
        \frac{\partial\rho }{\partial T} \geq 0 \quad(\text{positive heat capacity}).
    \end{cases} 
\end{equation}
Finally we confirm the speed of sound $c_s^2$, defined as the coefficient of the gradient term for the fluctuations above, always matches the thermodynamical definition of \emph{adiabatic} sound speed \eqref{eq:adiabatic_cs_nonadiabatic_pressure}, namely
\begin{align}\label{eq:cs_equiv_ca_flat_spacetime}
    c_a^2:= \frac{\partial p}{\partial \rho}_{\mid \sigma} \overset{\text{proved before}}{=}\frac{-b^2F_{bb} F_{yy}+(F_y-bF_{yb})^2}{F_{yy}\,(yF_y-bF_b)} \overset{\text{easy check}}{\equiv} c_s^2.
\end{align}
This is expected: longitudinal phonons, i.e. sound waves, do propagate with the speed of sound!





%-----------------------------------------------------------------
\subsubsection{Quadratic action in curved spacetime {\color{red}(in progress)}}
%--------------------------------------------------------------
%
{\color{red}[I am reporting the full computation here since I am not yet satisfied with the final form. It will eventually be moved to the computation appendix.]}\todotag{Read \& feel the struggle}

\noindent
We now consider a general perturbed FRW metric written as in \eqref{eq:general_perturbed_FLRW_metric} and expand the action up to quadratic order  $\pi^\alpha \pi^\beta,\,h_{\mu\nu}\,\pi^\alpha$ for fluid perturbations
{\small\begin{align}
\begin{aligned}
    S&=\int d^4x \sqrt{-g} F(b,y) = \int d^4x\,a^4 \big[1+\tfrac12 (h^{ii}-h^{00}) \big] F(\bar{b}(1+\delta b), \bar{y}(1+\delta y))
    \\
    &= \int \!\!\!d^4\!x\, a^4 \!\big[1+\tfrac12 (h^{ii}-h^{00}) \big]
    \Bigg[ F(\bar{b}, \bar{y}) + (\bar{b}F_b)  (\delta b^{(1)} + \delta b^{(2)}) + (\bar{y}F_y) (\delta y^{(1)} + \delta y^{(2)})\\
    &\,\,\qquad\qquad\qquad\qquad\qquad\qquad+ \tfrac12 (\bar{b}^2 F_{bb})(\delta b^{(1)})^2 + \tfrac12 (\bar{y}^2 F_{yy}) (\delta y^{(1)})^2
    + (\bar{b}\bar{y} F_{by}) \delta b^{(1)} \delta y^{(1)} \Bigg]
\end{aligned}
\end{align}}
Inserting the expansions \eqref{eq:expansions_b_y_u} for $b$ and $y$, the quadratic part of the action is then
{\small\begin{align}
\begin{aligned}
    S^{(2)}&=\int d^4x\,\, a^4 \bigg\{
        (\bar{b}F_b) (h^{ii}-h^{00}) \Big[(\partial_j \pi^j) - \tfrac12 h^{ii}\Big]
        + (\bar{y}F_y) (h^{ii}-h^{00}) \Big[\tfrac12 h^{00}+ \dot{\pi}^0\Big]\\
    &\quad + (\bar{b}F_b) \Big[ - h^{0 i} \dot{\pi}^i - \tfrac12  h^{ii} (\partial_j \pi^j) - \tfrac12 (\dot{\pi}^i)^2 + \tfrac12 (\partial_j \pi^j)^2 -\tfrac12 \partial_i\pi^j\partial_j\pi^i\Big]
    \\
    &\quad + (\bar{y}F_y) \Big[ \tfrac12 h^{00}\dot{\pi}^0+ h^{0 i} \dot{\pi}^i + \tfrac12 (\dot{\pi}^i)^2 -\partial_j\pi^0\dot{\pi}^j\Big]
    + \tfrac12 (\bar{b}^2 F_{bb})\Big[ (\partial_j \pi^j) - \tfrac12 h^{ii}\Big]^2
    \\
    &\quad + \tfrac12 (\bar{y}^2 F_{yy})\Big[\tfrac12 h^{00}+ \dot{\pi}^0\Big]^2
    + (\bar{b}\bar{y} F_{by}) \Big[(\partial_j \pi^j) - \tfrac12 h^{ii}\Big]\Big[\tfrac12 h^{00}+ \dot{\pi}^0\Big]
        \bigg\}.
\end{aligned}
\end{align}}
Dropping  further second orders in metric perturbations and {\color{blue} integrating by parts in space}\footnote{\label{fn:integrating_parts_space_backround}Space derivatives yield no further terms since the background isspatially homogenous.}, we get
{\small\begin{align}
\begin{aligned}
    S^{(2)}=
    \int d^4x\,\, a^4 \bigg\{&
        (\bar{b}F_b) (h^{ii}-h^{00}) (\partial_j \pi^j)
        + (\bar{y}F_y) (h^{ii}-h^{00}) \dot{\pi}^0 \\
    &\quad + (\bar{b}F_b) \Big[ - h^{0 i} \dot{\pi}^i - \tfrac12  h^{ii} (\partial_j \pi^j) - \tfrac12 (\dot{\pi}^i)^2 + \bluecancel{\tfrac12 (\partial_j \pi^j)^2} -\bluecancel{\tfrac12 \partial_i\pi^j\partial_j\pi^i}\Big]
    \\
    &\quad + (\bar{y}F_y) \Big[ \tfrac12 h^{00}\dot{\pi}^0+ h^{0 i} \dot{\pi}^i + \tfrac12 (\dot{\pi}^i)^2 -\partial_j\pi^0\dot{\pi}^j\Big]
    \\
    &\quad + \tfrac12 (\bar{b}^2 F_{bb})\Big[ (\partial_j \pi^j)^2 - h^{ii}(\partial_j \pi^j)\Big]
    + \tfrac12 (\bar{y}^2 F_{yy})\Big[h^{00} \dot{\pi}^0+ (\dot{\pi}^0)^2\Big]\\
    &\quad+ (\bar{b}\bar{y} F_{by}) \Big[ \dot{\pi}^0(\partial_j \pi^j) - \tfrac12 h^{ii}\dot{\pi}^0 + \tfrac12 h^{00} (\partial_j\pi^j)\Big]
        \bigg\}.
\end{aligned}
\end{align}}
Collecting terms $\pi^\alpha\pi^\beta$ and $\pi^\alpha h_{\mu\nu}$, the final expression for the quadratic action is then
{\small\begin{align}\label{eq:final_quadratic_action_curved_spacetime}
\begin{aligned}
    S^{(2)}=
    &\int d^4x\,\,a^4\,\,\bigg\{
    \tfrac12 (yF_y-bF_b) (\dot{\pi}^i)^2+\tfrac12 b^2 F_{bb} (\partial_j \pi^j)^2
    +\tfrac12 y^2 F_{yy} (\dot{\pi}^0)^2 \\
    &\qquad\qquad\qquad+ (b y F_{by}) (\dot{\pi}^0 \partial_j \pi^j)-(yF_y) (\partial_j \pi^0\dot{\pi}^j)
    \bigg\}\\
    &\quad + a^4\,\,\bigg\{
    \tfrac12\!\left(\bar{b}F_b - \bar{b}^2 F_{bb}\right) h^{ii}(\partial_j \pi^j)
    + \tfrac12\!\left(-2\bar{b}F_b + \bar{b}\bar{y}F_{by}\right) h^{00}(\partial_j \pi^j)\\
    &\qquad\qquad+ \left(\bar{y}F_y - \bar{b}F_b\right) h^{0j}\,\dot{\pi}^j + \tfrac12\!\left(2\bar{y}F_y - \bar{b}\bar{y}F_{by}\right) h^{ii}\dot{\pi}^0
    + \tfrac12\!\left(\bar{y}^2 F_{yy} - \bar{y}F_y\right) h^{00}\dot{\pi}^0
    \bigg\}\,.
\end{aligned}
\end{align}}
Note that only \emph{longitudinal} modes $\pi^j_{||}\propto\, \partial_j\pi^j$ are dynamical.
Transverse modes $\pi^j_{\perp}$ simply mix with vector metric perturbations $h^{0j}$ with resulting equation of motion
{\small\begin{align}\label{eq:eom_transverse_phonons}
    a^4\,\big(\bar{\rho}+\bar{p}\big)\,\big(\dot{\pi}_\perp^j + h^{0j}\,\big)=\text{const}\,.
\end{align}}
This is expected since vorticity in the fluid sources vector modes in the metric and viceversa.

In the following we thus restrict to longitudinal modes $\pi_{||}^j$.
Gathering $\rho+p=yF_y-bF_b$ outside the brackets to highlight ghosts, we rewrite the action \eqref{eq:final_quadratic_action_curved_spacetime} as\todotag{Probably should wait and integrate bp first}
{\small\begin{align}
\begin{aligned}
    S^{(2)}= &\int d^4x\,\,a^4\, (\rho+p)\,\bigg\{
    \tfrac12 (\dot{\pi}^i)^2-\tfrac12 \underbrace{\tfrac{-b^2 F_{bb}}{(yF_y-bF_b)}}_{=\mathcal{C}} (\partial_j \pi^j)^2
    +\tfrac12 \underbrace{\tfrac{y^2 F_{yy}}{(yF_y-bF_b)}}_{=\mathcal{A}} (\dot{\pi}^0)^2 \\
    &\quad\qquad\qquad\qquad\qquad+ \tfrac{(b y F_{by})}{(yF_y-bF_b)} (\dot{\pi}^0 \partial_j \pi^j)-\tfrac{(yF_y)}{(yF_y-bF_b)} (\partial_j \pi^0\dot{\pi}^j)
    \bigg\}\\
    & \,\,+ a^4\, (\rho+p)\, \bigg\{
    \tfrac12\!\left(\tfrac{\bar{b}F_b - \bar{b}^2 F_{bb}}{(yF_y-bF_b)}\right) h^{ii}(\partial_j \pi^j)
    + \tfrac12\!\left(\tfrac{-2\bar{b}F_b + \bar{b}\bar{y}F_{by}}{(yF_y-bF_b)}\right) h^{00}(\partial_j \pi^j)\\
    &\quad\qquad\qquad+ \left(\tfrac{\bar{y}F_y - \bar{b}F_b}{(yF_y-bF_b)}\right) h^{0 j}\,\dot{\pi}^j + \tfrac12\!\left(\tfrac{2\bar{y}F_y - \bar{b}\bar{y}F_{by}}{(yF_y-bF_b)}\right) h^{ii}\dot{\pi}^0
    + \tfrac12\!\left(\tfrac{\bar{y}^2 F_{yy} - \bar{y}F_y}{(yF_y-bF_b)}\right) h^{00}\dot{\pi}^0
    \bigg\}.
\end{aligned}
\end{align}}
%
Integrating by parts in both time and space the term $\partial_j \pi^0\dot{\pi}^j$ we get\footnotemark[\getrefnumber{fn:integrating_parts_space_backround}]
{\small
\begin{align}
\begin{aligned}
    S^{(2)}&=\int d^4x\, -(\partial_j \pi^j\, \pi^{0})\tfrac{d}{d\tau}(a^4yF_y)\\
    &+a^4\,\bigg\{
    \tfrac12 (yF_y-bF_b) (\dot{\pi}^i)^2+ \tfrac12 b^2 F_{bb} (\partial_j \pi^j)^2 +\tfrac12 y^2 F_{yy} (\dot{\pi}^0)^2+(b y F_{by}-yF_y) (\dot{\pi}^0 \partial_j \pi^j)\bigg\}\\
    &+ a^4\,\,\bigg\{
    \tfrac12\!\left(\bar{b}F_b - \bar{b}^2 F_{bb}\right) h^{ii}(\partial_j \pi^j)
    + \tfrac12\!\left(-2\bar{b}F_b + \bar{b}\bar{y}F_{by}\right) h^{00}(\partial_j \pi^j)+\left(\bar{y}F_y - \bar{b}F_b\right) h^{0 j}\,\dot{\pi}^j\\
    &\qquad\qquad +\tfrac12\!\left(2\bar{y}F_y - \bar{b}\bar{y}F_{by}\right) h^{ii}\dot{\pi}^0
    + \tfrac12\!\left(\bar{y}^2 F_{yy} - \bar{y}F_y\right) h^{00}\dot{\pi}^0
    \bigg\}.
\end{aligned}
\end{align}}
%
The constraint equation for $\pi^0$ now reads
{\small\begin{align}
\begin{aligned}
    \frac{d}{d\tau}&a^4\bigg[y^2F_{yy}\dot{\pi}^0+(byF_{by}-yF_y) (\partial_j\pi^j)+\tfrac12\!\big(2{y}F_y - byF_{by}\big) h^{ii}
    + \tfrac12\!\left({y}^2 F_{yy} - {y}F_y\right) h^{00}\bigg]
    \\
    &=-(\partial_j \pi^j)\tfrac{d}{d\tau}(a^4yF_y)\,.
\end{aligned}
\end{align}}
Resolving the constraint and plugging it back into the action we get
{\small\begin{align}
\begin{aligned}
    S^{(2)}&=
    \int d^4x\,\, a^4\,\bigg\{
    \tfrac12 (yF_y-bF_b) (\dot{\pi}^i)^2+ \tfrac12 b^2 F_{bb} (\partial_j \pi^j)^2\bigg\} -\tfrac12 a^4 y^2F_{yy} (\dot{\pi}^0)^2\\
    & \,\,\,+ a^4\bigg\{\tfrac12\!\left(\bar{b}F_b - \bar{b}^2 F_{bb}\right) h^{ii}(\partial_j \pi^j)
    + \tfrac12\!\left(\bar{b}\bar{y}F_{by}-2\bar{b}F_b\right) h^{00}(\partial_j \pi^j)+\left(\bar{y}F_y - \bar{b}F_b\right) h^{0j}\,\dot{\pi}^j \bigg\},
\end{aligned}
\end{align}}
where now\todotag{2nd form likely better}
{\small
\begin{align}
\begin{aligned}
    a^4y^2F_{yy}\dot{\pi}^0
    =&-a^4\bigg[(byF_{by}-yF_y) (\partial_j\pi^j)+\tfrac12\!\big(2{y}F_y - byF_{by}\big) h^{ii}
    + \tfrac12\!\left({y}^2 F_{yy} - {y}F_y\right) h^{00}\bigg]\\
    &-\int d\tau\, (\partial_j \pi^j)\tfrac{d}{d\tau}(a^4yF_y)\\
    &{\color{red}=-a^4\bigg[(byF_{by}-yF_y) (\partial_j\pi^j)+\bigg(\tfrac{1}{a^4}\!\!\int d\tau(\partial_j \pi^j)\tfrac{d}{d\tau}(a^4yF_y)\bigg)\bigg]}\\
    &{\color{red}\qquad-a^4\tfrac12\Big[\big(2{y}F_y - byF_{by}\big) h^{ii}+ \left({y}^2 F_{yy} - {y}F_y\right) h^{00}\Big]}.
\end{aligned}
\end{align}}
That is
{\small
\begin{align}
\begin{aligned}
    -\tfrac12&a^4y^2F_{yy}\big(\dot{\pi}^0\big)^2\\
    &=-\tfrac12 a^4 \tfrac{(byF_{by}-yF_y)^2}{y^2F_{yy}} (\partial_j\pi^j)^2\\
    &\quad-\tfrac{(byF_{by}-yF_y)}{y^2F_{yy}} (\partial_j\pi^j)\bigg(\int(\partial_\ell\pi^\ell)\tfrac{d}{d\tau}(a^4yF_y) \bigg)
    -\tfrac12 \tfrac{a^{-4}}{y^2F_{yy}}\bigg(\int(\partial_\ell\pi^\ell)\tfrac{d}{d\tau}(a^4yF_y)\bigg)^2
    \\
    &\quad-\tfrac12 a^4\tfrac{\big(2{y}F_y-byF_{by}\big)(byF_{by}-yF_y)}{y^2F_{yy}}h^{ii}(\partial_j\pi^j)
    -\tfrac12 a^4\tfrac{\left({y}^2 F_{yy} - {y}F_y\right)(byF_{by}-yF_y)}{y^2F_{yy}}h^{00}(\partial_j\pi^j)
    \\
    &\quad -\tfrac12 \tfrac{\big(2{y}F_y-byF_{by}\big)}{y^2F_{yy}}h^{ii}\bigg(\int\!d\tau(\partial_\ell\pi^\ell)\tfrac{d}{d\tau}(a^4yF_y)\bigg)
    -\tfrac12 \tfrac{\left({y}^2 F_{yy} - {y}F_y\right)}{y^2F_{yy}}h^{00}\bigg(\int\! d\tau(\partial_\ell\pi^\ell)\tfrac{d}{d\tau}(a^4yF_y)\bigg).
\end{aligned}
\end{align}}
Or equivalently keeping therms $\propto \, \nabla \pi+\int d\tau \nabla\pi$ together
{\small\color{red}
\begin{align}
\begin{aligned}
    -\tfrac12 a^4y^2F_{yy}\big(\dot{\pi}^0\big)^2
    &=-\tfrac{1}{2}\tfrac{a^4}{y^2F_{yy}}\bigg[(byF_{by}-yF_y) (\partial_j\pi^j)+\bigg(\tfrac{1}{a^4}\!\!\int\! d\tau\,(\partial_j \pi^j)\tfrac{d}{d\tau}(a^4yF_y)\bigg)\bigg]^2\\
    &\quad-\tfrac{a^4}{y^2F_{yy}}\Big[\big(2{y}F_y - byF_{by}\big) h^{ii}+ \left({y}^2 F_{yy} - {y}F_y\right) h^{00}\Big]\\
    &\qquad\qquad\cdot \bigg[(byF_{by}-yF_y) (\partial_j\pi^j)+\bigg(\tfrac{1}{a^4}\!\!\int\! d\tau\,(\partial_j \pi^j)\tfrac{d}{d\tau}(a^4yF_y)\bigg)\bigg]\,.
\end{aligned}
\end{align}}
%
Substituting back in the action we finally get\todotag{Rewrite w red exp, keeping $\nabla\pi+\int$ together}
{\small
\begin{align}
\begin{aligned}
    S^{(2)}=&\int d^4x\,\, a^4\,\frac{\rho+p}{2}\Bigg\{
    (\dot{\pi}^i)^2-\bigg[-\tfrac{b^2 F_{bb}}{(yF_y-bF_b)} +\tfrac{(byF_{by}-yF_y)^2}{y^2F_{yy}(yF_y-bF_b)}\bigg] (\partial_j \pi^j)^2 \\
    &\quad-2\tfrac{(byF_{by}-yF_y)}{y^2F_{yy}(yF_y-bF_b)} (\partial_j\pi^j)\bigg(\tfrac{1}{a^4}\int\!d\tau(\partial_\ell\pi^\ell)\tfrac{d}{d\tau}(a^4yF_y) \bigg)
    -\tfrac{1}{y^2F_{yy}(yF_y-bF_b)}\bigg(\tfrac{1}{a^4}\int(\partial_\ell\pi^\ell)\tfrac{d}{d\tau}(a^4yF_y)\bigg)^2\\
    &\quad +\left(\tfrac{\bar{b}F_b-\bar{b}^2 F_{bb}}{yF_y-bF_b} - \tfrac{\big(2{y}F_y-byF_{by}\big)(byF_{by}-yF_y)}{(yF_y-bF_b)y^2F_{yy}}\right) h^{ii}(\partial_j \pi^j)\\
    &\qquad\qquad\qquad+\!\left(\tfrac{\bar{b}\bar{y}F_{by}-2\bar{b}F_b}{yF_y-bF_b} - \tfrac{\left({y}^2 F_{yy} - {y}F_y\right)(byF_{by}-yF_y)}{(yF_y-bF_b)y^2F_{yy}}\right) h^{00}(\partial_j \pi^j)
    +2h^{0 j}\,\dot{\pi}^j\\
    &-\tfrac{\big(2{y}F_y-byF_{by}\big)}{y^2F_{yy}(yF_y-bF_b)}h^{ii}\bigg(\tfrac{1}{a^4}\int(\partial_\ell\pi^\ell)\tfrac{d}{d\tau}(a^4yF_y)\bigg)
    -\tfrac{\left({y}^2 F_{yy} - {y}F_y\right)}{y^2F_{yy}(yF_y-bF_b)}h^{00}\bigg(\tfrac{1}{a^4}\int(\partial_\ell\pi^\ell)\tfrac{d}{d\tau}(a^4yF_y)\bigg)\Bigg\}.
\end{aligned}
\end{align}}







%================================================================================
\subsection{Related computations (to be removed at the end)}
%================================================================================

Dear collaborators, I know you just want to see the results, but I need a safe place to keep my computations for several reasons: they might (often) be wrong, I want to improve or simplify them, we might disagree on the results, they remind me how to do the math the fastest way in case I need to redo them, etc.
I promise I will keep them hidden here without polutting the main text and will remove them at the end.


%-------------------------------------------------------------------------
\subsubsection{Thermo formulas adiabatic sound speed \& non-adiabatic pressure}
%--------------------------------------------------------------------------

The entropy per particle is $\sigma=F_y/b$, imposing it be preserved gives
\begin{align}
    0=d\sigma = \frac{F_{yy}}{b}dy+\Big(\frac{bF_{by}-F_y}{b^2}\Big)db\quad \Rightarrow\quad \frac{dy}{db}_{\mid\sigma}= \frac{F_y-b F_{yb}}{bF_{yy}}=\frac{y}{b}\frac{p_y}{\rho_y}.
\end{align}
The adiabatic sound speed is then by definition\notetag{Polish redundant formulas}
\begin{align}
    c_b^2:=\frac{dp}{d\rho}_{\mid\sigma}&=\frac{p_b + p_y\, \tfrac{dy}{db}_{\mid\sigma}}{\rho_b + \rho_y\, \tfrac{dy}{db}_{\mid\sigma}} 
    =\frac{-b^2F_{bb} F_{yy}+(F_y-bF_{yb})^2}{F_{yy}\,(yF_y-bF_b)}\\[5pt]
    &= \frac{b p_b\, y\rho_y+(yp_y)^2}{b\rho_b\,y\rho_y+y\rho_y\,yp_y} = \frac{b\,p_b \,y\rho_y + (y\,p_y)^2}{y \rho_y (\rho+p)}\,.
\end{align}
The rewritings in the second line might come handy when trying to show positivity or other relations.
Using $F=-\mathfrak{a}$ and $s=F_y=-\partial_T\mathfrak{a}$, we rewrite the adiabatic sound speed as
\begin{align}
    c_b^2:=
    &=\frac{-b^2F_{bb} F_{yy}+(F_y-bF_{yb})^2}{F_{yy}\,(yF_y-bF_b)} = \frac{n^2\,\partial_n^2\mathfrak{a}\,\partial_Ts + (s-n\partial_ns)^2}{\partial_Ts\,\,(\rho+p)}\,,
\end{align}
which makes it manifest that $c_b^2\geq0$ provided no ghosts $\rho+P>0$, the second principle of thermodynamics $\partial_Ts\geq 0$ and thermodynamic stability $\partial_n^2\mathfrak{a}\geq 0$ hold.
The equivalence $c_a^2\equiv c_s^2$ is proved in \eqref{eq:cs_equiv_ca_flat_spacetime} for flat spacetime.\todotag{check curved spacetime}

Similarly we compute the nonadiabatic pressure coefficient $\frac{\partial P}{\partial \sigma}_{|\rho}$.
We proceed as before enforcing 
\begin{equation}
    0=d\rho=yF_{yy}dy+(yF_{yb}-F_b)db\quad \Rightarrow\quad \frac{dy}{db}_{|\rho}=\frac{F_b-yF_{yb}}{yF_{yy}}
\end{equation}
As above we get, with analgous possible thermoodynamical rewriting in terms of the free energy density $\mathfrak{a}$,  
{\small
\begin{align}\label{eq:nonadiabatic_pressure_coefficient_gamma}
\begin{aligned}
    \Gamma:=\frac{\partial p}{\partial \sigma}_{\mid\rho}&=\frac{p_b + p_y\, \tfrac{dy}{db}_{\mid\rho}}{\sigma_b + \sigma_y\, \tfrac{dy}{db}_{\mid\rho}}=\frac{-bF_{bb}+(F_y-bF_{by})\tfrac{F_b-yF_{yb}}{yF_{yy}}}{\Big(\tfrac{bF_{by}-F_y}{b^2}\Big)+\tfrac{F_{yy}}{b}\frac{F_b-yF_{yb}}{yF_{yy}}}
    \\
    &=\frac{-b^2F_{bb}y^2F_{yy}+(yF_y-byF_{by})(bF_b-byF_{yb})}{y^2F_{yy}\Big(\tfrac{bF_{by}-F_y}{b}\Big)+yF_{yy}(F_b-yF_{yb})}
    \\
    &=\frac{b^2F_{bb}y^2F_{yy}-(yF_y-byF_{by})(bF_b-byF_{yb})}{\tfrac{1}{by}y^2F_{yy}(yF_y-bF_b)}.
\end{aligned}\end{align}}



\newpage
%============================================================================
\section{Recovering fluids with known EoS in the EFT language}
%============================================================================

Recall the expressions
\begin{align}
    \rho=-F+yF_y,\quad p=F-bF_b,\quad w=\frac{F-bF_b}{yF_y-F},\quad c_s^2=\frac{\partial p}{\partial \rho}_{|\sigma}=\frac{-y^2F_{yy}b^2F_{bb}+(yF_y-bF_b)^2}{y^2F_{yy}(yF_y-bF_b)}.
\end{align}
and in the barotropic case
\begin{align}
    \rho=-F,\quad p=F-bF_b,\quad w=\frac{bF_b-F}{F},\quad \frac{d p}{d \rho}=c_s^2=\frac{b^2F_{bb}}{bF_b}.
\end{align}
Recall $[\phi]=[\pi]=[x]=E^{-1}$, $b=[\partial_\mu\phi\partial^\mu\phi]=E^0$, $[u^\mu]=E^0$ and $y=[u^\mu\partial_\mu\phi]=E^0$.


\subsection*{Dark Matter}
Demanding Dark Matter be a barotropic fluid and imposing $=0$ we get
\begin{align}
    0= w=\frac{bF_b-F}{F}\quad \Rightarrow\quad F=\Lambda^4 \,b\quad \text{for some scale } \, [\Lambda]=E,
\end{align}
which gives energy density $\rho=-F\propto \, b$ with the expecte behavior for cold matter $\rho \approx \, m \, n$.



\subsection*{Relativistic fluids}
How to match a relativistic fluid in the EFT, say the photon-baryon fluid early on?

A fully relativistic fluid $p\simeq E$ that can be described in the Boltzmann formalism is necessarily barotropic, with the familiar EoS $w=\tfrac13$, namely
\begin{align}
    \rho:=T^0_0=\int\frac{d^3\vec{p}}{(2\pi)^3} E\, f, \quad P:= \tfrac13 T^i_i =\int\frac{d^3\vec{p}}{(2\pi)^3} \frac{p^2}{3E}= \frac13 \int\frac{d^3\vec{p}}{(2\pi)^3} E\, f = \tfrac13 \rho.
\end{align}
This holds at any order since perturbations are encoded into $f=\bar{f}+\delta f$.
In particular we also find $c_s^2=\tfrac13$ necessarily
\begin{align}
    \delta P:=\int\frac{d^3\vec{p}}{(2\pi)^3} \frac{p^2}{3E}\, \delta f = \frac13 \int\frac{d^3\vec{p}}{(2\pi)^3} E \delta f =: \tfrac13 \delta\rho.
\end{align}
There is no need to specify the thermodynamic variables we vary/keep fixed, since there is only one independent parameter be it $\rho,P,T,s,n,$.
For the record the entropy is
\begin{align}
    s=\frac{\rho+p}{T}= \frac{\tfrac43 \rho}{T}, \quad \sigma=\frac{s}{n}=\left(\int\frac{d^3\vec{p}}{(2\pi)^3} f\right)^{-1} \frac{\tfrac43 \rho}{T}.
\end{align}

In the fluid EFT treatment, for a barotropic fluid we enforce
\begin{align}
    \tfrac13 \overset{!}{=}w_{\textbf{baro}}=\frac{bF_b-F}{F}\quad \Rightarrow\quad F(b)=\Lambda^4\, b^{4/3}
\end{align}
Which captures the usual scaling for radiation $\rho\propto T^4 \propto n^{4/3}$.
Consistently, this automatically implies 
\begin{align}
    c_s^2=\frac{b^2F_{bb}}{bF_b}=\frac{1}{3}.
\end{align}

This begs the question:  are there \emph{nonbarotropic} relativistic fluids, in the fluid EFT this simply means enforcing $w=\tfrac13$, where however $c_s^2\neq\tfrac13$?\, NO.
Enforcing $w=\tfrac13$ we find
\begin{align}
    \tfrac13=w=\frac{F-bF_b}{yF_y-F} \quad \Rightarrow\quad \tfrac43 F=\tfrac13 yF_y+bF_b.
\end{align}
Differentiating this relation wrt $y$ and $b$ respectively and rearranging a bit we get
\begin{align}
    yF_y-ybF_{yb}=\tfrac13 y^2F_{yy}, \quad \tfrac13 (bF_b-ybF_{yb})=bF_{bb}.
\end{align}
Plugging these into
\begin{align}
    c_s^2=\frac{-y^2F_{yy}b^2F_{bb}+(yF_y-bF_b)^2}{y^2F_{yy}(yF_y-bF_b)}\overset{identically}{=}\frac13\,.
\end{align}


\subsection*{Fluids with constant EoS parameter}

Consider fluids with an almost constant EoS
\begin{align}
    \frac{F-F_b}{yF_y-F}=w\approx \mathrm{const}\quad\Rightarrow\quad (1+w)F=w\, yF_y+bF_b.
\end{align}
The first order PDE is solved by characteristic\footnote{A convenient choice of variables is $\gamma:=\log uy$ and $\beta:=\log b$.}.
The most general solution is, for arbitrary one variable function $f=f(z)$,
\begin{align}
    F(y,b)=f(y\,b^{-w})\, b^{(1+\omega)}.
\end{align}
It captures the barotropic case simply setting $y=0$.

Let us study the implications of $w=\mathrm{const}$ on $c_s^2$ defined in \eqref{eq:adiabatic_sound_speed} and $\Gamma$ defined in \eqref{eq:nonadiabatic_pressure_coefficient_gamma}.
Of course we expect $\Gamma=0$ since we enforced the constant EoS in full generality...\todotag{finish}


\subsection*{Fluids with arbitrary EoS parameter $w(b,y)$}

More generally we can solve by characteristics the EoS for arbitrary parameters $w=w(b,\gamma)$
\begin{align}
    \frac{F-F_b}{yF_y-F}=w\approx \mathrm{const}\quad\Rightarrow\quad (1+w)F=w\, yF_y+bF_b.
\end{align}
This is just a linear transport equation.
Choosing variables $\gamma:=\log uy$ and $\beta:=\log b$, the characteristics are
{\small
\begin{align}\begin{aligned}
    \dot{\beta}&=1\\
    \dot{\gamma}&=w(\gamma,\beta)\\
    \dot{z}&=(1+w)z
\end{aligned}\end{align}
}
The solution can be written in general but is probably not worth it.

Crucially, we cannot say anything more on $c_s^2$ or $\Gamma$for a completely general $w(y,b)$.
This is no surprise: in general we always have 2 independent thermodynamical DoFs and 1 independent function of these, furnishing the `equation of state' in a broader sense.
These can be for example $b,y$, $\rho,p$ or $\rho,w$ and the function might be $F$.
Here we are trading some of the arbitrariness of $F$ into the arbitrariness of $w$ and of $f$ (where $f$ is the initial function/data on the hypersurface e.g. $b=b_0$ we solve the PDE with characteristic cf. the constant $w$ case for the notation) with the functional dependence of $f$ constrained by characteristics.

As soon as we say something more on $w$ e.g. that $w=\mathrm{const}$ then we can draw further conclusions.

It is quite simple when variables split $w=w_1(y)w_2(b)$.
For example when $w=w(b)$ characteristic simply becomes
{\small
\begin{align}\begin{aligned}
    \gamma(\beta,\gamma_0)=\int_{\beta_0}^\beta d\beta\, w(\beta)+\gamma_0\quad\Rightarrow\quad \gamma_0=\gamma-\int_{b_0}^b \!\!d\log b\,\, w(b)\\
    z(\beta):=F(\gamma(\beta,\gamma_0),\beta)= f(\gamma_0)\,\exp\left(\int_{\beta_0}^\beta\, d\beta\,\,(1+w(\beta))\right)
\end{aligned}\end{align}
}
The general solution for $w=w(b)$ is then
\begin{align}
    F(y,b)=f\Big(y\, e^{-\int_{b_0}^bd\log b\, w(b)}\Big)\exp\left(\int_{b_0}^bd\log b\, \big(1+w(b)\big)\right)
\end{align}
Similarly the general solutions for $w=w(y)$ is
\begin{align}
    F(y,b)=f\Big(b\, e^{-\int_{y_0}^yd\log y\, \frac{1}{w(y)}}\Big)\exp\left(\int_{y_0}^bd\log y\,\Big(1+\tfrac{1}{w(y)}\Big)\right)
\end{align}




\newpage
%-------------------------------------------------------
%=============================================================
\section{Matching to phonons}
%===============================================================
%------------------------------------------------------------


Matching the general expression for the 4-velocity \eqref{eq:velocity_decomposition} to the one for the phonon field \eqref{eq:phonon_4_velocity} we get
\begin{align}\label{eq:peculiar_velocity_via_phonons}
    v^\ell = \frac{1}{2}\epsilon_{i j k}\,\epsilon^{\ell \alpha j k}\,\partial_\alpha\pi^i = -\dot{\pi}^\ell\quad \Rightarrow \quad v = -\dot{\pi}_L\quad \text{where we define $\pi_L$ by}\quad \partial_\ell\pi_L=\pi^\ell
\end{align}
Dimensions are respected since $[\pi^\alpha]=E^{-1}$ and $[v^\ell]=E^0$, so that $[v]=E^{-1}$ and $[\pi_L]=E^{-2}$. 
From \label{eq:peculiar_velocity_via_phonons} we identify the comoving curvature perturbations \eqref{eq:comoving_curvature_perturbations_newtonian} in terms of metric potentials and phonon potential
\begin{equation}
    \mathcal{R} = -\phi +\mathcal{H}v = -\phi - \mathcal{H}\,\dot{\pi}_L\,.
\end{equation}


In Newtonian gauge \eqref{eq:newtonian_gauge_metric}, the quadratic lagrangian becomes 
{\small
\begin{align}
\begin{aligned}
    S^{(2)}=
    &\int d^4x\,\, a^4\,\bigg\{
    \tfrac12 (yF_y-bF_b) (\dot{\pi}^i)^2+ \tfrac12 b^2 F_{bb} (\partial_j \pi^j)^2\bigg\} -\tfrac12 a^4 y^2F_{yy} (\dot{\pi}^0)^2\\
    &\quad + a^4\,\,\bigg\{-3\phi(\partial_j \pi^j)\left(\bar{b}F_b - \bar{b}^2 F_{bb}\right) 
    -\psi(\partial_j \pi^j)\!\left(-2\bar{b}F_b + \bar{b}\bar{y}F_{by}\right) \bigg\}\,,
\end{aligned}
\end{align}}
where
{\small
\begin{align}
\begin{aligned}
    -\tfrac12 a^4y^2F_{yy}\big(\dot{\pi}^0\big)^2
    &=-\tfrac{1}{2}\tfrac{a^4}{y^2F_{yy}}\bigg[(byF_{by}-yF_y) (\partial_j\pi^j)+\bigg(\tfrac{1}{a^4}\!\!\int(\partial_j \pi^j)\tfrac{d}{d\tau}(a^4yF_y)\bigg)\bigg]^2\\
    &\quad+\tfrac{a^4}{y^2F_{yy}}\Big[\big(2{y}F_y - byF_{by}\big) 6\phi+ \left({y}^2 F_{yy} - {y}F_y\right) 2\psi\Big]\\
    &\qquad\qquad\cdot \bigg[(byF_{by}-yF_y) (\partial_j\pi^j)+\bigg(\tfrac{1}{a^4}\!\!\int(\partial_j \pi^j)\tfrac{d}{d\tau}(a^4yF_y)\bigg)\bigg]\,.
\end{aligned}
\end{align}}
Taking $\psi=\phi$ in the large scale limit and simplifying terms we finally get\todotag{write}
\begin{align}
    ...
\end{align}
The equation of motion for phonons will be something like
\begin{align}
    \ddot{\pi}_L=-\underbrace{c_s^2 k^2\pi_L}_{\to0}+\text{friction}
\end{align}
giving that $\ddot{\pi}_L\to 0$ as $k\to0$.
For the comoving curvature then
\begin{align}
    \dot{\mathcal{R}}=-\dot{\phi}+\dot{\mathcal{H}}\,\dot{\pi}_L+\mathcal{H}\,\ddot{\pi}_L
\end{align}
and, upon using the 0i Einstein equation for $\dot{\phi}+\mathcal{H}\phi=...$, the vanishing of $\ddot{\pi}$ from the EoM of phonons might be the fastest way to predict that $\mathcal{R}$ is conserved on superhorizon scales.\todotag{Finish this argument}


\newpage
%============================================================================
\section{Matching the quintessence field (in progress)}
%============================================================================

In the 5-essence case we can express kinetic energy and potential in terms of energy density and pressure. 
Next we use the expression \eqref{eq:rho_pressure_exp_nonbaro} of the stress tensor in the fluid EFT to express kinetic energy and potential in terms of $F$ and its derivatives
\begin{align}\label{eq:identiyinf_rho_p_vs_scalar_field}
    \rho=\tfrac12 X +V, \quad p =\tfrac12 X -V\quad \Rightarrow\quad \begin{array}{c}
        X=\rho+p = yF_y + bF_b,\\
        V(\varphi) = \tfrac12(\rho - p) \equiv -F +\tfrac12 (yF_y+ bF_b)\,.
    \end{array}
\end{align}
The function $F$ is yet unknown.
In fact, imposing further constraints from the matching, we expect to get a set of PDEs which will single out a whole family of suitable candidates $F$.

In the scalar field case we have no notion of entropy or temperature whatsoever, as expected since it should be at temperature zero.
On the other hand, we have a well-defined notion of velocity $\vec{v}_\varphi$, with potential $v_\varphi$ or divergence $\theta_\varphi=\nabla^2v_\varphi=\vec{\nabla}\cdot\vec{v}_\varphi$, derived from the perturbations of the energy-momentum tensor
\begin{align}
    T^{\mu}_{\nu}=u^\mu u_\nu X+\delta^{\mu}_{\nu}(\tfrac12 X-V), \quad X=-\partial_\mu\varphi g^{\mu\nu}\partial_\nu\varphi, \,\, u_\mu=X^{-1/2}\partial_\mu\varphi.
\end{align}
Namely expanding $\varphi=\bar{\varphi}+\delta\varphi$ we have\todotag{check conto, guessato}
\begin{align}
    (\rho+p)\,\vec{v}_\phi\approx a^{-2}\dot{\bar{\varphi}}\vec{\nabla} \delta\varphi.
\end{align}
From \eqref{eq:peculiar_velocity_fluid_vs_phonons} we then identify
\begin{align}\label{eq:identifying_scalar_field_velocity_vs_phonon}
    v_\varphi = -\dot{\pi}_L\, \quad\text{that is}\quad \theta_\varphi=-\partial_j\dot{\pi}^j=k^2\dot{\pi}_L
\end{align}

On the other hand, we have a well-defined notion of temperature $y\equiv T$ and entropy density $s=F_y$ and comoving entropy $\sigma=F_y/b$ in the fluid EFT.
Recalling the expansions \eqref{eq:expansions_b_y} we have
{\small
\begin{align}\label{eq:expansions_b_y}
\begin{aligned}
    &\delta b^{(1)} = -\tfrac12 h^{ii}-k^2\pi_L,\\[3pt]
    &\delta y^{(1)} = \tfrac12 h^{00}+ \dot{\pi}^0 \approx \tfrac12 h^{00} +{\color{red}(\text{bkg coefficients)}} k^2 \pi_L +{\color{red}\text{(other metric terms)}}
\end{aligned}
\end{align}}
we find
\begin{align}\label{eq:identifying_entropy_vs_scala_field}
    \delta \sigma = \frac{bF_{yb}-F_y}{b} \delta b + \frac{yF_{yy}}{b}\delta y \propto \, {\color{red}(\text{bkg coefficients)}}\,k^2\pi_L+\text{(metric stuff)}
\end{align}
So that we identify $\dot{\delta\sigma}\approx \theta_\varphi$.

Let's take stock.
We have an expression for $\theta_\varphi$ in terms of the fluid EFT, and it is also immediate to identify $\delta\rho_\varphi$ and $\delta p_\varphi$ with the corresponsing fluid EFt quantities $\delta\rho$ and $\delta p$.
We should now be able to match the scalar field respecting the constraints on
\begin{align}
    w=p/\rho,\quad c_s^2:=\frac{\partial p}{\partial \rho}_{|\sigma}, \Gamma:=\frac{\partial p}{\partial \sigma}_{|\rho}
\end{align}
Indeed $\Gamma$ enter the Euler equation for the scalar field, so that, regardless of not having a notion of entropy we should be able to find an explicit expression for $\Gamma$.

Alternatively, we could simply define $c_s^2$ as the effective coefficients for fluctuations in $\ddot{\delta\varphi}=k^2c_s^2\delta\varphi+...$ and then simply subtract it off
\begin{align}
    \Gamma\delta\sigma:=\delta p-c_s^2\delta\rho
\end{align}
to get an expression for $\Gamma \delta\sigma$ in terms of 5-essence variables.
If I am not retarded, we should hopefully recover the above relation for $\dot{\delta\sigma}$ in terms of $v_\phi+\text{(metric)}$ confirming the two ways are indeed equivalent.








%%%%%%%%%%%%%%%%%%%%%%%%%%%%%%%%%%%%%%%%%%%%%%%%%%%%%%%%%%%%%%%%%%%%%%%%%%%

\bibliography{bib.bib}

\end{document}
