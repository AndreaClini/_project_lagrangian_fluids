\title{Accelerating the Universe with non-barotropic fluids}
\author{Diego Redigolo \thanks{INFN, Florence section \& Physics Department, University of Florence (diego.redigolo@fi.infn.it)}
        \and Andrea Clini \thanks{Physics Department, University of Milan (andrea.clini@unimi.it)}
        \and La Pimpa
        \and Rambo}
\date{\today} 


%%%%%%%%%%%%% FRONT %%%%%%%%%%%%%%%%%%%%%%%%%%%%%%%%%
\begin{document}
\section{Introduction}
Which fluid allows for accelerated expansion? Assuming a single fluid dominates the energy density of the Universe we can relate the acceleration to its equation of state 
\begin{equation}
\frac{\ddot{a}}{a}=-\frac{4\pi G_N\rho}{3}(1+3w)    
\end{equation}
where $w=p/\rho$ is the equation of stare. This equation tells us the well known thing that having accelerated expansion implies $w<-1/3$. We also need to check that the accelerating background is stable which amounts to exclude the presence of negative 






\end{document}