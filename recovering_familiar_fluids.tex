%============================================================================
\section{Recovering fluids with known EoS in the EFT language}
%============================================================================

Recall the expressions
\begin{align}
    \rho=-F+yF_y,\quad p=F-bF_b,\quad w=\frac{F-bF_b}{yF_y-F},\quad c_s^2=\frac{\partial p}{\partial \rho}_{|\sigma}=\frac{-y^2F_{yy}b^2F_{bb}+(yF_y-bF_b)^2}{y^2F_{yy}(yF_y-bF_b)}.
\end{align}
and in the barotropic case
\begin{align}
    \rho=-F,\quad p=F-bF_b,\quad w=\frac{bF_b-F}{F},\quad \frac{d p}{d \rho}=c_s^2=\frac{b^2F_{bb}}{bF_b}.
\end{align}
Recall $[\phi]=[\pi]=[x]=E^{-1}$, $b=[\partial_\mu\phi\partial^\mu\phi]=E^0$, $[u^\mu]=E^0$ and $y=[u^\mu\partial_\mu\phi]=E^0$.


\subsection*{Dark Matter}
Demanding Dark Matter be a barotropic fluid and imposing $=0$ we get
\begin{align}
    0= w=\frac{bF_b-F}{F}\quad \Rightarrow\quad F=\Lambda^4 \,b\quad \text{for some scale } \, [\Lambda]=E,
\end{align}
which gives energy density $\rho=-F\propto \, b$ with the expecte behavior for cold matter $\rho \approx \, m \, n$.



\subsection*{Relativistic fluids}
How to match a relativistic fluid in the EFT, say the photon-baryon fluid early on?

A fully relativistic fluid $p\simeq E$ that can be described in the Boltzmann formalism is necessarily barotropic, with the familiar EoS $w=\tfrac13$, namely
\begin{align}
    \rho:=T^0_0=\int\frac{d^3\vec{p}}{(2\pi)^3} E\, f, \quad P:= \tfrac13 T^i_i =\int\frac{d^3\vec{p}}{(2\pi)^3} \frac{p^2}{3E}= \frac13 \int\frac{d^3\vec{p}}{(2\pi)^3} E\, f = \tfrac13 \rho.
\end{align}
This holds at any order since perturbations are encoded into $f=\bar{f}+\delta f$.
In particular we also find $c_s^2=\tfrac13$ necessarily
\begin{align}
    \delta P:=\int\frac{d^3\vec{p}}{(2\pi)^3} \frac{p^2}{3E}\, \delta f = \frac13 \int\frac{d^3\vec{p}}{(2\pi)^3} E \delta f =: \tfrac13 \delta\rho.
\end{align}
There is no need to specify the thermodynamic variables we vary/keep fixed, since there is only one independent parameter be it $\rho,P,T,s,n,$.
For the record the entropy is
\begin{align}
    s=\frac{\rho+p}{T}= \frac{\tfrac43 \rho}{T}, \quad \sigma=\frac{s}{n}=\left(\int\frac{d^3\vec{p}}{(2\pi)^3} f\right)^{-1} \frac{\tfrac43 \rho}{T}.
\end{align}

In the fluid EFT treatment, for a barotropic fluid we enforce
\begin{align}
    \tfrac13 \overset{!}{=}w_{\textbf{baro}}=\frac{bF_b-F}{F}\quad \Rightarrow\quad F(b)=\Lambda^4\, b^{4/3}
\end{align}
Which captures the usual scaling for radiation $\rho\propto T^4 \propto n^{4/3}$.
Consistently, this automatically implies 
\begin{align}
    c_s^2=\frac{b^2F_{bb}}{bF_b}=\frac{1}{3}.
\end{align}

This begs the question:  are there \emph{nonbarotropic} relativistic fluids, in the fluid EFT this simply means enforcing $w=\tfrac13$, where however $c_s^2\neq\tfrac13$?\, NO.
Enforcing $w=\tfrac13$ we find
\begin{align}
    \tfrac13=w=\frac{F-bF_b}{yF_y-F} \quad \Rightarrow\quad \tfrac43 F=\tfrac13 yF_y+bF_b.
\end{align}
Differentiating this relation wrt $y$ and $b$ respectively and rearranging a bit we get
\begin{align}
    yF_y-ybF_{yb}=\tfrac13 y^2F_{yy}, \quad \tfrac13 (bF_b-ybF_{yb})=bF_{bb}.
\end{align}
Plugging these into
\begin{align}
    c_s^2=\frac{-y^2F_{yy}b^2F_{bb}+(yF_y-bF_b)^2}{y^2F_{yy}(yF_y-bF_b)}\overset{identically}{=}\frac13\,.
\end{align}


\subsection*{Fluids with constant EoS parameter}

Consider fluids with an almost constant EoS
\begin{align}
    \frac{F-F_b}{yF_y-F}=w\approx \mathrm{const}\quad\Rightarrow\quad (1+w)F=w\, yF_y+bF_b.
\end{align}
The first order PDE is solved by characteristic\footnote{A convenient choice of variables is $\gamma:=\log uy$ and $\beta:=\log b$.}.
The most general solution is, for arbitrary one variable function $f=f(z)$,
\begin{align}
    F(y,b)=f(y\,b^{-w})\, b^{(1+\omega)}.
\end{align}
It captures the barotropic case simply setting $y=0$.

Let us study the implications of $w=\mathrm{const}$ on $c_s^2$ defined in \eqref{eq:adiabatic_sound_speed} and $\Gamma$ defined in \eqref{eq:nonadiabatic_pressure_coefficient_gamma}.
Of course we expect $\Gamma=0$ since we enforced the constant EoS in full generality...\todotag{finish}


\subsection*{Fluids with arbitrary EoS parameter $w(b,y)$}

More generally we can solve by characteristics the EoS for arbitrary parameters $w=w(b,\gamma)$
\begin{align}
    \frac{F-F_b}{yF_y-F}=w\approx \mathrm{const}\quad\Rightarrow\quad (1+w)F=w\, yF_y+bF_b.
\end{align}
This is just a linear transport equation.
Choosing variables $\gamma:=\log uy$ and $\beta:=\log b$, the characteristics are
{\small
\begin{align}\begin{aligned}
    \dot{\beta}&=1\\
    \dot{\gamma}&=w(\gamma,\beta)\\
    \dot{z}&=(1+w)z
\end{aligned}\end{align}
}
The solution can be written in general but is probably not worth it.

Crucially, we cannot say anything more on $c_s^2$ or $\Gamma$for a completely general $w(y,b)$.
This is no surprise: in general we always have 2 independent thermodynamical DoFs and 1 independent function of these, furnishing the `equation of state' in a broader sense.
These can be for example $b,y$, $\rho,p$ or $\rho,w$ and the function might be $F$.
Here we are trading some of the arbitrariness of $F$ into the arbitrariness of $w$ and of $f$ (where $f$ is the initial function/data on the hypersurface e.g. $b=b_0$ we solve the PDE with characteristic cf. the constant $w$ case for the notation) with the functional dependence of $f$ constrained by characteristics.

As soon as we say something more on $w$ e.g. that $w=\mathrm{const}$ then we can draw further conclusions.

It is quite simple when variables split $w=w_1(y)w_2(b)$.
For example when $w=w(b)$ characteristic simply becomes
{\small
\begin{align}\begin{aligned}
    \gamma(\beta,\gamma_0)=\int_{\beta_0}^\beta d\beta\, w(\beta)+\gamma_0\quad\Rightarrow\quad \gamma_0=\gamma-\int_{b_0}^b \!\!d\log b\,\, w(b)\\
    z(\beta):=F(\gamma(\beta,\gamma_0),\beta)= f(\gamma_0)\,\exp\left(\int_{\beta_0}^\beta\, d\beta\,\,(1+w(\beta))\right)
\end{aligned}\end{align}
}
The general solution for $w=w(b)$ is then
\begin{align}
    F(y,b)=f\Big(y\, e^{-\int_{b_0}^bd\log b\, w(b)}\Big)\exp\left(\int_{b_0}^bd\log b\, \big(1+w(b)\big)\right)
\end{align}
Similarly the general solutions for $w=w(y)$ is
\begin{align}
    F(y,b)=f\Big(b\, e^{-\int_{y_0}^yd\log y\, \frac{1}{w(y)}}\Big)\exp\left(\int_{y_0}^bd\log y\,\Big(1+\tfrac{1}{w(y)}\Big)\right)
\end{align}


