%==========================================================================
\section{Quadratic action in curved spacetime}
%===========================================================================
We analyze the fluctuations in a general curved background metric, written wrt conformal time $\tau$ as
{\small 
\begin{align}
\begin{aligned}
    &g_{\mu\nu}=a^2\Big(\eta_{\mu\nu}+h_{\mu\nu}\Big)\,,\quad g^{\mu\nu}=a^{-2}\Big(\eta^{\mu\nu}-h^{\mu\nu}\Big)\quad \text{with} \quad h^{\mu\nu}=\eta^{\mu\alpha}\eta^{\nu\beta}h_{\alpha\beta},\\
    &\quad \sqrt{|g|}= a^4\big(1+\tfrac12 h^{\mu\nu}\eta_{\mu\nu} +O(h^2)\big)\ = a^4\Big[1+\tfrac12 (h^{ii}-h^{00}) +O(h^2)\Big]\,.
\end{aligned}
\end{align}
}
The fluids variables are written for simplicity with a rescaling factor $\gamma$ and $\lambda$ as
\begin{align}
    \phi^0=\gamma\Big(\tau+\pi^0\Big), \quad \phi^i=\lambda\Big(x^i+\pi^i\Big)\,.
\end{align}
The invariants are again the number density $n\equiv b$ defined by
\begin{align}
    b=\sqrt{\det B}\quad \text{for}\quad
    B^{ij}= g^{\mu\nu} \partial_\mu \phi^i \partial_\nu \phi^j,
\end{align}
and the temperature $T\equiv y$ defined by
\begin{align}
    y= u^\mu \partial_\mu \phi^0
    \quad\text{for}\quad u^\mu=& \frac{1}{\sqrt{-g}\, 3!\,b} \epsilon^{\mu \alpha \beta \gamma} \epsilon_{ijk} \, \partial_\alpha \phi^i \partial_\beta \phi^j \partial_\gamma \phi^k \propto\,  \star(d\phi^1\wedge d\phi^2\wedge d \phi^3).
\end{align}

We now expand up to quadratic order in the fields $\pi$ and mixed order in $\pi h$, ignoring pure second orders in the metric $h^2$.
The latter would indeed give a contribution $\propto \, \sqrt{|g|}\, F(\bar{b}, \bar{y}) h^2$ that could be reabsorbed in a redefinition of the cosmological constant.
We have the expansions
{
\begin{align}\label{eq:expansions_b_y}
\begin{aligned}
    &b= \bar{b}(1+\delta b),\quad \bar{b}= \frac{\lambda^3}{a^3},\quad
    \delta b^{(1)} = -\tfrac12 h^{ii} +(\partial_j \pi^j),\\[3pt]
    &\delta b^{(2)} = - h^{0 i} \dot{\pi}^i - \tfrac12  h^{ii} (\partial_j \pi^j) - \tfrac12 (\dot{\pi}^i)^2 + \tfrac12 (\partial_j \pi^j)^2 -\tfrac12 \partial_i\pi^j\partial_j\pi^i\,,
    \\[3pt]
    &y= \bar{y}(1+\delta y),\quad \bar{y}= \frac{\gamma}{a},\quad \delta y^{(1)} = \tfrac12 h^{00}+ \dot{\pi}^0,\\[3pt]
    &\delta y^{(2)} = \tfrac12 h^{00}\dot{\pi}^0+ h^{0 i} \dot{\pi}^i  + \tfrac12 (\dot{\pi}^i)^2 -\partial_j\pi^0\dot{\pi}^j.
\end{aligned}
\end{align}
}
In turn we can express the fluctuations in density $\rho=-F+yF_y$ and pressure $p=F - bF_b$ as
\begin{align}\label{eq:rho_p_fluctuations_via_b_y}
\begin{aligned}
    \delta\rho &= y^2F_{yy}\delta y + (-bF_b+byF_{by})\delta b
    \\
    &= (y^2F_{yy})\left(\tfrac12 h^{00}+ \dot{\pi}^0\right) + (-bF_b+byF_{by})\left(-\tfrac12 h^{ii} +(\partial_j \pi^j)\right) + O(2)
    \\[3pt]
    \delta p &= (yF_y-byF_{by})\delta y - (b^2 F_{bb})\delta b
    \\
    &= (yF_y-byF_{by})\left(\tfrac12 h^{00}+ \dot{\pi}^0\right) - (b^2 F_{bb})\left(-\tfrac12 h^{ii} +(\partial_j \pi^j)\right)+ O(2)\,.
\end{aligned}
\end{align}


We can now expand the action up to second order 
{\small
\begin{align}
\begin{aligned}
    S&=\int d^4x \sqrt{-g} F(b,y) = \int d^4x\,a^4 \big[1+\tfrac12 (h^{ii}-h^{00}) \big] F(\bar{b}(1+\delta b), \bar{y}(1+\delta y))
    \\
    &= \int d^4x\, a^4 \big[1+\tfrac12 (h^{ii}-h^{00}) \big]
    \Bigg[ F(\bar{b}, \bar{y}) + (\bar{b}F_b)  (\delta b^{(1)} + \delta b^{(2)}) + (\bar{y}F_y) (\delta y^{(1)} + \delta y^{(2)})\\
    &\qquad\qquad\qquad\qquad\qquad\qquad\qquad+ \tfrac12 (\bar{b}^2 F_{bb})(\delta b^{(1)})^2 + \tfrac12 (\bar{y}^2 F_{yy}) (\delta y^{(1)})^2
    + (\bar{b}\bar{y} F_{by}) \delta b^{(1)} \delta y^{(1)} \Bigg]
\end{aligned}
\end{align}
}
The quadratic part of the action for fluid perturbations in a general curved background is therefore
{\small
\begin{align}
\begin{aligned}
    S^{(2)}&=\int d^4x\,\, a^4 \bigg\{
        (\bar{b}F_b) (h^{ii}-h^{00}) \Big[(\partial_j \pi^j) - \tfrac12 h^{ii}\Big]
        + (\bar{y}F_y) (h^{ii}-h^{00}) \Big[\tfrac12 h^{00}+ \dot{\pi}^0\Big]\\
    &\quad + (\bar{b}F_b) \Big[ - h^{0 i} \dot{\pi}^i - \tfrac12  h^{ii} (\partial_j \pi^j) - \tfrac12 (\dot{\pi}^i)^2 + \tfrac12 (\partial_j \pi^j)^2 -\tfrac12 \partial_i\pi^j\partial_j\pi^i\Big]
    \\
    &\quad + (\bar{y}F_y) \Big[ \tfrac12 h^{00}\dot{\pi}^0+ h^{0 i} \dot{\pi}^i + \tfrac12 (\dot{\pi}^i)^2 -\partial_j\pi^0\dot{\pi}^j\Big]
    \\
    &\quad + \tfrac12 (\bar{b}^2 F_{bb})\Big[ (\partial_j \pi^j) - \tfrac12 h^{ii}\Big]^2 
    + \tfrac12 (\bar{y}^2 F_{yy})\Big[\tfrac12 h^{00}+ \dot{\pi}^0\Big]^2
    + (\bar{b}\bar{y} F_{by}) \Big[(\partial_j \pi^j) - \tfrac12 h^{ii}\Big]\Big[\tfrac12 h^{00}+ \dot{\pi}^0\Big]
        \bigg\}.
\end{aligned}
\end{align}
}
Dropping  further second orders in metric perturbations and {\color{blue} integrating by parts in space}, we get
{\small
\begin{align}
\begin{aligned}
    S^{(2)}=
    \int d^4x\,\, a^4 \bigg\{&
        (\bar{b}F_b) (h^{ii}-h^{00}) (\partial_j \pi^j)
        + (\bar{y}F_y) (h^{ii}-h^{00}) \dot{\pi}^0 \\
    &\quad + (\bar{b}F_b) \Big[ - h^{0 i} \dot{\pi}^i - \tfrac12  h^{ii} (\partial_j \pi^j) - \tfrac12 (\dot{\pi}^i)^2 + \bluecancel{\tfrac12 (\partial_j \pi^j)^2} -\bluecancel{\tfrac12 \partial_i\pi^j\partial_j\pi^i}\Big]
    \\
    &\quad + (\bar{y}F_y) \Big[ \tfrac12 h^{00}\dot{\pi}^0+ h^{0 i} \dot{\pi}^i + \tfrac12 (\dot{\pi}^i)^2 -\partial_j\pi^0\dot{\pi}^j\Big]
    \\
    &\quad + \tfrac12 (\bar{b}^2 F_{bb})\Big[ (\partial_j \pi^j)^2 - h^{ii}(\partial_j \pi^j)\Big]
    + \tfrac12 (\bar{y}^2 F_{yy})\Big[h^{00} \dot{\pi}^0+ (\dot{\pi}^0)^2\Big]\\
    &\quad+ (\bar{b}\bar{y} F_{by}) \Big[ \dot{\pi}^0(\partial_j \pi^j) - \tfrac12 h^{ii}\dot{\pi}^0 + \tfrac12 h^{00} (\partial_j\pi^j)\Big]
        \bigg\}
\end{aligned}
\end{align}
}
Collecting terms, the final expression for the quadratic action is
{\small
\begin{align}\label{eq:final_quadratic_action_curved_spacetime}
\begin{aligned}
    S^{(2)}=
    &\int d^4x\,\,a^4\,\,\bigg\{
    \tfrac12 (yF_y-bF_b) (\dot{\pi}^i)^2+\tfrac12 b^2 F_{bb} (\partial_j \pi^j)^2
    +\tfrac12 y^2 F_{yy} (\dot{\pi}^0)^2 \\
    &\qquad\qquad\qquad+ (b y F_{by}) (\dot{\pi}^0 \partial_j \pi^j)-(yF_y) (\partial_j \pi^0\dot{\pi}^j)
    \bigg\}\\
    &\quad + a^4\,\,\bigg\{
    \tfrac12\!\left(\bar{b}F_b - \bar{b}^2 F_{bb}\right) h^{ii}(\partial_j \pi^j)
    + \tfrac12\!\left(-2\bar{b}F_b + \bar{b}\bar{y}F_{by}\right) h^{00}(\partial_j \pi^j)\\
    &\qquad\qquad+ \left(\bar{y}F_y - \bar{b}F_b\right) h^{0 i}\,\dot{\pi}^j + \tfrac12\!\left(2\bar{y}F_y - \bar{b}\bar{y}F_{by}\right) h^{ii}\dot{\pi}^0
    + \tfrac12\!\left(\bar{y}^2 F_{yy} - \bar{y}F_y\right) h^{00}\dot{\pi}^0
    \bigg\}
\end{aligned}
\end{align}
}
Note that only \emph{longitudinal} modes $\pi^j_L\propto\, \partial_j\pi^j$ results dynamical.

Finally integrating by parts in both time and space the term $\partial_j \pi^0\dot{\pi}^j$ we get\todoAC{Collect $\rho+p$ \& rewrite}
{\small
\begin{align}
\begin{aligned}
    S^{(2)}=
    &\int d^4x\,\,a^4\,\,\bigg\{
    \tfrac12 (yF_y-bF_b) (\dot{\pi}^i)^2+ \tfrac12 b^2 F_{bb} (\partial_j \pi^j)^2 +\tfrac12 y^2 F_{yy} (\dot{\pi}^0)^2\\
    &\qquad\qquad\qquad-\Big[ 4\mathcal{H}(yF_y)+ \tfrac{d}{d\tau}(yF_y)\Big] (\partial_j \pi^j) \pi^{0}
    + (b y F_{by}-yF_y) (\dot{\pi}^0 \partial_j \pi^j)
    \bigg\}\\
    &\quad + a^4\,\,\bigg\{
    \tfrac12\!\left(\bar{b}F_b - \bar{b}^2 F_{bb}\right) h^{ii}(\partial_j \pi^j)
    + \tfrac12\!\left(-2\bar{b}F_b + \bar{b}\bar{y}F_{by}\right) h^{00}(\partial_j \pi^j)\\
    &\qquad\qquad+\left(\bar{y}F_y - \bar{b}F_b\right) h^{0 i}\,\dot{\pi}^j +\tfrac12\!\left(2\bar{y}F_y - \bar{b}\bar{y}F_{by}\right) h^{ii}\dot{\pi}^0
    + \frac12\!\left(\bar{y}^2 F_{yy} - \bar{y}F_y\right) h^{00}\dot{\pi}^0
    \bigg\}
\end{aligned}
\end{align}
}
%
Resolving the constraint for $\pi^0$ we now get\todotag{Do it}
{\small
\begin{align}
\begin{aligned}
    ...
\end{aligned}
\end{align}
}
%
We can now specialize this to specific choices of gauge...
\bigskip


%-----------------------------------------------------------------
\subsection{Quadratic action in flat spacetime}
%-------------------------------------------------------
In flat spacetime $a=1$ and $h_{\mu\nu}=0$, only the first terms in the quadratic action \eqref{eq:final_quadratic_action_curved_spacetime} survive, and we can integrate by parts in both space and time the term $\partial_j \pi^0\dot{\pi}^j = -\dot{\pi}^0(\partial_j \pi^j)$ to get
{\small
\begin{align}
\begin{aligned}
    S^{(2)}&= \int d^4x \frac{(yF_y-bF_b)}{2}\bigg\{(\dot{\pi}^i)^2 - (\partial_j\pi^j)^2\Big[-\frac{b^2F_{bb}}{(yF_y-bF_b)}\Big]\\
    &\qquad\qquad\qquad\qquad\qquad+(\dot{\pi}^0)^2 \frac{y^2F_{yy}}{(yF_y-bF_b)}
    - 2\dot{\pi}^0 (\partial_j \pi^j) \frac{(yF_y-byF_{by})}{(yF_y-bF_b)}
    \bigg\}\\
    &=\int d^4x \frac{(\rho+p)}{2}\bigg\{(\dot{\pi}^i)^2 - \mathcal{C} (\partial_j\pi^j)^2 
    + (\dot{\pi}^0)^2 \mathcal{A}
    - 2\mathcal{M}\dot{\pi}^0 (\partial_j \pi^j) 
    \bigg\}
\end{aligned}
\end{align}
}
for $\rho= -F + y F_y$, $p= F - b F_b$ and 
{\small
\begin{align}
\begin{aligned}
    &\mathcal{C} = -\frac{b^2F_{bb}}{(yF_y-bF_b)},\quad 
    \mathcal{A} = \frac{y^2F_{yy}}{(yF_y-bF_b)} = \frac{T\, \frac{\partial\rho }{\partial T}}{\rho+p},\quad
    \mathcal{M} = \frac{(yF_y-byF_{by})}{(yF_y-bF_b)}.
\end{aligned}
\end{align}
}
The entropy field $\pi^0$ is thus non-dynamical and the constraint equations gives
\begin{align}
    \dot{\pi}^0 = \frac{\mathcal{M}}{\mathcal{A}} (\partial_j \pi^j) + \text{const.}
\end{align}
The constant is not physical and is reabsorbed in the original definition of $\phi^0=\gamma(t+\pi^0)$.
Solving the constraint the action for the coordinate $\pi^i$ becomes
{\small
\begin{align}
    S^{(2)}&= \int d^4x \frac{(\rho+p)}{2}\Big[(\dot{\pi}^i)^2 - c_s^2 (\partial_j\pi^j)^2 \Big], \quad \text{for}\quad c_s^2 = \mathcal{C} + \frac{\mathcal{M}^2}{\mathcal{A}}.
\end{align}
}
We note the entropy contribution to the speed of sound is always positive in \emph{physical} situations.
Indeed $\mathcal{M}^2$ is a true square and, recalling $y\equiv T$, we have
\begin{equation}
    \mathcal{A}= \frac{T\, \frac{\partial\rho }{\partial T}}{\rho+p}\geq 0 \quad\text{provided}\quad
    \begin{cases}
        \rho+p\geq 0 \quad(\text{no ghosts}),\\
        \frac{\partial\rho }{\partial T} \geq 0 \quad(\text{positive heat capacity}).
    \end{cases} 
\end{equation}
Finally we immediately check the speed of sound $c_s^2$, defined as the coefficient of the fluctuations above, always matches the thermodynamical definition of \emph{adiabatic sound speed}
\begin{align}\label{eq:cs_equiv_cb_flat_spacetime}
    c_b^2:= \frac{\partial p}{\partial \rho}_{\mid \sigma} \overset{\text{proved before}}{=}\frac{-b^2F_{bb} F_{yy}+(F_y-bF_{yb})^2}{F_{yy}\,(yF_y-bF_b)} \overset{\text{easy check}}{\equiv} c_s^2.
\end{align}