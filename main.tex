\documentclass[a4paper,11pt]{article}
%\pdfoutput=1 

\usepackage{graphicx}  % needed for figures
\usepackage{dcolumn}   % needed for some tables
\usepackage{bm,relsize}        % for math
\usepackage{amssymb, amsmath}
\usepackage{textcomp}
\usepackage{wasysym}
\usepackage{slashed}
\usepackage{caption, subcaption}
\usepackage{multirow}
\usepackage{gensymb}
\usepackage{subcaption}
%\usepackage[table]{xcolor}
\usepackage{colortbl}
\usepackage{booktabs} % opzionale ma utile per tabelle più eleganti
\usepackage{tabularx}
\definecolor{headergray}{gray}{0.9}
\definecolor{rowgray}{gray}{0.97}


\usepackage[nolist, nohyperlinks]{acronym}


\usepackage{lipsum, color}
\usepackage[dvipsnames,svgnames,table]{xcolor}
\usepackage{braket}
\usepackage{jheppub} 
\usepackage{booktabs}
\usepackage{pdflscape}
\usepackage[utf8]{inputenc} 

\usepackage{booktabs}
\usepackage{tabularx}

\usepackage{mathrsfs}
\usepackage{bm,amssymb,slashed,graphicx,multirow,soul,mathtools,xspace,array,tikz,amsmath, gensymb}
\usetikzlibrary{patterns}
\usepackage{siunitx} 
\usepackage{float}   
\usepackage{cancel}
\allowdisplaybreaks
\usepackage{ bbold }
%\usepackage{subfigure}
\usepackage{caption,subcaption}
\usepackage{hyperref}
\usepackage{colortbl}
\usepackage{tcolorbox}

\usepackage{comment}

\usepackage[capitalise, english]{cleveref}

\definecolor{nicered}{rgb}{0.7,0.1,0.1}
\definecolor{nicegreen}{rgb}{0.1,0.5,0.1}
\definecolor{violet}{rgb}{0.7,0.3,0.3}
\hypersetup{colorlinks,citecolor= nicegreen,linkcolor= nicered}

\setcounter{tocdepth}{2}

\newcommand{\lp}{\left(}
\newcommand{\rp}{\right)}
\newcommand{\ov}{\overline}
\newcommand{\ds}{\displaystyle}
\newcommand{\g}{\gamma}
\newcommand{\be}{\begin{equation}}
\newcommand{\ee}{\end{equation}}

\newcommand{\eventname}{KM3-230213A}
\newcommand{\kmn}{KM3NeT}
\newcommand{\ic}{IceCube}

\newcommand{\nn}{\nonumber}
\newcommand{\TeV}{\si{\tera\electronvolt}}
\newcommand{\GeV}{\si{\giga\electronvolt}}
\newcommand{\PeV}{\si{\peta\electronvolt}}
\newcommand{\Br}{\text{Br}}

\newcommand {\vek}[1]{\mathbf{#1}}
\newcommand {\E}[1]{\times 10^{#1}}	% scientific exponent notation
\newcommand {\e}[1]{\mathrm{~#1}}       % units
\newcommand{\re}[0]{\mathrm{Re}\,}
\newcommand{\im}[0]{\mathrm{Im}\,}
\newcommand{\mc}[1]{\mathcal{#1}}
\newcommand{\dd}[0]{\mathrm{d}}
\newcommand{\eq}[1]{\begin{equation} #1 \end{equation}}
\newcommand{\beq}{\begin{equation} }
\newcommand{\eeq}{\end{equation}} 
\newcommand{\bi}{\begin{itemize} }
\newcommand{\ei}{\end{itemize} }
\newcommand{\EeV}{\mathrm{EeV}}
\newcommand{\TT}{\mathcal{T}}

\newcommand{\deriv}[2]{\frac{\partial #1}{\partial #2}}
\newcommand{\totalderiv}[2]{\frac{d #1}{d #2}}

\definecolor{Red}{rgb}{1.,0.,0.}
\definecolor{Grn}{rgb}{0.,0.75,0.}
\definecolor{Blu}{rgb}{0.,0.,1.}
\definecolor{Pink}{rgb}{1,0.08,0.58}
\newcommand{\Red}[1]{{\color{nicered}{#1}}}
\newcommand{\Grn}[1]{{\color{nicegreen}{#1}}}
\newcommand{\Blu}[1]{{\color{niceblue}{#1}}}        
   


\DeclareMathOperator{\diag}{diag}   
\let\Re\relax
\DeclareMathOperator{\Re}{Re}
\let\Im\relax
\DeclareMathOperator{\Im}{Im}
\DeclareMathOperator{\Tr}{Tr}
\newcommand{\lrpartial}{\negthickspace\stackrel{\leftrightarrow}{\partial}\negthickspace{}}
\newcommand{\lrPartial}{\negthickspace\stackrel{\leftrightarrow}{D}\negthickspace{}}

\usepackage[T1]{fontenc} % if needed

\usepackage{amsmath,amssymb,epsfig,color,slashed}
\allowdisplaybreaks  

\setcounter{MaxMatrixCols}{20}

\newcommand*{\I}{\mathrm{i}}

\newcommand\scalemath[2]{\scalebox{#1}{\mbox{\ensuremath{\displaystyle #2}}}}



\definecolor{verdino}{rgb}{0.66, 0.89, 0.63}

\bibliographystyle{JHEP}

\newcommand{\DR}[1]{{\color{blue}[DR: #1]}}
\newcommand{\CA}[1]{{\color{nicered}[CA: #1]}}



%============= PREAMBLE BY ANDREA ==================
%====================================================

%\usepackage{comment} % for \begin{comment} environment %already loaded above
%\usepackage{xspace} %already loaded above
\usepackage{todonotes} % for TODO tags, switch tags off with [disable] flag

\definecolor{AnswerColor}{RGB}{100,255,100} % green
\definecolor{NoteColor}{RGB}{255,255,170} % pale yellow
\definecolor{QuestionColor}{RGB}{255,220,150} % light orange
\definecolor{TodoColor}{RGB}{255,170,170}   % pale red

%-------------------- tags: answer, note, question, todo ----------------------
\newcommand{\notetag}[1]{
  {\setlength{\fboxsep}{1pt}   \todo[backgroundcolor=NoteColor, bordercolor=black, linecolor=black, size=\scriptsize]{#1}}}
\newcommand{\questiontag}[1]{
  {\setlength{\fboxsep}{1pt}   \todo[backgroundcolor=QuestionColor, bordercolor=black, linecolor=black, size=\scriptsize]{#1}}}
\newcommand{\todotag}[1]{
  {\setlength{\fboxsep}{1pt}  \todo[backgroundcolor=TodoColor, bordercolor=black, linecolor=black, size=\scriptsize]{#1}}}
\newcommand{\answertag}[1]{
  {\setlength{\fboxsep}{1pt}  \todo[backgroundcolor=AnswerColor, bordercolor=black, linecolor=black, size=\scriptsize]{#1}}}

\newcommand{\noteAC}[1]{
  {\setlength{\fboxsep}{1pt}\todo[backgroundcolor=NoteColor, bordercolor=black, linecolor=black, size=\scriptsize]{AC: #1}}}
\newcommand{\questionAC}[1]{
  {\setlength{\fboxsep}{1pt}\todo[backgroundcolor=QuestionColor, bordercolor=black, linecolor=black, size=\scriptsize]{AC: #1}}}
\newcommand{\todoAC}[1]{
  {\setlength{\fboxsep}{1pt}\todo[backgroundcolor=TodoColor, bordercolor=black, linecolor=black, size=\scriptsize]{AC: #1}}}
\newcommand{\answerAC}[1]{
  {\setlength{\fboxsep}{1pt}\todo[backgroundcolor=AnswerColor, bordercolor=black, linecolor=black, size=\scriptsize]{AC: #1}}}

%--------------- Colored cancel lines to strike through terms----------
\newcommand{\redcancel}[1]{\renewcommand{\CancelColor}{\color{red}}\cancel{#1}}
\newcommand{\bluecancel}[1]{\renewcommand{\CancelColor}{\color{blue}}\cancel{#1}}
\newcommand{\greencancel}[1]{\renewcommand{\CancelColor}{\color{green}}\cancel{#1}}
\newcommand{\graycancel}[1]{\renewcommand{\CancelColor}{\color{gray}}\cancel{#1}}
\newcommand{\purplecancel}[1]{\renewcommand{\CancelColor}{\color{purple}}\cancel{#1}}
%=================================================



\begin{document} 

\begin{acronym}
\acro{IC}{IceCube}
\end{acronym}

%\preprint{}



\title{Accelerating the Universe with non-barotropic fluids}

\author[a]{Andrea Clini,}
\author[a]{Michele Redi,}
\author[b]{Diego Redigolo,}
\author[b]{Lorenzo La Penna,}

\affiliation[a]{Dipartimento di Fisica “Aldo Pontremoli”, Universit\`a degli Studi di Milano,
Via Celoria 16, 20133 Milan, Italy}
\affiliation[b]{INFN Sezione di Firenze, Via G. Sansone 1, I-50019 Sesto Fiorentino, Italy}
        
\date{\today} 


\abstract{Long-lasting accelerated expansion cannot be supported by perfect barotropic fluid where pressure fluctuations are fully determined by the energy density ones. The manifestation of this obstruction is that the adiabatic sound speed is fully determined by the equation of state so that requiring a negative equation of state to enforce acceleration unavoidably lead to gradient instabilities. We show that non-barotropic perfect fluids allow for accelerated expansion by decoupling the speed of sound from the equation of state similarly to what happens in generalized quintessence models. fluid stability requires: $w>-1$ and $0<c_s^2<1$. The former seems intimately connected with the absence of dissipation at least at leading order in the derivative expansion. Non-barotropic perfect fluids are somewhat non-generic from the ultraviolet perspective }

\maketitle



%%%%%%%%%%%%% FRONT %%%%%%%%%%%%%%%%%%%%%%%%%%%%%%%%%

\section{Introduction}
Which fluid allows for accelerated expansion? Assuming a single fluid dominates the energy density of the Universe we can relate the acceleratio of the universe to the equation of state of the fluid that drives it
\begin{equation}
    \frac{\ddot{a}}{a}=-\frac{4\pi G_N\rho}{3}(1+3w)\,,
\end{equation}
where $w=p/\rho$ is the equation of state of the so-called dark energy. This equation tells us the well known thing that having accelerated expansion implies $w<-1/3$. We also need to check that the accelerating background is stable which amounts to exclude the presence of  instabilities in the fluctuations around it. 

Moreover, given the little we know about the nature of dark matter we might want to look for a overarching framework describing the behavior of both dark matter and dark energy in terms of the properties that control their interaction with Einstein gravity. This exercise has of course also phenomenological relevance because through their gravitational imprints in the large scale structure of the Universe we could be able to observe the properties of these misterious fluids.  

Very much in the spirit of Ref.~\cite{Hu:1998kj} we would like to be guided by symmetries to construct this general fluid. Contrary to older approaches we will consistently make use of the lagrangian effective field theory framework to describe fluids. 






%==============================================
\section{Dark Energy as a single fluid}
%=============================================
Here we summarize some key results about fluid in the literature.


%---------------------------------------------------------
\subsection{Barotropic perfect fluid can't be dark energy}
%-------------------------------------------------------
A relativistic barotropic perfect fluid is described by three scalar fields $\phi^I(x)$ with $I=1,2,3$ which label the fluid elements~\cite{Dubovsky:2005xd}. The lagrangian should be invariant under volume-preserving diffeos
\begin{equation}
\phi^I\to f^I(\phi)\quad ,\quad \det\frac{\partial f}{\partial \phi^I}=1\ .   
\end{equation}
We define
\begin{equation}
    B^{IJ}\equiv g^{\mu\nu}\partial_\mu\phi^I\partial_\nu\phi^J\quad,\quad b=\sqrt{\det B_{IJ}}\, ,   \label{eq:b}
\end{equation}
which is invariant under volume-preserving diffeos. The physical interpretaion of $b$ can be understood in terms of the conservation of the comoving volume current $b=\sqrt{-J^\mu J_\mu}$, where 
\begin{equation}
    J^\mu_n=\frac{1}{6\sqrt{-g}}\epsilon^{\mu\nu\rho\sigma}\epsilon_{IJK}\partial_\nu\phi^I\partial_\rho\phi^I\partial_\sigma\phi^K\quad \rm{s.t.}\quad  \nabla_\mu J^\mu=0  \,.  
\end{equation}
This allows to define the flow velocity orthogonal to the constant $\Phi^I$ surfaces
\begin{equation}
J^\mu_n=b u^\mu\quad \rm{s.t}\quad u^\mu u_\mu=-1\ ,
\end{equation}
where the definition of the current makes it manifest that $b$ is the comoving number density.
In the barotropic case $b$ also coincides with the entropy density, and it is conserved in the absence of dissipation. The entropy per particle $\sigma=s/n$ is exactly constant in this case, and the entropy is directly proportional to the number density $s\propto n$.

The action is then specified by a single function
\begin{equation}
S=\int d^4x\sqrt{-g} F(b)\ .    
\end{equation}
The function $F(b)$ is in one to one correspondence with the equation of state, which also fixes the behavior of fluctuations. Explicitly we can derive the background thermodynamic by writing the stress energy tensor and matching it to the one of a perfect fluid 
\begin{equation}
    T_{\mu\nu}=(p+\rho)u_\mu u_\nu+p g_{\mu\nu}=F_{IJ} \partial_\mu\phi^I \partial_\nu\phi^I-g_{\mu\nu} F\ ,
\end{equation}
This results in
\begin{equation}
    \rho=-F\,,\quad p=F-bF_b\,, \quad \rho+p=-bF_b\,,
\end{equation}
and the equation of state 
\begin{equation}
    w\equiv\frac{p}{\rho}=-1+\frac{b F_b}{F}\ .
\end{equation}
Notice that at the level of the background the barotropic perfect fluid is fully specified by a single function which specifies its equation of state.  

Explicitly we can write the lagrangian at quadratic order in the fluctuation around a fluid background configuration $\phi^I=x^i+\pi^I$
\begin{equation}
S_2^{\rm{baro}}=\int d^4x\sqrt{-g}\,\,\left(\frac{p+\rho}{2}\right)\left[\dot\pi^2-c_s^2(\nabla\pi)^2\right]\ ,
\end{equation}
where\footnote{Coupling a barotropic fluid to FRW the conservation of the entropy current implies $\nabla_\mu J^{\mu}=\dot{b}+3Hb=0$.} 
\begin{equation}
 c_s^2=\frac{b F_{bb}}{ F_b}=\frac{d p}{d\rho}=\frac{\dot{p}}{\dot{\rho}}=w-\frac{\dot{w}}{3H(1+w)}\,,  \label{eq:adiabatic}
\end{equation}
and it is fully determined by the equation of state as expected for a barotropic fluid. 
Notice that the absence of ghosts, and gradient instabilities and superluminal modes imply respectively 
\begin{equation}
w>-1\,\qquad 0<c_s^2\leq1\ .
\end{equation}
Moreover, using $c_s^2(w)$ derived from \eqref{eq:adiabatic} we can rewrite the absence of gradient instabilities as a further constraint on the equation of state 
\begin{equation}
w-\frac{\dot{w}}{3H(1+w)}>0\ , \label{eq:nogradbaro}
\end{equation}
which parametrizing $\dot w= \beta_w H w$ becomes 
\begin{equation}
 w\left(1-\frac{\beta_w}{3(1+w)}\right)>0\ .   
\end{equation}
Notice that, in order to have $-1<w<-1/3$ and hence accelerated expansion, stability requires $\beta_w>3(1+w)\sim\mathcal{O}(1)$ which then implies that $w$ must evolve rapidly, making a long accelerating phase impossible. Indeed going back to Eq.~\eqref{eq:nogradbaro} and assuming constant equation of state we get $w>0$ which is incompatible with accelerated expansion. 


As usual from the conservation of the stress tensor in a perturbed FRW background
\begin{equation}
ds^2=-(1+2\Phi)dt^2+a^2(1-2\Psi)dx^2 
\end{equation}
we can derive the equation controlling the behavior of the fluctuations of energy density and pressure: energy conservation gives the continuity equation and momentum conservation the Euler equation. Crucially for a barotropic fluid 
\begin{equation}
\delta p=c_s^2\delta\rho\,,
\end{equation}
so that the above conservation equations heavily simplify
\begin{align}
&\dot{\delta\rho}+3H(1+c_s^2)\delta\rho-(1+w)\rho(3\dot{\Phi}+\frac{k^2}{a^2}v)=0\\
&(\dot v+\Phi)+\frac{c_s^2}{1+w}\frac{\delta\rho}{\rho}=0
\end{align}
where the velocity potential is $\delta u_i=\partial_i v$. 
Differentiate continuity w.r.t. time and eliminate the velocity derivative using Euler to get an equation for the density contrast $\delta=\delta\rho/\rho$:
\begin{equation}
\ddot{\delta\rho}+3H(1+c_s^2))\dot{\delta\rho}+\left[\frac{c_s^2k^2}{a^2}-4\pi G(1+w)\rho\right]\delta\rho=(1+w)\left[\ddot{\Phi}+6H\dot{\Phi}\right]\ .
\end{equation}
Now for $k^2/a^2\gg H^2$ and $\dot{\Phi}\approx 0$ the equation reduces to 
\begin{equation}
\ddot{\delta\rho}+3H(1+c_s^2)\dot{\delta\rho}+\left[\frac{c_s^2k^2}{a^2}-4\pi G(1+w)(1+3c_s^2)\rho+3H\dot c_s^2\right]\delta\rho=0\ ,
\end{equation}
and we recognize the usual competition between pressure support (for $c_s^2>0$) and gravitational instability. The barotropicity of the fluid implies that the adiabatic sound speed completely fixes the pressure support. 

\subsection{Non-barotropic perfect fluid}
To describe a fluid with entropy we can follow \cite{Dubovsky:2011sj,Ballesteros:2016kdx} and just add an extra scalar $\Phi^0$ to the construction above which enjoys a shift symmetry. 
\begin{equation}
\Phi^0\to\Phi^0+c\label{eq:shifttime}
\end{equation}
which allow us to define a new invariant 
\begin{equation}
y=u^\mu\partial_\mu\Phi^0
\end{equation}
which plays the role of the temperature or the chemical potential. This quantity will control the entropy per particle independently on the particle number density which is always controlled by $b$ defined in Eq.~\eqref{eq:b}.

The action is now described by a single function of two scalar quantities
\begin{equation}
S=\int d^4 x\sqrt{-g} F(b,y)\ .
\end{equation}
From this we can derive the stress energy tensor
\begin{equation}
T_{\mu\nu}=(-b F_b+y F_y )\,u_\mu u_\nu
           + (F - b F_b)\,g_{\mu\nu}\ ,
\end{equation}
which matching to the perfect fluid expression gives
\begin{equation}\label{eq:rho_pressure_exp_nonbaro}
\rho = - F+yF_y , \qquad p = F- b F_b \ .
\end{equation}
so that 
\begin{equation}
w=\frac{ F- b F_b}{- F+yF_y}\,. 
\end{equation}

Notice that in this case the presence of $y$ makes the entropy per particle not fixed by the equation of state but still conserved along the flow lines as expected in the absence of dissipation. In the EFT language the entropy per particle current is nothing else than the Noether current associated to the shift symmetry in Eq.~\eqref{eq:shifttime}
\begin{equation}
J_s^{\mu}=F_y u_\mu\quad \rm{s.t.}\quad \nabla_\mu J^\mu_s=0\ .
\end{equation}
The entropy per particle is not constant in this case, and it can be identified as
\begin{align}
    \sigma=\frac{s}{n}=\frac{F_y}{b}\ .
\end{align}
The adiabatic sound speed is then by definition
\begin{equation}
c_b^2:=\frac{dp}{d\rho}_{\mid\sigma}=w+\rho\frac{dw}{d\rho}_{\mid\sigma}=\frac{-b^2F_{bb} F_{yy}+(F_y-bF_{yb})^2}{F_{yy}\,(yF_y-bF_b)} = \frac{b\,p_b \,y\rho_y + (y\,p_y)^2}{y \rho_y (\rho+p)}\,.
\end{equation}
Finally, identifying the system temperature with $y=T$, then \eqref{eq:rho_pressure_exp_nonbaro} implies the function $F$ is minus the Helmholtz free energy density 
\begin{equation}
    F=-(\rho-Ts)\equiv-\mathfrak{a}\,.
\end{equation}

Explicitly we can write the lagrangian at quadratic order in the fluctuation around a fluid background configuration $\phi^0=t+\pi^0$ and $\phi^I=x^i+\pi^I$
\begin{equation}
S_2=\int d^4x\sqrt{-g}\left(\frac{p+\rho}{2}\right)\left[\dot\pi^2-\mathcal{C}_s^2(\nabla\pi)^2-2\mathcal{M}\dot{\pi}_0 (\nabla\!\!\cdot\!\pi)+\mathcal{A} \dot{\pi}_0^2\right]\ , 
\end{equation}
the sound speed is now 
\begin{equation}
\mathcal{C} = -\frac{b^2F_{bb}}{(yF_y-bF_b)}\,.
\end{equation}
The mixing of the entropy mode with the phonons is given by 
\begin{equation}
\mathcal{M} = \frac{(yF_y-byF_{by})}{(yF_y-bF_b)}.
\end{equation}
and the inertia of the entropy mode is
\begin{equation}
\mathcal{A}= \frac{y^2 F_{yy}}{y F_y-b F_b}\,.
\end{equation}
Crucially in this lagrangian only $p+\rho>0$ is necessary to avoid ghost instabilities.
In case $\rho+p>0$, thermodynamic stability also demands $\mathcal{A}>0$, however at this level the speed of sound $c_s^2$ can be negative.


The entropy mode is non propagating and can be integrated out as a constraint to get 
\begin{equation}
S_2=\int d^4xd^4x\sqrt{-g}\left(\frac{p+\rho}{2}\right)\left[\dot\pi^2-c_s^2(\nabla\pi)^2\right]\ , 
\end{equation}
where
\begin{equation}
    c_s^2:=\mathcal{C}_s^2+\frac{\mathcal{M}^2}{\mathcal{A}}\,.\;\label{eq:csnonbaro} 
\end{equation}
It is immediate to check the speed of sound $c_s^2$, defined as the coefficient of the fluctuations, always matches the thermodynamical definition of \emph{adiabatic sound speed} $c_b^2:= \frac{\partial p}{\partial \rho}_{\mid \sigma}$.
The entropy contribution to the speed of sound is always positive in \emph{physical} situations since
\begin{equation}
    \mathcal{A}= \frac{T\, \frac{\partial\rho }{\partial T}}{\rho+p}\geq 0 \quad\text{provided}\quad
    \begin{cases}
        \rho+p\geq 0 \quad(\text{no ghosts}),\\
        \frac{\partial\rho }{\partial T} \geq 0 \quad(\text{positive heat capacity}).
    \end{cases} 
\end{equation}
Now $c_s^2$ is actually an independent function of $c_a^2$. So that requiring the absence of gradient instabilities does not forbid accelerated expansion which is only constrained by the absence of ghosts. As a consequnce the non-barotropic fluid EFT can span the whole space of 
\begin{equation}
w>-1\quad 0<c_s^2\leq1\ .    
\end{equation}

In order to understand non barotropic fluids it is important to write the pressure fluctuations. These can be written in two ways 
\begin{align}
\delta p &=c_s^2\delta\rho+\Gamma\delta \sigma\ ,
\end{align}
where we defined $c_s^2=\partial p/\partial \rho\vert_\sigma$ is exactly the as the speed of sound appearing in the phonon Lagrangian defined in Eq.~\eqref{eq:csnonbaro}. In addition to the speed of sound the pressure fluctuation receive contribution from the fluctuation of the entropy mode $\Gamma=\partial p/\partial \sigma\vert_\rho$. 
Now we can write the continuity and Euler equation.
Consider the metric in conformal time $ds^2=a^2\left[-(1+2\Phi)d\tau^2+(1-2\Psi)dx^2\right]$.
Decomposing the energy-momentum tensor as $T^{\mu\nu}=(\rho+P)u^\mu u^\nu+Pg^{\mu\nu}+a^{-2}\Pi^{\mu\nu}$ with $\Pi^{00}=\Pi^{i0}=0$ and $\Pi^{ij}=(\partial_i\partial_j-\tfrac{1}{3}\delta_{ij}\partial^2)\Pi$, we have
\begin{align}
    \delta\rho^\prime= -3\mathcal{H}(\delta\rho+\delta P) + (\bar{\rho}+\bar{P})\big(3\psi^\prime-\partial_j \,(\delta u^j)\big)\,\\
    (\delta u^j)^\prime+ \mathcal{H}(\delta u^j)-3\mathcal{H}\frac{\bar{P}^\prime}{\bar{\rho}^\prime}(\delta u^j)+\frac{\partial_j\delta P}{(\bar{\rho}+\bar{P})}+\partial_j\Phi{\color{red}+\frac{\partial_\ell \Pi^{\ell j}}{(\bar{\rho}+\bar{P})}}=0\,.
\end{align}
where we write $u^\mu=\frac{1}{a}(1-\Phi, \delta u^j)$. \todotag{Check convention with Diego}
Introducing the velocity potential $\delta u^j=\partial_j v$ so that $\partial_j\delta u^j=-k^2 v$, and defining the overdensity $\delta:=\delta\rho/\bar{\rho}$ we rewrite
\begin{align}
    &\delta^\prime= 3\mathcal{H}\left(\frac{\bar{P}}{\bar{\rho}}-\frac{\delta P}{\delta\rho}\right)\delta + \left(1+\frac{\bar{P}}{\bar{\rho}}\right)3\psi^\prime + \left(1+\frac{\bar{P}}{\bar{\rho}}\right)k^2 v\,\\
    &v^\prime+ \mathcal{H}v-3\mathcal{H}\frac{\bar{P}^\prime}{\bar{\rho}^\prime}v+\frac{\delta P}{(\bar{\rho}+\bar{P})}+\Phi\,\,{\color{red}-\frac{2}{3}\frac{k^2\,\Pi}{(\bar{\rho}+\bar{P})}}=0\,.
\end{align}
These are combined into a 2nd order equation for overdensities
\begin{align}
\begin{aligned}
    \delta\rho^{\prime\prime}=&-3\big(\dot{\mathcal{H}}+4\mathcal{H}^2\big)\left(1+\frac{\delta P}{\delta\rho}\right)\delta\rho
    -\mathcal{H}(7\delta\rho^\prime+3\delta P^\prime)-k^2\delta P -k^2(\bar{rho}+\bar{P})\Phi\\
    &+3(\bar{\rho}+\bar{P})\left[\left(\mathcal{H}+\frac{\bar{P}^\prime}{\bar{\rho}+\bar{P}}\right)\Psi^\prime+\Psi^{\prime\prime}\right] {\color{red}+ \frac{2}{3}k^4\Pi}\,.
\end{aligned}
\end{align}
Equivalently, in terms of the overdensity $\delta$ we have
\begin{align}
\begin{aligned}
    \delta^{\prime\prime}=&-\delta^\prime\mathcal{H}\left(1-6\frac{\bar{P}}{\bar{\rho}}\right)
    +\delta\left[3(\mathcal{H}^\prime+4\mathcal{H}^2)\left(\frac{\bar{P}}{\bar{\rho}}-\frac{\delta P}{\delta\rho}\right)+3\mathcal{H}\frac{\bar{P}^\prime}{\bar{\rho}}\right]
    -k^2\frac{\delta P}{\bar{\rho}} -k^2\left(1+\frac{\bar{P}}{\bar{\rho}}\right)\Phi\\
    &+3\left(1+\frac{\bar{P}}{\bar{\rho}}\right)\left[\left(\mathcal{H}+\frac{\bar{P}^\prime}{\bar{\rho}+\bar{P}}\right)\Psi^\prime+\Psi^{\prime\prime}\right] {\color{red}+ \frac{2}{3}k^4\frac{\Pi}{\bar{\rho}}}\,.
\end{aligned}
\end{align}
Since the entropy is conserved along the flow $\dot{\delta y}=0$\questionAC{Mean $\dot{\delta \sigma}=0$? } we can write the equation in comoving gauge $(\delta p=0)$ and eliminate the non-propagating entropy fluctuation to get an expression for the sound speed similar to the one in 2.34. {\bf I would like to see the equations for the density constrast with $\Gamma$} {\bf I would like to do the gauging of the fluid with gravity showing that the phonons are identified with the adiabatic curvature mode a' la Weinberg and hence phonon scattering satisfies soft theorems as well as the adibatic curvature mode as it is well known. Don't know if it in this language.}

Essentially we write something like {\bf okkio che la parametrizzazione e' una cazzata tremenda}
\begin{equation}
ds^2=-(1+2\Phi)dt^2+2a \partial_iB dt dx^i+ a^2 e^{2\xi} d\vec{x}^2    
\end{equation}
Now varying with respect to $\Phi$ gives the constraint on $\delta\rho$ and through that one shoud relate $\xi\sim\nabla\pi$. {\bf do this explicitly!} 

\subsection{Scalar fluids}
Here I will match the scalar fluids to the fluid EFTs developed so far. This will allow us to show in which sense the fluid EFT remains more general 

\paragraph{Quintessence} is typically defined as a scalar with a canonical kinetic term $X=-g^{\mu\nu}\partial_\mu\phi\partial_\nu\phi$, so that $\partial_\mu\phi=\sqrt{X}u_\mu$ with $u_\mu u^\mu=-1$, and an arbitrary potential $V(\phi)$
\begin{equation}
S=\int d^4x\sqrt{-g}\left[\tfrac{1}{2}X-V(\phi)\right]
\end{equation}
I can map this scalar theory into the non-barotropic fluid EFT identifying $\Phi^0=\phi$ $\phi^I=x^I$ which is the unitary gauge for the fluid element. Then $b=\sqrt{\det g^{ij}}=a^{-3}$ while $y=u^\mu\partial_\mu\phi=-\sqrt{X}$. Taking 
\begin{equation}
F(b,y)=y^2/2-V(\Phi^0) 
\end{equation}
we get exactly the quintessence lagrangian. Using the general result on the speed of sound we get ...


Plugging this $F$ in the above formulas gives
\begin{align}
    &c_b^2=1\,,\quad \mathcal{C}^2_s=-3\,,\quad\mathcal{M}=-2,\quad \mathcal{A}=1\quad\Rightarrow\quad c_s^2=\mathcal{C}_s^2+\frac{\mathcal{M}^2}{\mathcal{A}}=1\equiv c_b^2\,.
    \\
    &w=\frac{y^2/2-V}{y^2/2+V}\,,\quad c_a^2=\frac{\dot{p}}{\dot{\rho}}=w-\frac{\dot{w}}{3H(1+w)}=1+\frac{2}{3}\frac{V_\phi}{H\dot{\phi}}
\end{align}
Moreover 
\begin{equation}
 \Gamma\delta\sigma=-2 V_\phi \delta\phi   
\end{equation}
 
\paragraph{k-essence} \answerAC{See appendix}
Next we extend to k-essence...

\paragraph{generalized k-essence}

\section{Adding dark matter}
Let us now consider the lagrangian for a two fluid system where the two fluids interact only through gravity 
\begin{equation}
S_2=\sum_{A=1,2}\int d^4x\sqrt{-g}\left(\frac{p_A+\rho_A}{2}\right)\left[\dot\pi_A^2-\mathcal{C}_{s,A}^2(\nabla\pi_A)^2+2\mathcal{M}_A\dot{\pi}_{0,A} (\nabla\!\!\cdot\!\pi_A)+\mathcal{A}_A \dot{\pi}_{0,A}^2\right]\ , 
\end{equation}   
In this setup I can define the sum of the two fluids and their difference. Let us define 
\begin{align}
&\pi_{\rm{tot}}=\frac{(\rho_A+p_A)\pi_A+(\rho_B+p_B)\pi_B}{\rho_A+p_A+\rho_B+p_B}\\
&S\sim\frac{\rho^B_y}{f^B_{yy}} \pi^0_A-\frac{\rho_y^A}{f^A_{yy}} \pi^0_B
\end{align}
The first is the adiabatic mode again while the second (or a similar expression) should encode the relative entropy fluctuations which are now propagating! {\bf this is the idea but it should be done with more care not sure I did the diagonalization correctly}
\section{Introducing bulk viscosity (to be continued)}
The simple example one could think of to construct a dissipative system is to start with two scalar fluids in the UV
\begin{equation}
S=\int d^4x\left[\frac{1}{2}(\partial\phi)^2+\frac{1}{2}(\partial\chi)^2+\frac{m_\chi^2}{2}\chi^2+ y \chi \psi^2+ M_\psi\psi^2+ \epsilon\partial\phi \partial\chi\right]    
\end{equation}
Where we wrote an interaction between the two scalars that respect the shift symmetry and we assume the mass hierarchy $m_\chi>2 M_\psi$. As we will see the presence of fermions is required to give to the heavy scalar $\chi$ a finite width.  
\begin{equation}
\Gamma_\chi\approx\frac{y^2}{8\pi}m_\chi\ ,
\end{equation}
Since we want the evolution of the scalar fluids we need to go to the SK formalism~\cite{Crossley:2015evo,Liu:2018kfw} putting the theory above on a SK time contour. In SK every field is doubled because $\phi_{\pm}, \chi_{\pm},\psi_{\pm}$ encode the time on the forward/backward branch. This is unavoidable to distinguish retarded and advanced response. The SK actions is
\begin{equation}
S_{\rm{SK}}=S[\phi_+,\chi_+]-S[\phi_-,\chi_-]    
\end{equation}
Now I can rotate to the SK variables
\begin{align}
&\phi_{p,m}=\frac{\phi_+\pm \phi_-}{2}
&\chi_{p,m}=\frac{\chi_+\pm \chi_-}{2}
\end{align}
The SK action for the heavy sector is
\begin{equation}
\begin{split}
S^{\rm{heavy}}_{\rm SK}
=
&\int d^4x\left\{\chi_a (\Box - M^2) \chi_r
+
\bar{\psi}_a (i\slashed{\partial}-m_\psi)\psi_r
+
\bar{\psi}_r (i\slashed{\partial}-m_\psi)\psi_a
]\right.
\\
&\left.+y[
\chi_a \bar{\psi}_r \psi_r
+\chi_r(\bar{\psi}_a \psi_r + \bar{\psi}_r \psi_a)\right\}
\end{split}
\end{equation}



\appendix
%=========================================
\section{Computations}
%=========================================
In this section we collect some useful or additional computations.

%--------------------------------------------
\subsection{Thermodynamic formula for adiabatic sound speed}
%------------------------------------------------

The entropy per particle is $\sigma=F_y/b$, imposing it be preserved gives\footnote{Subscripts mean $X_b=\partial_b X_{\mid y}$ and $X_y=\partial_y X_{\mid b}$ i.e. main variables $(b,y)$, unless otherwise stated.}
\begin{align}
    0=d\sigma = \frac{F_{yy}}{b}dy+\Big(\frac{bF_{by}-F_y}{b^2}\Big)db\quad \Rightarrow\quad \frac{dy}{db}_{\mid\sigma}= \frac{F_y-b F_{yb}}{bF_{yy}}=\frac{y}{b}\frac{p_y}{\rho_y}
\end{align}
The adiabatic sound speed is then by definition
\begin{align}
    c_b^2:=\frac{dp}{d\rho}_{\mid\sigma}&=\frac{p_b + p_y\, \tfrac{dy}{db}_{\mid\sigma}}{\rho_b + \rho_y\, \tfrac{dy}{db}_{\mid\sigma}} = \frac{b p_b\, y\rho_y+(yp_y)^2}{b\rho_b\,y\rho_y+y\rho_y\,yp_y}\,
    \\[5pt]
    &=\frac{-b^2F_{bb} F_{yy}+(F_y-bF_{yb})^2}{F_{yy}\,(yF_y-bF_b)} = \frac{b\,p_b \,y\rho_y + (y\,p_y)^2}{y \rho_y (\rho+p)}\,.
\end{align}
Using $F=-\mathfrak{a}$ and $s=F_y=-\partial_T\mathfrak{a}$, we rewrite the adiabatic sound speed as
\begin{align}
    c_b^2:=
    &=\frac{-b^2F_{bb} F_{yy}+(F_y-bF_{yb})^2}{F_{yy}\,(yF_y-bF_b)} = \frac{n^2\,\partial_n^2\mathfrak{a}\,\partial_Ts + (s-n\partial_ns)^2}{\partial_Ts\,\,(\rho+p)}\,,
\end{align}
which makes it manifest that $c_b^2\geq0$ provided no ghosts $\rho+P>0$, the second principle of thermodynamics $\partial_Ts\geq 0$ and thermodynamic stability $\partial_n^2\mathfrak{a}\geq 0$ hold.

We also note that $c_b^2\equiv c_s^2$ always, as proved in \eqref{eq:cs_equiv_cb_flat_spacetime} below for flat spacetime, which shows that absence of ghosts, thermodynamic stability and gradient instabilities are closely related.


%----------------------------------------------------------------
\subsection{Matching to quintessence scalar field}
%----------------------------------------------------------------

Consider the metric in Newtonian gauge
\begin{equation}
    g_{\mu\nu}=a^2\begin{pmatrix}
        -(1+2\Phi) & 0 \\
        0 & (1-2\Psi)\delta_{ij}
    \end{pmatrix}\, \quad \Rightarrow \quad \sqrt{-g}=a^4(1+\Phi-3\Psi)\,.
\end{equation}
The action for the scalar field $\varphi$ with potential $V(\varphi)$ is
\begin{equation}
    S_\varphi = \int d^4x \sqrt{-g} \left[-\frac{1}{2} g^{\mu\nu}\partial_\mu\varphi \partial_\nu\varphi - V(\varphi)\right]\,, \quad X:= -g^{\mu\nu}\partial_\mu\varphi \partial_\nu\varphi\,.
\end{equation}
The energy-momentum tensor is
{\small
\begin{align}
\begin{aligned}
    T_{\mu\nu}&=-\frac{2}{\sqrt{-g}}\frac{\partial \left(\sqrt{-g}\mathcal{L}\right)}{\partial g^{\mu\nu}}
    = -\frac{2}{\sqrt{-g}}\left(\sqrt{-g}\frac{\partial \mathcal{L}}{\partial g^{\mu\nu}}-\frac{1}{2}\sqrt{-g}g_{\mu\nu}\mathcal{L}\right)
    \\
     &= \partial_\mu\varphi \partial_\nu\varphi + g_{\mu\nu}\left[-\frac{1}{2}g^{\alpha\beta}\partial_\alpha\varphi \partial_\beta\varphi - V(\varphi)\right]\\
    &= \frac{\partial_\mu\varphi}{\sqrt{-g^{\alpha\beta}\partial_\alpha\varphi \partial_\beta\varphi}} \frac{\partial_\nu\varphi}{\sqrt{-g^{\alpha\beta}\partial_\alpha\varphi \partial_\beta\varphi}} \left(-g^{\alpha\beta}\partial_\alpha\varphi \partial_\beta\varphi\right) + g_{\mu\nu}\left[\frac{1}{2}\left(-g^{\alpha\beta}\partial_\alpha\varphi \partial_\beta\varphi\right) - V(\varphi)\right]\,.
\end{aligned}
\end{align}
}
We expand around a homogeneous background $\bar{\varphi}(\tau)$ as $\varphi(\tau,\vec{x})=\bar{\varphi}(\tau)+\delta\varphi(\tau,\vec{x})$.
Dropping mixed and second order in the metric perturbations we have 
\begin{align}\label{eq:expansion_X}
\begin{aligned}
    X&= -g^{\mu\nu}\partial_\mu\varphi \partial_\nu\varphi
    = \frac{1}{a^2}(1-2\Phi)(\bar{\varphi}'+\delta\varphi')^2 - \frac{1}{a^2}(1+2\Psi)(\nabla\delta\varphi)^2\\
    &\simeq \frac{1}{a^2}(\bar{\varphi}')^2+ \frac{2}{a^2}(\bar{\varphi}'\delta\varphi'-\Phi (\bar{\varphi}')^2) + \frac{1}{a^2}\left[(\delta\varphi')^2-(\nabla\delta\varphi)^2\right]\\
    &\simeq \frac{1}{a^2}(\bar{\varphi}')^2\left[1+ 2\left(\frac{\delta\varphi'}{\bar{\varphi}'}-\Phi\right) + \left(\frac{\delta\varphi'}{\bar{\varphi}'}\right)^2 - \frac{(\nabla\delta\varphi)^2}{(\bar{\varphi}')^2}\right]+\mathcal{O}(3,\mathrm{mix})\,.
\end{aligned}
\end{align}
Then, again dropping mixed and secnd order in the metric perturbations, we have
\begin{align}
   X^{-1/2}&\simeq \frac{a}{\bar{\varphi}'}\left[1 - \left(\frac{\delta\varphi'}{\bar{\varphi}'}-\Phi\right) +\left(\frac{\delta\varphi'}{\bar{\varphi}'}\right)^2 + \frac{1}{2}\frac{(\nabla\delta\varphi)^2}{(\bar{\varphi}')^2} \right] +\mathcal{O}(3,\mathrm{mix})\,.
\end{align}
%
Matching to a perfect fluid $T_{\mu\nu}=(\rho+p)u_\mu u_\nu + p g_{\mu\nu}$ we identify the velocity field as
\begin{align}\label{eq:velocity_field_expansion}
\begin{aligned}
    u_\mu &= \frac{\partial_\mu\varphi}{\sqrt{-g^{\alpha\beta}\partial_\alpha\varphi \partial_\beta\varphi}} =  X^{-1/2}\left(\bar{\varphi}^\prime+\delta\varphi',\, \nabla\delta\varphi\right)\\
    &\simeq \frac{a}{\bar{\varphi}'}\left[1 - \left(\frac{\delta\varphi'}{\bar{\varphi}'}-\Phi\right) +\left(\frac{\delta\varphi'}{\bar{\varphi}'}\right)^2 + \frac{1}{2}\frac{(\nabla\delta\varphi)^2}{(\bar{\varphi}')^2} \right]\left(\bar{\varphi}^\prime+\delta\varphi',\, \nabla\delta\varphi\right)\\
    &\simeq a\left(1+ \Phi,\, \frac{\nabla\delta\varphi}{\bar{\varphi}'}\right) + \mathcal{O}(2)\,.
\end{aligned}
\end{align}
Expanding $V(\varphi)=V(\bar{\varphi}) + V_{\varphi}(\bar{\varphi})\delta\varphi + \frac{1}{2}V_{\varphi\varphi}(\bar{\varphi})(\delta\varphi)^2$ we identify the \textbf{energy density}
\begin{align}
\begin{aligned}
    \rho &= \frac{1}{2}X + V(\varphi)\\
    &\simeq  \frac{1}{2\,a^2}(\bar{\varphi}')^2\left[1+ 2\left(\frac{\delta\varphi'}{\bar{\varphi}'}-\Phi\right) + \left(\frac{\delta\varphi'}{\bar{\varphi}'}\right)^2 - \frac{(\nabla\delta\varphi)^2}{(\bar{\varphi}')^2}\right]\\
    &\quad + V(\bar{\varphi}) + V_{\varphi}(\bar{\varphi})\delta\varphi + \frac{1}{2}V_{\varphi\varphi}(\bar{\varphi})(\delta\varphi)^2 +\mathcal{O}(3,\mathrm{mix})
\end{aligned}
\end{align}
and the \textbf{pressure} 
\begin{align}
\begin{aligned}
    p &= \frac{1}{2}X - V(\varphi) \\
    &\simeq  \frac{1}{2\,a^2}(\bar{\varphi}')^2\left[1+ 2\left(\frac{\delta\varphi'}{\bar{\varphi}'}-\Phi\right) + \left(\frac{\delta\varphi'}{\bar{\varphi}'}\right)^2 - \frac{(\nabla\delta\varphi)^2}{(\bar{\varphi}')^2}\right]\\
    &\quad- V(\bar{\varphi}) - V_{\varphi}(\bar{\varphi})\delta\varphi - \frac{1}{2}V_{\varphi\varphi}(\bar{\varphi})(\delta\varphi)^2 +\mathcal{O}(3,\mathrm{mix})\,.
\end{aligned}
\end{align}
Finally we have 
\begin{align}
    p=\rho-2V(\varphi) \quad \Rightarrow \quad \delta p = \delta\rho - 2 V_{\varphi}(\bar{\varphi})\delta\varphi\,.
\end{align}
Since we showed that $c_b^2:=\frac{\delta p}{\delta \rho}_{\mid \sigma}=1$ we identify the \textbf{non-adiabatic pressure perturbations} of the field
\begin{align}
    \delta p_{\text{nad}} =\frac{\partial p}{\partial \sigma}_{\mid \rho}\delta\sigma= \delta p - c_b^2 \delta \rho = - 2 V_{\varphi}(\bar{\varphi})\delta\varphi\,\quad \Rightarrow\quad \Gamma:= \frac{\partial p}{\partial \sigma}_{\mid \rho}= - \frac{\delta\varphi}{\delta \sigma_{\mid \rho}}2 V_{\varphi}(\bar{\varphi})\,
\end{align}
and we confim below that $\Gamma$ is a function of the potential only and that indeed $\delta\sigma\propto \delta\varphi$.

Finally note that in the comoving gauge $\varphi^I=x^I$ we have
\begin{align}
    b=\left(\det(g^{\mu\nu}\partial_\mu\varphi^I\partial_\nu\varphi^J)\right)^{1/2} = \sqrt{\det(g^{ij})} = \frac{1}{a^3}(1+3\Psi)\,.
\end{align}
and
\begin{align}\label{eq:expansion_y_quintessence}
    y=-\sqrt{X}&= -\frac{\bar{\varphi}'}{a}\left[1 + \left(\frac{\delta\varphi'}{\bar{\varphi}'}-\Phi\right) - \frac{1}{2}\frac{(\nabla\delta\varphi)^2}{(\bar{\varphi}')^2} \right] +\mathcal{O}(3,\mathrm{mix})\\
    &= -\frac{\bar{\varphi}'}{a} -\frac{\delta\varphi'}{a}+\frac{\bar{\varphi}'\Phi}{a} + \mathcal{O}(2)\,.
\end{align}
Therefore the \textbf{comoving entropy for the scalar field} is 
\begin{align}
    \sigma = \frac{F_y}{b} = a^3(1-3\Psi) y =  -a^2\bar{\varphi}'\left[1 + \left(\frac{\delta\varphi'}{\bar{\varphi}'}-\Phi-3\Psi\right) - \frac{1}{2}\frac{(\nabla\delta\varphi)^2}{(\bar{\varphi}')^2} \right] +\mathcal{O}(3,\mathrm{mix})\,.
\end{align}
and the \textbf{entropy density} (fluctuations) are directly proportional to field fluctuations 
\begin{align}
    s = F_y = y =  -\frac{\bar{\varphi}'}{a} -\frac{\delta\varphi'}{a}+\frac{\bar{\varphi}'\Phi}{a} + \mathcal{O}(2)\,.
\end{align}

Finally the \textbf{equation of state}, upon inverting $\sigma$ to get $\varphi$, reads
\begin{align}
    p=\rho-2V(\varphi) \sim \rho +\mathbf{f}(\sigma)\,,
\end{align}
confirming it is \emph{not} a function of the potential only.


%----------------------------------------------------------------
\subsection{Matching to k-essence scalar field}
%----------------------------------------------------------------
Consider the action for a k-essence scalar field $\varphi$ with lagrangian $P(X,\varphi)$
\begin{equation}
    S_\varphi = \int d^4x \sqrt{-g} P(X,\varphi)\,, \quad X:= -g^{\mu\nu}\partial_\mu\varphi \partial_\nu\varphi\,.
\end{equation}
The energy-momentum tensor is
\begin{align}
\begin{aligned}
    T_{\mu\nu}&=-\frac{2}{\sqrt{-g}}\frac{\partial \left(\sqrt{-g}\mathcal{L}\right)}{\partial g^{\mu\nu}}
    = -\frac{2}{\sqrt{-g}}\left(\sqrt{-g}\frac{\partial \mathcal{L}}{\partial g^{\mu\nu}}-\frac{1}{2}\sqrt{-g}g_{\mu\nu}\mathcal{L}\right)
    \\[4pt]
     &= 2P_X \partial_\mu\varphi \partial_\nu\varphi + g_{\mu\nu} P(X,\varphi)
     = u_\mu u_\nu 2X P_X + g_{\mu\nu} P(X,\varphi)\,.
\end{aligned}
\end{align}
The velocity field is $u_\mu = \partial_\mu\varphi/\sqrt{X}$ as in the quintessence case, and it enjoys the same expansion \eqref{eq:velocity_field_expansion}.
Matching to a perfect fluid we identify the \textbf{energy density} and \textbf{pressure} as
\begin{align}\label{eq:energy_density_pressure_k_essence}
\begin{aligned}
    \rho &= 2 X P_X - P(X,\varphi) = \bar{\rho} + \underbrace{(P_X+2XP_{XX})\delta X + (2XP_{X\varphi}-P_\varphi)\delta\varphi}_{=\delta\rho}\,,\\[4pt]
    p&= P(X,\varphi) = \bar{p} + \underbrace{P_X \delta X + P_\varphi \delta\varphi}_{=\delta p}
\end{aligned}
\end{align}
The expressions of $X$ is again given by \eqref{eq:expansion_X}, so that we find the \textbf{energy density and pressure fluctuations}
\begin{align}
\begin{aligned}
    \delta\rho &= \big(P_X+2XP_{XX}\big)\,\,\frac{2}{a^2}\left[\bar{\varphi}'\delta\varphi' - (\bar{\varphi}')^2\Phi\right] + (2XP_{X\varphi}-P_\varphi)\,\delta\varphi\,,\\[4pt]
    \delta p &= P_X\, \frac{2}{a^2}\left[\bar{\varphi}'\delta\varphi' - (\bar{\varphi}')^2\Phi\right] + P_\varphi \, \delta\varphi\,.
\end{aligned}
\end{align}
Similarly we still have $y=-\sqrt{X}$ with epansion \eqref{eq:expansion_y_quintessence}, so that the \textbf{entropy density} is
\begin{align}
    s= F_y = \frac{\partial P}{\partial y}= 2yP_X = -\frac{2P_X}{a} \left(\bar{\varphi}' +\delta\varphi' - \bar{\varphi}'\Phi \right) + \mathcal{O}(2)\,
\end{align}
and since also $b=\det(\partial_\mu\phi^I\partial_\nu\phi^J g^{\mu\nu})^{1/2}$ is unchanged, the \textbf{comoving entropy} is
\begin{align}
    \sigma = \frac{F_y}{b} = a^3(1-3\Psi) \frac{\partial P}{\partial y}= -2a^2P_X \left(\bar{\varphi}' +\delta\varphi' - \bar{\varphi}'(\Phi+3\Psi) \right) + \mathcal{O}(2)\,.
\end{align}
In particular we see that $\delta\sigma\propto \delta\varphi$ as in the quintessence case.

Finally we compute $c_b^2= \frac{\partial p}{\partial \rho}_{\mid \sigma}$ using the general formula in the previous sections, and in turn identify the non-adiabatic pressure perturbations $\delta p_{\text{nad}} = \frac{\partial p}{\partial \sigma}_{\mid \rho}\delta\sigma$.
We have $F(y,b)=P(X,\varphi)$ and $dX=-2y\,dy$ so that
\begin{align}
\begin{aligned}
    c_b^2&\equiv \frac{-b^2F_{bb} F_{yy}+(F_y-bF_{yb})^2}{F_{yy}\,(yF_y-bF_b)}= \frac{(P_X)^2}{(P_X)^2+2XP_XP_{XX}}
\end{aligned}
\end{align}
Comparing to \eqref{eq:energy_density_pressure_k_essence} above we find
\begin{align}
\begin{aligned}
    \delta p &= c_b^2 \delta\rho + \frac{\partial p}{\partial \sigma}_{\mid \rho}\delta\sigma
    = P_X \delta X + \frac{P_X\Big(2XP_{X\varphi}-P_\varphi)}{P_X+2XP_{XX}}\delta\varphi  + \frac{\partial p}{\partial \sigma}_{\mid \rho}\delta\sigma \overset{!}{=} P_X\delta_X + P_\varphi\delta\varphi\,,
\end{aligned}
\end{align}
we identify the \textbf{non-adiabatic pressure perturbations} of the k-essence field
\begin{align}
\begin{aligned}
    \delta p_{\text{nad}} &=\frac{\partial p}{\partial \sigma}_{\mid \rho}\delta\sigma
    = \left(P_\varphi - \frac{P_X\Big(2XP_{X\varphi}-P_\varphi)}{P_X+2XP_{XX}}\right)\delta\varphi\\
    & \Rightarrow\quad \Gamma:= \frac{\partial p}{\partial \sigma}_{\mid \rho}= \left(P_\varphi - \frac{P_X\Big(2XP_{X\varphi}-P_\varphi)}{P_X+2XP_{XX}}\right)\frac{\delta\varphi}{\delta \sigma_{\mid \rho}}\,.
\end{aligned}
\end{align}


%----------------------------------------------------------------
\subsection{Matching to k-essence remastered}
%----------------------------------------------------------------
{\color{red} 
The matching should be adjusted so as to preserve shift symmetry of $\phi^0$, thus forbidding the dentification of $\varphi$ with $\phi^0$ directly since $V(\varphi)$ would otherwise break it.
In turn, the identification $y=\sqrt{X}$ need some rethinking.

In the 5-essence case we can express the potential in terms of the fluid variables as
\begin{align}
    \rho=\tfrac12 X +V, \quad p =\tfrac12 X -V\quad \Rightarrow\quad V(\varphi) = \tfrac12(\rho - p) \equiv -F +\tfrac12 (yF_y+ bF_b)\,.
\end{align}
Thermodyncamically this identifies the potential with the legendre transform of the free energy density $\mathfrak{a}$ with respect to both $T$ and $n$.
We also recover $X$ as 
\begin{align}
    X=\rho+p = yF_y + bF_b\,.
\end{align}
We should now invert this to identify $y$ and $b$...
}

%==========================================================================
\section{Quadratic action in curved spacetime}
%===========================================================================
We analyze the fluctuations in a general curved background metric, written wrt conformal time $\tau$ as
{\small 
\begin{align}
\begin{aligned}
    &g_{\mu\nu}=a^2\Big(\eta_{\mu\nu}+h_{\mu\nu}\Big)\,,\quad g^{\mu\nu}=a^{-2}\Big(\eta^{\mu\nu}-h^{\mu\nu}\Big)\quad \text{with} \quad h^{\mu\nu}=\eta^{\mu\alpha}\eta^{\nu\beta}h_{\alpha\beta},\\
    &\quad \sqrt{|g|}= a^4\big(1+\tfrac12 h^{\mu\nu}\eta_{\mu\nu} +O(h^2)\big)\ = a^4\Big[1+\tfrac12 (h^{ii}-h^{00}) +O(h^2)\Big]\,.
\end{aligned}
\end{align}
}
The fluids variables are written for simplicity with a rescaling factor $\gamma$ and $\lambda$ as
\begin{align}
    \phi^0=\gamma\Big(\tau+\pi^0\Big), \quad \phi^i=\lambda\Big(x^i+\pi^i\Big)\,.
\end{align}
The invariants are again the number density $n\equiv b$ defined by
\begin{align}
    b=\sqrt{\det B}\quad \text{for}\quad
    B^{ij}= g^{\mu\nu} \partial_\mu \phi^i \partial_\nu \phi^j,
\end{align}
and the temperature $T\equiv y$ defined by
\begin{align}
    y= u^\mu \partial_\mu \phi^0
    \quad\text{for}\quad u^\mu=& \frac{1}{\sqrt{-g}\, 3!\,b} \epsilon^{\mu \alpha \beta \gamma} \epsilon_{ijk} \, \partial_\alpha \phi^i \partial_\beta \phi^j \partial_\gamma \phi^k \propto\,  \star(d\phi^1\wedge d\phi^2\wedge d \phi^3).
\end{align}

We now expand up to quadratic order in the fields $\pi$ and mixed order in $\pi h$, ignoring pure second orders in the metric $h^2$.
The latter would indeed give a contribution $\propto \, \sqrt{|g|}\, F(\bar{b}, \bar{y}) h^2$ that could be reabsorbed in a redefinition of the cosmological constant.
We have the expansions
{
\begin{align}\label{eq:expansions_b_y}
\begin{aligned}
    &b= \bar{b}(1+\delta b),\quad \bar{b}= \frac{\lambda^3}{a^3},\quad
    \delta b^{(1)} = -\tfrac12 h^{ii} +(\partial_j \pi^j),\\[3pt]
    &\delta b^{(2)} = - h^{0 i} \dot{\pi}^i - \tfrac12  h^{ii} (\partial_j \pi^j) - \tfrac12 (\dot{\pi}^i)^2 + \tfrac12 (\partial_j \pi^j)^2 -\tfrac12 \partial_i\pi^j\partial_j\pi^i\,,
    \\[3pt]
    &y= \bar{y}(1+\delta y),\quad \bar{y}= \frac{\gamma}{a},\quad \delta y^{(1)} = \tfrac12 h^{00}+ \dot{\pi}^0,\\[3pt]
    &\delta y^{(2)} = \tfrac12 h^{00}\dot{\pi}^0+ h^{0 i} \dot{\pi}^i  + \tfrac12 (\dot{\pi}^i)^2 -\partial_j\pi^0\dot{\pi}^j\,\\[3pt]
    &au^\mu=\delta_0^\mu \bigg(1 \!+\! \Big[\tfrac12 h^{00} -(\partial_j \pi^j)\Big]\! +\! \Big[ h^{0 i} \dot{\pi}^i -\tfrac12 h^{00} (\partial_j \pi^j) + \tfrac12 (\dot{\pi}^i)^2 + \tfrac12 (\partial_j \pi^j)^2 +\tfrac12 \partial_i\pi^j\partial_j\pi^i\Big]\bigg)
    \\
    &\quad\qquad+\tfrac{1}{2}\left[1 + \tfrac12 h^{00}-(\partial_\ell\pi^\ell)\right]
     \epsilon_{i j k}\,\epsilon^{\mu \alpha j k}\,\partial_\alpha\pi^i 
     +\frac{1}{2} \epsilon_{ijk}\, \epsilon^{\mu i \beta\gamma}\partial_\beta\pi^j\,\partial_\gamma\pi^k
    + O(\pi^3, h^2)
\end{aligned}
\end{align}
}
In turn we can express the fluctuations in density $\rho=-F+yF_y$ and pressure $p=F - bF_b$ as
\begin{align}\label{eq:rho_p_fluctuations_via_b_y}
\begin{aligned}
    \delta\rho &= y^2F_{yy}\delta y + (-bF_b+byF_{by})\delta b
    \\
    &= (y^2F_{yy})\left(\tfrac12 h^{00}+ \dot{\pi}^0\right) + (-bF_b+byF_{by})\left(-\tfrac12 h^{ii} +(\partial_j \pi^j)\right) + O(2)
    \\[3pt]
    \delta p &= (yF_y-byF_{by})\delta y - (b^2 F_{bb})\delta b
    \\
    &= (yF_y-byF_{by})\left(\tfrac12 h^{00}+ \dot{\pi}^0\right) - (b^2 F_{bb})\left(-\tfrac12 h^{ii} +(\partial_j \pi^j)\right)+ O(2)\,.
\end{aligned}
\end{align}

We also introduce the phonon potential $\partial_\ell\pi_L=\pi^\ell$, so that $\partial_j \pi^j = \nabla^2 \pi_L$ and $[\pi_L]=E^{-2}$.
From the velocity expansion $au^\mu =a(1+\dots, v^\ell)$ above we identify the peculiar velocity and the velocity potential (for phonons/scalar perturbations) as
\begin{align}
    av^\ell = \frac{1}{2}\epsilon_{i j k}\,\epsilon^{\ell \alpha j k}\,\partial_\alpha\pi^i = -\dot{\pi}^\ell\quad \Rightarrow \quad av = -\dot{\pi}_L\quad \text{where}\quad \partial_\ell\pi_L=\pi^\ell
\end{align}

We can now expand the action up to second order 
{\small
\begin{align}
\begin{aligned}
    S&=\int d^4x \sqrt{-g} F(b,y) = \int d^4x\,a^4 \big[1+\tfrac12 (h^{ii}-h^{00}) \big] F(\bar{b}(1+\delta b), \bar{y}(1+\delta y))
    \\
    &= \int d^4x\, a^4 \big[1+\tfrac12 (h^{ii}-h^{00}) \big]
    \Bigg[ F(\bar{b}, \bar{y}) + (\bar{b}F_b)  (\delta b^{(1)} + \delta b^{(2)}) + (\bar{y}F_y) (\delta y^{(1)} + \delta y^{(2)})\\
    &\qquad\qquad\qquad\qquad\qquad\qquad\qquad+ \tfrac12 (\bar{b}^2 F_{bb})(\delta b^{(1)})^2 + \tfrac12 (\bar{y}^2 F_{yy}) (\delta y^{(1)})^2
    + (\bar{b}\bar{y} F_{by}) \delta b^{(1)} \delta y^{(1)} \Bigg]
\end{aligned}
\end{align}
}
The quadratic part of the action for fluid perturbations in a general curved background is therefore
{\small
\begin{align}
\begin{aligned}
    S^{(2)}&=\int d^4x\,\, a^4 \bigg\{
        (\bar{b}F_b) (h^{ii}-h^{00}) \Big[(\partial_j \pi^j) - \tfrac12 h^{ii}\Big]
        + (\bar{y}F_y) (h^{ii}-h^{00}) \Big[\tfrac12 h^{00}+ \dot{\pi}^0\Big]\\
    &\quad + (\bar{b}F_b) \Big[ - h^{0 i} \dot{\pi}^i - \tfrac12  h^{ii} (\partial_j \pi^j) - \tfrac12 (\dot{\pi}^i)^2 + \tfrac12 (\partial_j \pi^j)^2 -\tfrac12 \partial_i\pi^j\partial_j\pi^i\Big]
    \\
    &\quad + (\bar{y}F_y) \Big[ \tfrac12 h^{00}\dot{\pi}^0+ h^{0 i} \dot{\pi}^i + \tfrac12 (\dot{\pi}^i)^2 -\partial_j\pi^0\dot{\pi}^j\Big]
    \\
    &\quad + \tfrac12 (\bar{b}^2 F_{bb})\Big[ (\partial_j \pi^j) - \tfrac12 h^{ii}\Big]^2 
    + \tfrac12 (\bar{y}^2 F_{yy})\Big[\tfrac12 h^{00}+ \dot{\pi}^0\Big]^2
    + (\bar{b}\bar{y} F_{by}) \Big[(\partial_j \pi^j) - \tfrac12 h^{ii}\Big]\Big[\tfrac12 h^{00}+ \dot{\pi}^0\Big]
        \bigg\}.
\end{aligned}
\end{align}
}
Dropping  further second orders in metric perturbations and {\color{blue} integrating by parts in space}, we get
{\small
\begin{align}
\begin{aligned}
    S^{(2)}=
    \int d^4x\,\, a^4 \bigg\{&
        (\bar{b}F_b) (h^{ii}-h^{00}) (\partial_j \pi^j)
        + (\bar{y}F_y) (h^{ii}-h^{00}) \dot{\pi}^0 \\
    &\quad + (\bar{b}F_b) \Big[ - h^{0 i} \dot{\pi}^i - \tfrac12  h^{ii} (\partial_j \pi^j) - \tfrac12 (\dot{\pi}^i)^2 + \bluecancel{\tfrac12 (\partial_j \pi^j)^2} -\bluecancel{\tfrac12 \partial_i\pi^j\partial_j\pi^i}\Big]
    \\
    &\quad + (\bar{y}F_y) \Big[ \tfrac12 h^{00}\dot{\pi}^0+ h^{0 i} \dot{\pi}^i + \tfrac12 (\dot{\pi}^i)^2 -\partial_j\pi^0\dot{\pi}^j\Big]
    \\
    &\quad + \tfrac12 (\bar{b}^2 F_{bb})\Big[ (\partial_j \pi^j)^2 - h^{ii}(\partial_j \pi^j)\Big]
    + \tfrac12 (\bar{y}^2 F_{yy})\Big[h^{00} \dot{\pi}^0+ (\dot{\pi}^0)^2\Big]\\
    &\quad+ (\bar{b}\bar{y} F_{by}) \Big[ \dot{\pi}^0(\partial_j \pi^j) - \tfrac12 h^{ii}\dot{\pi}^0 + \tfrac12 h^{00} (\partial_j\pi^j)\Big]
        \bigg\}
\end{aligned}
\end{align}
}
Collecting terms, the final expression for the quadratic action is
{\small
\begin{align}\label{eq:final_quadratic_action_curved_spacetime}
\begin{aligned}
    S^{(2)}=
    &\int d^4x\,\,a^4\,\,\bigg\{
    \tfrac12 (yF_y-bF_b) (\dot{\pi}^i)^2+\tfrac12 b^2 F_{bb} (\partial_j \pi^j)^2
    +\tfrac12 y^2 F_{yy} (\dot{\pi}^0)^2 \\
    &\qquad\qquad\qquad+ (b y F_{by}) (\dot{\pi}^0 \partial_j \pi^j)-(yF_y) (\partial_j \pi^0\dot{\pi}^j)
    \bigg\}\\
    &\quad + a^4\,\,\bigg\{
    \tfrac12\!\left(\bar{b}F_b - \bar{b}^2 F_{bb}\right) h^{ii}(\partial_j \pi^j)
    + \tfrac12\!\left(-2\bar{b}F_b + \bar{b}\bar{y}F_{by}\right) h^{00}(\partial_j \pi^j)\\
    &\qquad\qquad+ \left(\bar{y}F_y - \bar{b}F_b\right) h^{0j}\,\dot{\pi}^j + \tfrac12\!\left(2\bar{y}F_y - \bar{b}\bar{y}F_{by}\right) h^{ii}\dot{\pi}^0
    + \tfrac12\!\left(\bar{y}^2 F_{yy} - \bar{y}F_y\right) h^{00}\dot{\pi}^0
    \bigg\}\,.
\end{aligned}
\end{align}}
Note that only \emph{longitudinal} modes $\pi^j_L\propto\, \partial_j\pi^j$ results dynamical.
Now gather $\rho+p=yF_y-bF_b$ outside the brackets and rewrite the action as
{\small
\begin{align}
\begin{aligned}
    S^{(2)}= &\int d^4x\,\,a^4\, (\rho+p)\,\bigg\{
    \tfrac12 (\dot{\pi}^i)^2-\tfrac12 \underbrace{\tfrac{-b^2 F_{bb}}{(yF_y-bF_b)}}_{=\mathcal{C}} (\partial_j \pi^j)^2
    +\tfrac12 \underbrace{\tfrac{y^2 F_{yy}}{(yF_y-bF_b)}}_{=\mathcal{A}} (\dot{\pi}^0)^2 \\
    &\quad\qquad\qquad\qquad\qquad+ \tfrac{(b y F_{by})}{(yF_y-bF_b)} (\dot{\pi}^0 \partial_j \pi^j)-\tfrac{(yF_y)}{(yF_y-bF_b)} (\partial_j \pi^0\dot{\pi}^j)
    \bigg\}\\
    &\quad + a^4\, (\rho+p)\, \bigg\{
    \tfrac12\!\left(\tfrac{\bar{b}F_b - \bar{b}^2 F_{bb}}{(yF_y-bF_b)}\right) h^{ii}(\partial_j \pi^j)
    + \tfrac12\!\left(\tfrac{-2\bar{b}F_b + \bar{b}\bar{y}F_{by}}{(yF_y-bF_b)}\right) h^{00}(\partial_j \pi^j)\\
    &\qquad\qquad\qquad\qquad+ \left(\tfrac{\bar{y}F_y - \bar{b}F_b}{(yF_y-bF_b)}\right) h^{0 j}\,\dot{\pi}^j + \tfrac12\!\left(\tfrac{2\bar{y}F_y - \bar{b}\bar{y}F_{by}}{(yF_y-bF_b)}\right) h^{ii}\dot{\pi}^0
    + \tfrac12\!\left(\tfrac{\bar{y}^2 F_{yy} - \bar{y}F_y}{(yF_y-bF_b)}\right) h^{00}\dot{\pi}^0
    \bigg\}.
\end{aligned}
\end{align}}
%
Integrating by parts in both time and space the term $\partial_j \pi^0\dot{\pi}^j$ we get
{\small
\begin{align}
\begin{aligned}
    S^{(2)}=&\int d^4x\, -(\partial_j \pi^j\, \pi^{0})\tfrac{d}{d\tau}(a^4yF_y)\\
    &\quad+a^4\,\bigg\{
    \tfrac12 (yF_y-bF_b) (\dot{\pi}^i)^2+ \tfrac12 b^2 F_{bb} (\partial_j \pi^j)^2 +\tfrac12 y^2 F_{yy} (\dot{\pi}^0)^2+(b y F_{by}-yF_y) (\dot{\pi}^0 \partial_j \pi^j)\bigg\}\\
    &\quad + a^4\,\,\bigg\{
    \tfrac12\!\left(\bar{b}F_b - \bar{b}^2 F_{bb}\right) h^{ii}(\partial_j \pi^j)
    + \tfrac12\!\left(-2\bar{b}F_b + \bar{b}\bar{y}F_{by}\right) h^{00}(\partial_j \pi^j)+\left(\bar{y}F_y - \bar{b}F_b\right) h^{0 j}\,\dot{\pi}^j\\
    &\qquad\qquad +\tfrac12\!\left(2\bar{y}F_y - \bar{b}\bar{y}F_{by}\right) h^{ii}\dot{\pi}^0
    + \tfrac12\!\left(\bar{y}^2 F_{yy} - \bar{y}F_y\right) h^{00}\dot{\pi}^0
    \bigg\}
\end{aligned}
\end{align}}
%
Resolving the constraint for $\pi^0$ we now get
{\small
\begin{align}
\begin{aligned}
    \frac{d}{d\tau}a^4\bigg[y^2F_{yy}\dot{\pi}^0+(byF_{by}-yF_y) (\partial_j\pi^j)+\tfrac12\!\big(2{y}F_y - byF_{by}\big) h^{ii}
    + \tfrac12\!\left({y}^2 F_{yy} - {y}F_y\right) h^{00}\bigg]=-(\partial_j \pi^j)\tfrac{d}{d\tau}(a^4yF_y)\,.
\end{aligned}
\end{align}}
Substituting back in the action we get
{\small
\begin{align}
\begin{aligned}
    S^{(2)}=
    &\int d^4x\,\, a^4\,\bigg\{
    \tfrac12 (yF_y-bF_b) (\dot{\pi}^i)^2+ \tfrac12 b^2 F_{bb} (\partial_j \pi^j)^2\bigg\} -\tfrac12 a^4 y^2F_{yy} (\dot{\pi}^0)^2\\
    &\quad + a^4\,\,\bigg\{\tfrac12\!\left(\bar{b}F_b - \bar{b}^2 F_{bb}\right) h^{ii}(\partial_j \pi^j)
    + \tfrac12\!\left(-2\bar{b}F_b + \bar{b}\bar{y}F_{by}\right) h^{00}(\partial_j \pi^j)+\left(\bar{y}F_y - \bar{b}F_b\right) h^{0j}\,\dot{\pi}^j \bigg\}\,,
\end{aligned}
\end{align}}
where now
{\small
\begin{align}
\begin{aligned}
    a^4y^2F_{yy}\dot{\pi}^0
    &=-a^4\bigg[(byF_{by}-yF_y) (\partial_j\pi^j)+\tfrac12\!\big(2{y}F_y - byF_{by}\big) h^{ii}
    + \tfrac12\!\left({y}^2 F_{yy} - {y}F_y\right) h^{00}\bigg]-\int d\tau\, (\partial_j \pi^j)\tfrac{d}{d\tau}(a^4yF_y)\\
    &{\color{red}=-a^4\bigg[(byF_{by}-yF_y) (\partial_j\pi^j)+\bigg(\tfrac{1}{a^4}\!\!\int(\partial_j \pi^j)\tfrac{d}{d\tau}(a^4yF_y)\bigg)\bigg]-a^4\tfrac12\Big[\big(2{y}F_y - byF_{by}\big) h^{ii}+ \left({y}^2 F_{yy} - {y}F_y\right) h^{00}\Big]}
\end{aligned}
\end{align}}
That is\todotag{Rewrite exp using 2nd form  red above}
{\small
\begin{align}
\begin{aligned}
    -\tfrac12 a^4y^2F_{yy}\big(\dot{\pi}^0\big)^2
    &=-\tfrac12 a^4 \tfrac{(byF_{by}-yF_y)^2}{y^2F_{yy}} (\partial_j\pi^j)^2\\
    &\quad-\tfrac{(byF_{by}-yF_y)}{y^2F_{yy}} (\partial_j\pi^j)\bigg(\int(\partial_\ell\pi^\ell)\tfrac{d}{d\tau}(a^4yF_y) \bigg)
    -\tfrac12 \tfrac{a^{-4}}{y^2F_{yy}}\bigg(\int(\partial_\ell\pi^\ell)\tfrac{d}{d\tau}(a^4yF_y)\bigg)^2
    \\
    &\quad-\tfrac12 a^4\tfrac{\big(2{y}F_y-byF_{by}\big)(byF_{by}-yF_y)}{y^2F_{yy}}h^{ii}(\partial_j\pi^j)
    -\tfrac12 a^4\tfrac{\left({y}^2 F_{yy} - {y}F_y\right)(byF_{by}-yF_y)}{y^2F_{yy}}h^{00}(\partial_j\pi^j)
    \\
    &\quad -\tfrac12 \tfrac{\big(2{y}F_y-byF_{by}\big)}{y^2F_{yy}}h^{ii}\bigg(\int(\partial_\ell\pi^\ell)\tfrac{d}{d\tau}(a^4yF_y)\bigg)
    -\tfrac12 \tfrac{\left({y}^2 F_{yy} - {y}F_y\right)}{y^2F_{yy}}h^{00}\bigg(\int(\partial_\ell\pi^\ell)\tfrac{d}{d\tau}(a^4yF_y)\bigg).
\end{aligned}
\end{align}}
Substituting back in the action we finally get\todotag{Rewrite using 2nd form red above, keep $\nabla\pi+\int$ together}
{\small
\begin{align}
\begin{aligned}
    S^{(2)}=&\int d^4x\,\, a^4\,\frac{\rho+p}{2}\Bigg\{
    (\dot{\pi}^i)^2-\bigg[-\tfrac{b^2 F_{bb}}{(yF_y-bF_b)} +\tfrac{(byF_{by}-yF_y)^2}{y^2F_{yy}(yF_y-bF_b)}\bigg] (\partial_j \pi^j)^2 \\
    &\quad-2\tfrac{(byF_{by}-yF_y)}{y^2F_{yy}(yF_y-bF_b)} (\partial_j\pi^j)\bigg(\tfrac{1}{a^4}\int(\partial_\ell\pi^\ell)\tfrac{d}{d\tau}(a^4yF_y) \bigg)
    -\tfrac{1}{y^2F_{yy}(yF_y-bF_b)}\bigg(\tfrac{1}{a^4}\int(\partial_\ell\pi^\ell)\tfrac{d}{d\tau}(a^4yF_y)\bigg)^2\\
    &\quad +\left(\tfrac{\bar{b}F_b-\bar{b}^2 F_{bb}}{yF_y-bF_b} - \tfrac{\big(2{y}F_y-byF_{by}\big)(byF_{by}-yF_y)}{(yF_y-bF_b)y^2F_{yy}}\right) h^{ii}(\partial_j \pi^j)\\
    &\qquad\qquad\qquad+\!\left(\tfrac{\bar{b}\bar{y}F_{by}-2\bar{b}F_b}{yF_y-bF_b} - \tfrac{\left({y}^2 F_{yy} - {y}F_y\right)(byF_{by}-yF_y)}{(yF_y-bF_b)y^2F_{yy}}\right) h^{00}(\partial_j \pi^j)
    +2h^{0 j}\,\dot{\pi}^j\\
    &-\tfrac{\big(2{y}F_y-byF_{by}\big)}{y^2F_{yy}(yF_y-bF_b)}h^{ii}\bigg(\tfrac{1}{a^4}\int(\partial_\ell\pi^\ell)\tfrac{d}{d\tau}(a^4yF_y)\bigg)
    -\tfrac{\left({y}^2 F_{yy} - {y}F_y\right)}{y^2F_{yy}(yF_y-bF_b)}h^{00}\bigg(\tfrac{1}{a^4}\int(\partial_\ell\pi^\ell)\tfrac{d}{d\tau}(a^4yF_y)\bigg)\Bigg\}.
\end{aligned}
\end{align}}


%-----------------------------------------------------------------
\subsection{Quadratic action in flat spacetime}
%-------------------------------------------------------
In flat spacetime $a=1$ and $h_{\mu\nu}=0$, only the first terms in the quadratic action \eqref{eq:final_quadratic_action_curved_spacetime} survive, and we can integrate by parts in both space and time the term $\partial_j \pi^0\dot{\pi}^j = -\dot{\pi}^0(\partial_j \pi^j)$ to get
{\small
\begin{align}
\begin{aligned}
    S^{(2)}&= \int d^4x \frac{(yF_y-bF_b)}{2}\bigg\{(\dot{\pi}^i)^2 - (\partial_j\pi^j)^2\Big[-\frac{b^2F_{bb}}{(yF_y-bF_b)}\Big]\\
    &\qquad\qquad\qquad\qquad\qquad+(\dot{\pi}^0)^2 \frac{y^2F_{yy}}{(yF_y-bF_b)}
    - 2\dot{\pi}^0 (\partial_j \pi^j) \frac{(yF_y-byF_{by})}{(yF_y-bF_b)}
    \bigg\}\\
    &=\int d^4x \frac{(\rho+p)}{2}\bigg\{(\dot{\pi}^i)^2 - \mathcal{C} (\partial_j\pi^j)^2 
    + (\dot{\pi}^0)^2 \mathcal{A}
    - 2\mathcal{M}\dot{\pi}^0 (\partial_j \pi^j) 
    \bigg\}
\end{aligned}
\end{align}
}
for $\rho= -F + y F_y$, $p= F - b F_b$ and 
{\small
\begin{align}
\begin{aligned}
    &\mathcal{C} = -\frac{b^2F_{bb}}{(yF_y-bF_b)},\quad 
    \mathcal{A} = \frac{y^2F_{yy}}{(yF_y-bF_b)} = \frac{T\, \frac{\partial\rho }{\partial T}}{\rho+p},\quad
    \mathcal{M} = \frac{(yF_y-byF_{by})}{(yF_y-bF_b)}.
\end{aligned}
\end{align}
}
The entropy field $\pi^0$ is thus non-dynamical and the constraint equations gives
\begin{align}
    \dot{\pi}^0 = \frac{\mathcal{M}}{\mathcal{A}} (\partial_j \pi^j) + \text{const.}
\end{align}
The constant is not physical and is reabsorbed in the original definition of $\phi^0=\gamma(t+\pi^0)$.
Solving the constraint the action for the coordinate $\pi^i$ becomes
{\small
\begin{align}
    S^{(2)}&= \int d^4x \frac{(\rho+p)}{2}\Big[(\dot{\pi}^i)^2 - c_s^2 (\partial_j\pi^j)^2 \Big], \quad \text{for}\quad c_s^2 = \mathcal{C} + \frac{\mathcal{M}^2}{\mathcal{A}}.
\end{align}
}
We note the entropy contribution to the speed of sound is always positive in \emph{physical} situations.
Indeed $\mathcal{M}^2$ is a true square and, recalling $y\equiv T$, we have
\begin{equation}
    \mathcal{A}= \frac{T\, \frac{\partial\rho }{\partial T}}{\rho+p}\geq 0 \quad\text{provided}\quad
    \begin{cases}
        \rho+p\geq 0 \quad(\text{no ghosts}),\\
        \frac{\partial\rho }{\partial T} \geq 0 \quad(\text{positive heat capacity}).
    \end{cases} 
\end{equation}
Finally we immediately check the speed of sound $c_s^2$, defined as the coefficient of the fluctuations above, always matches the thermodynamical definition of \emph{adiabatic sound speed}
\begin{align}\label{eq:cs_equiv_cb_flat_spacetime}
    c_b^2:= \frac{\partial p}{\partial \rho}_{\mid \sigma} \overset{\text{proved before}}{=}\frac{-b^2F_{bb} F_{yy}+(F_y-bF_{yb})^2}{F_{yy}\,(yF_y-bF_b)} \overset{\text{easy check}}{\equiv} c_s^2.
\end{align}

%============================================================================
\section{Matching to comoving curvature perturbations}
%============================================================================


With our convention for the metric perturbations in Newtonian gauge\todotag{TBC}
\begin{align}\label{eq:newtonian_gauge_metric}
\begin{aligned}
    &g_{\mu\nu}=a^2\begin{pmatrix}
        -(1+2\Phi) & 0 \\
        0 & (1-2\Psi)\delta_{ij}
    \end{pmatrix}\equiv a^2(\eta_{\mu\nu}+h_{\mu\nu})\\[5pt]
    &\quad \Rightarrow \quad h^{00}=h_{00}=-2\Phi,\quad h^{ii}=h_{ii}=-6\Psi\,.
\end{aligned}
\end{align}
%
We can thus plug this into \eqref{eq:rho_p_fluctuations_via_b_y} to explicitly express $\delta\rho$ and $\delta p$, we get
\begin{align}
\begin{aligned}
    \delta\rho &= (y^2F_{yy})\left(\tfrac12 h^{00}+ \dot{\pi}^0\right) + (-bF_b+byF_{by})\left(-\tfrac12 h^{ii} +(\partial_j \pi^j)\right) + O(2)\\
    &= (y^2F_{yy})\left(-\Phi+ \dot{\pi}^0\right) + (-bF_b+byF_{by})\left(3\Psi +(\partial_j \pi^j)\right) + O(2)
    \\[3pt]
    \delta p &= (yF_y-byF_{by})\left(\tfrac12 h^{00}+ \dot{\pi}^0\right) - (b^2 F_{bb})\left(-\tfrac12 h^{ii} +(\partial_j \pi^j)\right)+ O(2)\\
    &= (yF_y-byF_{by})\left(-\Phi+ \dot{\pi}^0\right) - (b^2 F_{bb})\left(3\Psi +(\partial_j \pi^j)\right)+ O(2)\,.
\end{aligned}
\end{align}
Introducing the phonon potential $\partial_\ell\pi_L=\pi^\ell$ we have $\partial_j \pi^j = \nabla^2 \pi_L$ and $[\pi_L]=E^{-2}$.
We also recall the 4-velocity expansion
\begin{align}
\begin{aligned}
    &au^\mu=\delta_0^\mu \bigg(1 \!+\! \Big[\tfrac12 h^{00} -(\partial_j \pi^j)\Big]\!\bigg) +\tfrac{1}{2}\epsilon_{i j k}\,\epsilon^{\mu \alpha j k}\,\partial_\alpha\pi^i +O(2)\,.
\end{aligned}
\end{align}
On the other hand, the velocity of the fluid is $au^\mu =a(1+\dots, v^\ell)$ where $v^\ell$ is the peculiar velocity.
We also denote the velocity divergence $\theta=\nabla\cdot\vec{v}$ and the velocity scalar potential $v$ by demanding\footnote{Indices raised/lowered with $\eta^{\mu\nu}=(-+++)$} $v^\ell=\partial_\ell v$ i.e. $\nabla^2 v=\theta$.
Matching the two expressions we find
\begin{align}\label{eq:peculiar_velocity_via_phonons}
    v^\ell = \frac{1}{2}\epsilon_{i j k}\,\epsilon^{\ell \alpha j k}\,\partial_\alpha\pi^i = -\dot{\pi}^\ell\quad \Rightarrow \quad v = -\dot{\pi}_L\quad \text{where we define $\pi_L$ by}\quad \partial_\ell\pi_L=\pi^\ell
\end{align}
Dimensions are respected since $[\pi^\alpha]=E^{-1}$ and $[v^\ell]=E^0$ so that $[v]=E^{-1}$ and $[\pi_L]=E^{-2}$. 

The gauge-invariant comoving curvature perturbation $\mathcal{R}$, i.e. the 3-curvature of comoving spatial slices, is defined in terms of metric and matter perturbations
\begin{align}
    \mathcal{R}\equiv C-\tfrac13\nabla^2E+\mathcal{H}(B+v) \overset{\text{newt. gauge}}{=} -\Psi +\mathcal{H}v = -\Psi.
\end{align}
since in Newtonian gauge $C=-\Psi$ and $B=E=0$.
From \label{eq:peculiar_velocity_via_phonons} we can identify the comoving curvature perturbations in terms of metric potentials and phonon potential
\begin{equation}
    \mathcal{R} = -\Psi +\mathcal{H}v = -\Psi - \mathcal{H}\,\dot{\pi}_L\,.
\end{equation}


In other works, another gauge invariant quantity $\zeta$ is used in place of the comoving curvature $\mathcal{R}$.
In Newtonian gauge it is defined by\footnote{The second equality holds for separately conserved em tensors}. 
\begin{align}
    \zeta\overset{\text{newt gauge}}{=} -\Psi-\mathcal{H}\frac{\delta \rho}{\dot{\rho}} = -\Psi +\tfrac13 \frac{\bar{\rho}\delta }{\bar{\rho}+\bar{p}}.
\end{align}
A linear combination of the 00 and 0i Einstein equation gives the gauge-invariant Poisson equation
\begin{align}
    \nabla^2\Psi= 4\pi G a^2 \Big(\underbrace{\bar{\rho}\delta-3\mathcal{H}(\bar{\rho}+\bar{p})v}_{=\bar{\rho}\Delta}\Big)
\end{align}
Therefore
\begin{align}
    \zeta-\mathcal{R}=-\mathcal{H}\frac{\delta \rho}{\dot{\rho}}-\mathcal{H}v=\bar{\rho}\delta-3\mathcal{H}(\bar{\rho}+\bar{p})v = \frac{-k^2}{4\pi G a^2}\Psi \,\,\to 0\quad\text{as}\,\,k\to0
\end{align}
confirming the two quantities indeed coincide on large scales.


In Newtonian gauge \eqref{eq:newtonian_gauge_metric}, the quadratic lagrangian becomes 
{\small
\begin{align}
\begin{aligned}
    S^{(2)}=
    &\int d^4x\,\, a^4\,\bigg\{
    \tfrac12 (yF_y-bF_b) (\dot{\pi}^i)^2+ \tfrac12 b^2 F_{bb} (\partial_j \pi^j)^2\bigg\} -\tfrac12 a^4 y^2F_{yy} (\dot{\pi}^0)^2\\
    &\quad + a^4\,\,\bigg\{-3\Psi(\partial_j \pi^j)\left(\bar{b}F_b - \bar{b}^2 F_{bb}\right) 
    -\Phi(\partial_j \pi^j)\!\left(-2\bar{b}F_b + \bar{b}\bar{y}F_{by}\right) \bigg\}\,,
\end{aligned}
\end{align}}
where
{\small
\begin{align}
\begin{aligned}
    -\tfrac12 a^4y^2F_{yy}\big(\dot{\pi}^0\big)^2
    &=-\tfrac{1}{2}\tfrac{a^4}{y^2F_{yy}}\bigg[(byF_{by}-yF_y) (\partial_j\pi^j)+\bigg(\tfrac{1}{a^4}\!\!\int(\partial_j \pi^j)\tfrac{d}{d\tau}(a^4yF_y)\bigg)\bigg]^2\\
    &\quad+\tfrac{a^4}{y^2F_{yy}}\Big[\big(2{y}F_y - byF_{by}\big) 6\Psi+ \left({y}^2 F_{yy} - {y}F_y\right) 2\Phi\Big]\\
    &\qquad\qquad\cdot \bigg[(byF_{by}-yF_y) (\partial_j\pi^j)+\bigg(\tfrac{1}{a^4}\!\!\int(\partial_j \pi^j)\tfrac{d}{d\tau}(a^4yF_y)\bigg)\bigg]\,.
\end{aligned}
\end{align}}
Taking $\Phi=\Psi$ in the large scale limit and simplifying terms we finally get\todotag{write}
\begin{align}
    ...
\end{align}
The equation of motion for phonons will be something like
\begin{align}
    \ddot{\pi}_L=-\underbrace{c_s^2 k^2\pi_L}_{\to0}+\text{friction}
\end{align}
giving that $\ddot{\pi}_L\to 0$ as $k\to0$.
For the comoving curvature then
\begin{align}
    \dot{\mathcal{R}}=-\dot{\Psi}+\dot{\mathcal{H}}\,\dot{\pi}_L+\mathcal{H}\,\ddot{\pi}_L
\end{align}
and, upon using the 0i Einstein equation for $\dot{\Psi}+\mathcal{H}\Psi=...$, the vanishing of $\ddot{\pi}$ from the EoM of phonons might be the fastest way to predict that $\mathcal{R}$ is conserved on superhorizon scales.\todotag{Finish this argument}



%%%%%%%%%%%%%%%%%%%%%%%%%%%%%%%%%%%%%%%%%%%%%%%%%%%%%%%%%%%%%%%%%%%%%%%%%%%

\bibliography{bib.bib}

\end{document}

