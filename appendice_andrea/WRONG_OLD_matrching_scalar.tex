
%---------------------------------------------------------------
%==============================================================
\section{Wrong old attempts to match the scalar fields [delete when we succed]}
%==============================================================
%----------------------------------------------------------------


%----------------------------------------------------------------
\subsection{Matching to quintessence scalar field {\color{red}(WRONG VERSION)}}
%----------------------------------------------------------------

Consider the metric in Newtonian gauge
\begin{equation}
    g_{\mu\nu}=a^2\begin{pmatrix}
        -(1+2\Phi) & 0 \\
        0 & (1-2\Psi)\delta_{ij}
    \end{pmatrix}\, \quad \Rightarrow \quad \sqrt{-g}=a^4(1+\Phi-3\Psi)\,.
\end{equation}
The action for the scalar field $\varphi$ with potential $V(\varphi)$ is
\begin{equation}
    S_\varphi = \int d^4x \sqrt{-g} \left[-\frac{1}{2} g^{\mu\nu}\partial_\mu\varphi \partial_\nu\varphi - V(\varphi)\right]\,, \quad X:= -g^{\mu\nu}\partial_\mu\varphi \partial_\nu\varphi\,.
\end{equation}
The energy-momentum tensor is
{\small
\begin{align}
\begin{aligned}
    T_{\mu\nu}&=-\frac{2}{\sqrt{-g}}\frac{\partial \left(\sqrt{-g}\mathcal{L}\right)}{\partial g^{\mu\nu}}
    = -\frac{2}{\sqrt{-g}}\left(\sqrt{-g}\frac{\partial \mathcal{L}}{\partial g^{\mu\nu}}-\frac{1}{2}\sqrt{-g}g_{\mu\nu}\mathcal{L}\right)
    \\
     &= \partial_\mu\varphi \partial_\nu\varphi + g_{\mu\nu}\left[-\frac{1}{2}g^{\alpha\beta}\partial_\alpha\varphi \partial_\beta\varphi - V(\varphi)\right]\\
    &= \frac{\partial_\mu\varphi}{\sqrt{-g^{\alpha\beta}\partial_\alpha\varphi \partial_\beta\varphi}} \frac{\partial_\nu\varphi}{\sqrt{-g^{\alpha\beta}\partial_\alpha\varphi \partial_\beta\varphi}} \left(-g^{\alpha\beta}\partial_\alpha\varphi \partial_\beta\varphi\right) + g_{\mu\nu}\left[\frac{1}{2}\left(-g^{\alpha\beta}\partial_\alpha\varphi \partial_\beta\varphi\right) - V(\varphi)\right]\,.
\end{aligned}
\end{align}
}
We expand around a homogeneous background $\bar{\varphi}(\tau)$ as $\varphi(\tau,\vec{x})=\bar{\varphi}(\tau)+\delta\varphi(\tau,\vec{x})$.
Dropping mixed and second order in the metric perturbations we have 
\begin{align}\label{eq:expansion_X}
\begin{aligned}
    X&= -g^{\mu\nu}\partial_\mu\varphi \partial_\nu\varphi
    = \frac{1}{a^2}(1-2\Phi)(\bar{\varphi}'+\delta\varphi')^2 - \frac{1}{a^2}(1+2\Psi)(\nabla\delta\varphi)^2\\
    &\simeq \frac{1}{a^2}(\bar{\varphi}')^2+ \frac{2}{a^2}(\bar{\varphi}'\delta\varphi'-\Phi (\bar{\varphi}')^2) + \frac{1}{a^2}\left[(\delta\varphi')^2-(\nabla\delta\varphi)^2\right]\\
    &\simeq \frac{1}{a^2}(\bar{\varphi}')^2\left[1+ 2\left(\frac{\delta\varphi'}{\bar{\varphi}'}-\Phi\right) + \left(\frac{\delta\varphi'}{\bar{\varphi}'}\right)^2 - \frac{(\nabla\delta\varphi)^2}{(\bar{\varphi}')^2}\right]+\mathcal{O}(3,\mathrm{mix})\,.
\end{aligned}
\end{align}
Then, again dropping mixed and secnd order in the metric perturbations, we have
\begin{align}
   X^{-1/2}&\simeq \frac{a}{\bar{\varphi}'}\left[1 - \left(\frac{\delta\varphi'}{\bar{\varphi}'}-\Phi\right) +\left(\frac{\delta\varphi'}{\bar{\varphi}'}\right)^2 + \frac{1}{2}\frac{(\nabla\delta\varphi)^2}{(\bar{\varphi}')^2} \right] +\mathcal{O}(3,\mathrm{mix})\,.
\end{align}
%
Matching to a perfect fluid $T_{\mu\nu}=(\rho+p)u_\mu u_\nu + p g_{\mu\nu}$ we identify the velocity field as
\begin{align}\label{eq:velocity_field_expansion}
\begin{aligned}
    u_\mu &= \frac{\partial_\mu\varphi}{\sqrt{-g^{\alpha\beta}\partial_\alpha\varphi \partial_\beta\varphi}} =  X^{-1/2}\left(\bar{\varphi}^\prime+\delta\varphi',\, \nabla\delta\varphi\right)\\
    &\simeq \frac{a}{\bar{\varphi}'}\left[1 - \left(\frac{\delta\varphi'}{\bar{\varphi}'}-\Phi\right) +\left(\frac{\delta\varphi'}{\bar{\varphi}'}\right)^2 + \frac{1}{2}\frac{(\nabla\delta\varphi)^2}{(\bar{\varphi}')^2} \right]\left(\bar{\varphi}^\prime+\delta\varphi',\, \nabla\delta\varphi\right)\\
    &\simeq a\left(1+ \Phi,\, \frac{\nabla\delta\varphi}{\bar{\varphi}'}\right) + \mathcal{O}(2)\,.
\end{aligned}
\end{align}
Expanding $V(\varphi)=V(\bar{\varphi}) + V_{\varphi}(\bar{\varphi})\delta\varphi + \frac{1}{2}V_{\varphi\varphi}(\bar{\varphi})(\delta\varphi)^2$ we identify the \textbf{energy density}
\begin{align}
\begin{aligned}
    \rho &= \frac{1}{2}X + V(\varphi)\\
    &\simeq  \frac{1}{2\,a^2}(\bar{\varphi}')^2\left[1+ 2\left(\frac{\delta\varphi'}{\bar{\varphi}'}-\Phi\right) + \left(\frac{\delta\varphi'}{\bar{\varphi}'}\right)^2 - \frac{(\nabla\delta\varphi)^2}{(\bar{\varphi}')^2}\right]\\
    &\quad + V(\bar{\varphi}) + V_{\varphi}(\bar{\varphi})\delta\varphi + \frac{1}{2}V_{\varphi\varphi}(\bar{\varphi})(\delta\varphi)^2 +\mathcal{O}(3,\mathrm{mix})
\end{aligned}
\end{align}
and the \textbf{pressure} 
\begin{align}
\begin{aligned}
    p &= \frac{1}{2}X - V(\varphi) \\
    &\simeq  \frac{1}{2\,a^2}(\bar{\varphi}')^2\left[1+ 2\left(\frac{\delta\varphi'}{\bar{\varphi}'}-\Phi\right) + \left(\frac{\delta\varphi'}{\bar{\varphi}'}\right)^2 - \frac{(\nabla\delta\varphi)^2}{(\bar{\varphi}')^2}\right]\\
    &\quad- V(\bar{\varphi}) - V_{\varphi}(\bar{\varphi})\delta\varphi - \frac{1}{2}V_{\varphi\varphi}(\bar{\varphi})(\delta\varphi)^2 +\mathcal{O}(3,\mathrm{mix})\,.
\end{aligned}
\end{align}
Finally we have 
\begin{align}
    p=\rho-2V(\varphi) \quad \Rightarrow \quad \delta p = \delta\rho - 2 V_{\varphi}(\bar{\varphi})\delta\varphi\,.
\end{align}
Since we showed that $c_b^2:=\frac{\delta p}{\delta \rho}_{\mid \sigma}=1$ we identify the \textbf{non-adiabatic pressure perturbations} of the field
\begin{align}
    \delta p_{\text{nad}} =\frac{\partial p}{\partial \sigma}_{\mid \rho}\delta\sigma= \delta p - c_b^2 \delta \rho = - 2 V_{\varphi}(\bar{\varphi})\delta\varphi\,\quad \Rightarrow\quad \Gamma:= \frac{\partial p}{\partial \sigma}_{\mid \rho}= - \frac{\delta\varphi}{\delta \sigma_{\mid \rho}}2 V_{\varphi}(\bar{\varphi})\,
\end{align}
and we confim below that $\Gamma$ is a function of the potential only and that indeed $\delta\sigma\propto \delta\varphi$.

Finally note that in the comoving gauge $\varphi^I=x^I$ we have
\begin{align}
    b=\left(\det(g^{\mu\nu}\partial_\mu\varphi^I\partial_\nu\varphi^J)\right)^{1/2} = \sqrt{\det(g^{ij})} = \frac{1}{a^3}(1+3\Psi)\,.
\end{align}
and
\begin{align}\label{eq:expansion_y_quintessence}
    y=-\sqrt{X}&= -\frac{\bar{\varphi}'}{a}\left[1 + \left(\frac{\delta\varphi'}{\bar{\varphi}'}-\Phi\right) - \frac{1}{2}\frac{(\nabla\delta\varphi)^2}{(\bar{\varphi}')^2} \right] +\mathcal{O}(3,\mathrm{mix})\\
    &= -\frac{\bar{\varphi}'}{a} -\frac{\delta\varphi'}{a}+\frac{\bar{\varphi}'\Phi}{a} + \mathcal{O}(2)\,.
\end{align}
Therefore the \textbf{comoving entropy for the scalar field} is 
\begin{align}
    \sigma = \frac{F_y}{b} = a^3(1-3\Psi) y =  -a^2\bar{\varphi}'\left[1 + \left(\frac{\delta\varphi'}{\bar{\varphi}'}-\Phi-3\Psi\right) - \frac{1}{2}\frac{(\nabla\delta\varphi)^2}{(\bar{\varphi}')^2} \right] +\mathcal{O}(3,\mathrm{mix})\,.
\end{align}
and the \textbf{entropy density} (fluctuations) are directly proportional to field fluctuations 
\begin{align}
    s = F_y = y =  -\frac{\bar{\varphi}'}{a} -\frac{\delta\varphi'}{a}+\frac{\bar{\varphi}'\Phi}{a} + \mathcal{O}(2)\,.
\end{align}

Finally the \textbf{equation of state}, upon inverting $\sigma$ to get $\varphi$, reads
\begin{align}
    p=\rho-2V(\varphi) \sim \rho +\mathbf{f}(\sigma)\,,
\end{align}
confirming it is \emph{not} a function of the potential only.


%----------------------------------------------------------------
\subsection{Matching to k-essence scalar field {\color{red}(WRONG VERSION)}}
%----------------------------------------------------------------
Consider the action for a k-essence scalar field $\varphi$ with lagrangian $P(X,\varphi)$
\begin{equation}
    S_\varphi = \int d^4x \sqrt{-g} P(X,\varphi)\,, \quad X:= -g^{\mu\nu}\partial_\mu\varphi \partial_\nu\varphi\,.
\end{equation}
The energy-momentum tensor is
\begin{align}
\begin{aligned}
    T_{\mu\nu}&=-\frac{2}{\sqrt{-g}}\frac{\partial \left(\sqrt{-g}\mathcal{L}\right)}{\partial g^{\mu\nu}}
    = -\frac{2}{\sqrt{-g}}\left(\sqrt{-g}\frac{\partial \mathcal{L}}{\partial g^{\mu\nu}}-\frac{1}{2}\sqrt{-g}g_{\mu\nu}\mathcal{L}\right)
    \\[4pt]
     &= 2P_X \partial_\mu\varphi \partial_\nu\varphi + g_{\mu\nu} P(X,\varphi)
     = u_\mu u_\nu 2X P_X + g_{\mu\nu} P(X,\varphi)\,.
\end{aligned}
\end{align}
The velocity field is $u_\mu = \partial_\mu\varphi/\sqrt{X}$ as in the quintessence case, and it enjoys the same expansion \eqref{eq:velocity_field_expansion}.
Matching to a perfect fluid we identify the \textbf{energy density} and \textbf{pressure} as
\begin{align}\label{eq:energy_density_pressure_k_essence}
\begin{aligned}
    \rho &= 2 X P_X - P(X,\varphi) = \bar{\rho} + \underbrace{(P_X+2XP_{XX})\delta X + (2XP_{X\varphi}-P_\varphi)\delta\varphi}_{=\delta\rho}\,,\\[4pt]
    p&= P(X,\varphi) = \bar{p} + \underbrace{P_X \delta X + P_\varphi \delta\varphi}_{=\delta p}
\end{aligned}
\end{align}
The expressions of $X$ is again given by \eqref{eq:expansion_X}, so that we find the \textbf{energy density and pressure fluctuations}
\begin{align}
\begin{aligned}
    \delta\rho &= \big(P_X+2XP_{XX}\big)\,\,\frac{2}{a^2}\left[\bar{\varphi}'\delta\varphi' - (\bar{\varphi}')^2\Phi\right] + (2XP_{X\varphi}-P_\varphi)\,\delta\varphi\,,\\[4pt]
    \delta p &= P_X\, \frac{2}{a^2}\left[\bar{\varphi}'\delta\varphi' - (\bar{\varphi}')^2\Phi\right] + P_\varphi \, \delta\varphi\,.
\end{aligned}
\end{align}
Similarly we still have $y=-\sqrt{X}$ with epansion \eqref{eq:expansion_y_quintessence}, so that the \textbf{entropy density} is
\begin{align}
    s= F_y = \frac{\partial P}{\partial y}= 2yP_X = -\frac{2P_X}{a} \left(\bar{\varphi}' +\delta\varphi' - \bar{\varphi}'\Phi \right) + \mathcal{O}(2)\,
\end{align}
and since also $b=\det(\partial_\mu\phi^I\partial_\nu\phi^J g^{\mu\nu})^{1/2}$ is unchanged, the \textbf{comoving entropy} is
\begin{align}
    \sigma = \frac{F_y}{b} = a^3(1-3\Psi) \frac{\partial P}{\partial y}= -2a^2P_X \left(\bar{\varphi}' +\delta\varphi' - \bar{\varphi}'(\Phi+3\Psi) \right) + \mathcal{O}(2)\,.
\end{align}
In particular we see that $\delta\sigma\propto \delta\varphi$ as in the quintessence case.

Finally we compute $c_b^2= \frac{\partial p}{\partial \rho}_{\mid \sigma}$ using the general formula in the previous sections, and in turn identify the non-adiabatic pressure perturbations $\delta p_{\text{nad}} = \frac{\partial p}{\partial \sigma}_{\mid \rho}\delta\sigma$.
We have $F(y,b)=P(X,\varphi)$ and $dX=-2y\,dy$ so that
\begin{align}
\begin{aligned}
    c_b^2&\equiv \frac{-b^2F_{bb} F_{yy}+(F_y-bF_{yb})^2}{F_{yy}\,(yF_y-bF_b)}= \frac{(P_X)^2}{(P_X)^2+2XP_XP_{XX}}
\end{aligned}
\end{align}
Comparing to \eqref{eq:energy_density_pressure_k_essence} above we find
\begin{align}
\begin{aligned}
    \delta p &= c_b^2 \delta\rho + \frac{\partial p}{\partial \sigma}_{\mid \rho}\delta\sigma
    = P_X \delta X + \frac{P_X\Big(2XP_{X\varphi}-P_\varphi)}{P_X+2XP_{XX}}\delta\varphi  + \frac{\partial p}{\partial \sigma}_{\mid \rho}\delta\sigma \overset{!}{=} P_X\delta_X + P_\varphi\delta\varphi\,,
\end{aligned}
\end{align}
we identify the \textbf{non-adiabatic pressure perturbations} of the k-essence field
\begin{align}
\begin{aligned}
    \delta p_{\text{nad}} &=\frac{\partial p}{\partial \sigma}_{\mid \rho}\delta\sigma
    = \left(P_\varphi - \frac{P_X\Big(2XP_{X\varphi}-P_\varphi)}{P_X+2XP_{XX}}\right)\delta\varphi\\
    & \Rightarrow\quad \Gamma:= \frac{\partial p}{\partial \sigma}_{\mid \rho}= \left(P_\varphi - \frac{P_X\Big(2XP_{X\varphi}-P_\varphi)}{P_X+2XP_{XX}}\right)\frac{\delta\varphi}{\delta \sigma_{\mid \rho}}\,.
\end{aligned}
\end{align}


%----------------------------------------------------------------
\subsection{Matching to 5-essence remastered}
%----------------------------------------------------------------
The matching should be adjusted so as to preserve shift symmetry of $\phi^0$, thus forbidding the identification of $\varphi$ with $\phi^0$ directly since $V(\varphi)$ would otherwise break it.
In turn, the identification $y=\sqrt{X}$ need some rethinking.

In the 5-essence case we can express kinetic energy and potential in terms of energy density and pressure. 
Next we use the expression \eqref{eq:rho_pressure_exp_nonbaro} of the stress tensor in the fluid EFT to express kinetic energy and potential in terms of $F$ and its derivatives
\begin{align}
    \rho=\tfrac12 X +V, \quad p =\tfrac12 X -V\quad \Rightarrow\quad \begin{array}{c}
        X=\rho+p = yF_y + bF_b,\\
        V(\varphi) = \tfrac12(\rho - p) \equiv -F +\tfrac12 (yF_y+ bF_b)\,.
    \end{array}
\end{align}
We now invert these relations to identify $y,b$ in terms of $X,V$.
We need further constraints to identify the form of the function $F$.
Ideally, we would like to identify temperature and kinetic energy $y\sim X$.
Enforcing $F_{yb}=0$ and the $c_b^2=1$ in the case with no potential, a solution is given by
\begin{equation}
    F=\frac{y^2}{2}-\beta \log b \quad \Rightarrow\quad \begin{array}{c}
         X=yF_y+bF_b=y^2-\beta\\
         V=-F+\tfrac12(yF_y+bF_b) = \beta\big(\log b-\tfrac12\big)
    \end{array}
\end{equation}
which allow for $V\neq0$ by taking $\beta\neq0$.
This suggests $X\sim y^2$ and $V\propto \log b$ which is somewhat physically reasonable.
In turn we get
\begin{align}
    c_b^2=\frac{\partial p}{\partial \rho}_{|\sigma}=\frac{-b^2F_{bb}y^2F_{yy}+\big(yF_y-byF_{by}\big)^2}{y^2F_{yy}(yF_y-bF_b)}=\frac{y^2-\beta}{y^2+\beta}
\end{align}
confirming we can have $c_s^2=1$ if $V=0$ that is $\beta=0$.
Next we identify energy density, pressure and comoving entropy specified by this form of $F$
\begin{align}\label{eq:enerrgy_density_pressure_entropy_vs_y_b}
    \rho=\tfrac12 y^2+\beta\log b, \quad p=\tfrac12 y^2+\beta-\beta\log b=\rho+\beta-2\beta\log b, \quad \sigma=\frac{F_y}{b}=\frac{y^2}{b}.
\end{align}
Now we invert the expressions to use $\rho,\sigma$ as independent thermodynamic variables in place of $y,b$.
This will allow to compute the nonadiabatic pressure.
Combining the equations \eqref{eq:enerrgy_density_pressure_entropy_vs_y_b} for $\rho$ and $\sigma$ we get
\begin{align}
    \rho+\beta\log\sigma=\tfrac12 y^2+\beta\log y^2\quad\Rightarrow\quad y=y(\rho,\sigma),\quad b=\frac{y^2}{\sigma}=\frac{y^2(\rho,\sigma)}{\sigma}.
\end{align}
There is no need to explicitly invert the relation for $y$, since we ultimately want to express $\delta P_{\text{nad}}$ in terms of $y,b$.
In practice, differentiating the first one wrt $\rho$ at fixed $\sigma$ we get
\begin{align}
    1=y\frac{\partial y}{\partial\rho}_{|\sigma}+\frac{2\beta}{y}\frac{\partial y}{\partial\rho}_{|\sigma}\quad\Rightarrow\quad \begin{array}{c}\frac{\partial y}{\partial\rho}_{|\sigma}=\frac{y}{y^2+2\beta}\\[3pt]
    \frac{\partial b}{\partial\rho}_{|\sigma}=\frac{2y}{\sigma}\frac{\partial y}{\partial\rho}_{|\sigma}=\frac{2y^2}{\sigma}\frac{1}{y^2+2\beta}=\frac{2b}{y^2+2\beta}\,.
    \end{array}
\end{align}
Similarly differentiating wrt $\sigma$ at fixed $\rho$ we get
\begin{align}
    \frac{\beta}{\sigma}=y\frac{\partial y}{\partial\sigma}_{|\rho}+\frac{2\beta}{y}\frac{\partial y}{\partial\sigma}_{|\rho}\quad\Rightarrow\quad \begin{array}{c}
        \frac{\partial y}{\partial\sigma}_{|\rho}=\frac{\beta}{\sigma}\frac{y}{y^2+2\beta}=\frac{\beta b}{y(y^2+2\beta)}, \\[3pt]
        \frac{\partial b}{\partial\sigma}_{|\rho}=\frac{2y}{\sigma}\frac{\partial y}{\partial\sigma}_{|\rho}-\frac{y^2}{\sigma^2}=\frac{2\beta y^2}{\sigma^2(y^2+2\beta)}-\frac{y^2}{\sigma^2}=-\frac{y^2}{\sigma^2}\frac{y^2}{y^2+2\beta}=-\frac{b^2}{y^2+2\beta}
        \end{array}
\end{align}
We can finally express the adiabatic and entropic pressure fluctuations
\begin{align}\label{eq:cb_gamma_vs_y_b}
    &c_b^2:=\frac{\partial p}{\partial\rho}_{|\sigma}=1-\frac{2\beta}{b}\frac{\partial b}{\partial\rho}_{|\sigma}= 1- \frac{2\beta}{b}\frac{2b}{y^2+2\beta}=\frac{y^2-2\beta}{y^2+2\beta}\\[3pt]
    &\Gamma:=\frac{\partial p}{\partial\sigma}_{|\rho}=-\frac{2\beta}{b}\frac{\partial b}{\partial\sigma}_{|\rho}=\frac{2\beta b}{y^2+2\beta}
\end{align}
In the absence of potential $\beta=0$, then $c_b^2\equiv1$ and $\Gamma$ vanishes.
The total pressure fluctuations are then
\begin{align}
    \delta p=\delta\rho-2\frac{\delta b}{b}= c_b^2\delta\rho+\Gamma\delta\sigma
\end{align}

Finally we go back to the variables $X$ and $V(\varphi)$.
From the above expression \eqref{eq:cb_gamma_vs_y_b} for $\Gamma$ rewrite the nonadiabatic pressure perturbations as
\begin{align}
    \Gamma = \frac{\partial p}{\partial \sigma}_{|\rho}=\frac{2\beta b}{y^2+2\beta} = \frac{2\beta}{X+3\beta}\,\exp\Big(\frac{\tfrac12 \beta+V(\varphi)}{\beta}\Big).
\end{align}
We also express the comoving entropy and its variation in terms of $X$ and $V(\varphi)$ as
{\small
\begin{align}
\begin{aligned}
    &\sigma=\frac{y^2}{b}=(X+\beta)\, {\exp\Big(-\tfrac{\tfrac12 \beta+V(\varphi)}{\beta}\Big)}\\
    &\,\Rightarrow\quad \delta\sigma = \frac{\partial\sigma}{\partial X}\delta X + \frac{\partial\sigma}{\partial V}\frac{\delta V}{\delta \varphi} \delta \varphi = \exp\Big(-\tfrac{\tfrac12 \beta+V(\varphi)}{\beta}\Big)\left(\delta X - \tfrac{X+\beta}{\beta} V_{\varphi}\delta \varphi\right)\\
    &\qquad\qquad= \exp\Big(-\tfrac{\tfrac12 \beta+V(\bar{\varphi})}{\beta}\Big)\left(2\dot{\bar{\varphi}}\dot{\delta\varphi}- \tfrac{\dot{\bar{\varphi}}^2+\beta}{\beta} V_{\varphi}\delta \varphi\right)
\end{aligned}
\end{align}}
In conclusion, the nonadiabatic pressure perturbations are
\begin{align}
    \delta P_{\text{nad}} = \Gamma \delta\sigma = \frac{2\beta}{X+3\beta}\,\left(2\dot{\bar{\varphi}}\,\dot{\delta\varphi}- \frac{\dot{\bar{\varphi}}^2+\beta}{\beta} V_{\varphi}\delta \varphi\right)\,.
\end{align}




\subsubsection{Alternative solution 1}
More generally, repeating the above argument (matching the stress tensor of the scalar field, imposing $F_{by}=0$, $y\sim \sqrt{X}$ and $c_b^2=1$) we find solutions of the form $F=\tfrac12 y^2 + C b^\beta$.
With such a form for $F$ we again identify $X,V(\varphi)$ in terms of $y,b$ 
\begin{align}
    X=yF_y+bF_b=y^2 + C\beta b^\beta,\quad V=-F+\tfrac12(yF_y+bF_b)=C\big(\beta-1\big) b^\beta.
\end{align}
In the case $V=0$ we might $\beta=1$ to allow for $C\neq0$ so that $X=y^2 + C b$.
However enforcing $c_s^2\equiv1$ in the quintessence case gives
\begin{align}
    1\overset{!}{=}\frac{-b^2F_{bb}+\frac{\big(yF_y-byF_{by}\big)^2}{y^2F_{yy}}}{(yF_y-bF_b)}=\frac{-C\beta(\beta-1) b^\beta+y^2}{y^2 + C\beta b^\beta}= 1 - \frac{C\beta^2 b^\beta}{y^2 + C\beta b^\beta}\quad \Rightarrow\quad \beta=0,
\end{align}
which again implies the potential vanishes...

We now repeat the previous reasoning.
We obtain energy density, pressure and comoving entropy
\begin{align}
    \rho=\tfrac12 y^2 -Cb^\beta,\quad p=\tfrac12 y^2 + C(1-\beta) b^\beta = \rho+C(2-\beta)b^\beta, \quad \sigma=\frac{F_y}{b}=\frac{y^2}{b}.
\end{align}
We combine the equations to (formally) use $\rho,\sigma$ as thermodynamic variables instead of $y,b$
\begin{align}
    \rho=\tfrac12 y^2 -C\big(\frac{y^2}{\sigma}\big)^\beta,\quad \sigma=\frac{y^2}{b}\quad \Rightarrow\quad y=y(\rho,\sigma),\quad b=\frac{y^2(\rho,\sigma)}{\sigma}.
\end{align}
We do some massaging to find $\frac{\partial \cdot}{\partial \rho}_{|\sigma}$ and $\frac{\partial \cdot}{\partial \sigma}_{|\rho}$ without having to invert the relation $y=y(\rho,\sigma),b=b(\rho,\sigma)$ explicitly.
Namely, differentiating the first one wrt $\rho$ at fixed $\sigma$ we get
\begin{align}
    1=y\frac{\partial y}{\partial\rho}_{|\sigma}-C\beta\big(\frac{y^2}{\sigma}\big)^{\beta-1}\frac{2y}{\sigma}\frac{\partial y}{\partial\rho}_{|\sigma}=y\frac{\partial y}{\partial\rho}_{|\sigma}-C\beta b^{\beta}\frac{2}{y}\frac{\partial y}{\partial\rho}_{|\sigma}
    \quad\Rightarrow\quad \begin{array}{c}
    \frac{\partial y}{\partial\rho}_{|\sigma}=\frac{y}{y^2-2C\beta b^{\beta}},\\[3pt]
    \frac{\partial b}{\partial\rho}_{|\sigma}=\frac{2y}{\sigma}\frac{\partial y}{\partial\rho}_{|\sigma}=\frac{2b}{y^2-2C\beta b^{\beta}}\,.
    \end{array}
\end{align}
Similarly differentiating wrt $\sigma$ at fixed $\rho$ we get
{\small
\begin{align}
    \frac{\partial b}{\partial \sigma}_{|\rho}=-\frac{b^2}{y^2+2C\beta b^\beta}, \quad
     \frac{\partial y}{\partial \sigma}_{|\rho}=-C\beta\frac{b^{\beta-1}}{y}\frac{\partial b}{\partial \sigma}_{|\rho}=-\frac{C\beta b^{\beta+1}}{y(y^2+2C\beta b^\beta)}\,.
\end{align}}
Then finally we compute
\begin{align}
    c_b^2&=\frac{\partial p}{\partial \rho}_{|\sigma}=1+C(2-\beta)\beta b^{\beta-1} \frac{\partial b}{\partial \rho}_{|\sigma}=1+\frac{2C(2-\beta)\beta b^{\beta}}{y^2-2C\beta b^{\beta}}=\frac{y^2-2C\beta^2 b^\beta}{y^2-2C\beta b^{\beta}}\,,
    \\[3pt]
    \Gamma&=\frac{\partial p}{\partial \sigma}_{|\rho}=C(2-\beta)\beta b^{\beta-1}\frac{\partial b}{\partial \sigma}_{|\rho}=-\frac{C(2-\beta)\beta b^{\beta+1}}{y^2+2C\beta b^{\beta}}\,.
\end{align}
{\color{red}
Now proceeding as in the previous solution, we do
\begin{itemize}
    \item now invert $y$ and $b$ in terms of $X$ and $V$ as above.
    \item next express the entropy and its variation in terms of $X$ and $V$.
    \item next express the non-adiabatic $\Gamma$ and pressure perturbations $\delta P_{\text{nad}}$ in terms of $X$ and $V$.
\end{itemize}
}




\subsubsection{Alternative solution 2}
Alternatively we can look for a solution $F=Cy^\alpha b^\beta$, then\todotag{Not sure it is worth it}
\begin{align}
    X=yF_y+bF_b=(\alpha+\beta)F,\quad V=-F+\tfrac12 (yF_y-bF_b)=\big(-1+\tfrac12(\alpha+\beta)\big)F.
\end{align}
In the case $V=0$ we take $\alpha+\beta=2$, that allows for $\alpha=1,\beta=0$ so that indeed $X=Cy^2$.
Enforcing $c_s^2\equiv1$ in the quintessence case gives, provided $\alpha\neq \beta$ 
\begin{align}
    1\overset{!}{=}\frac{-b^2F_{bb}+\frac{\big(yF_y-byF_{by}\big)^2}{y^2F_{yy}}}{(yF_y-bF_b)}=\frac{-\beta(\beta-1)F+\frac{\alpha(1-\beta)^2F^2}{(\alpha-1)F}}{(\alpha-\beta)F}=\frac{\cancel{F(\alpha-\beta)}}{\cancel{F(\alpha-\beta)}}\frac{1-\beta}{\alpha-1}
    \Rightarrow \alpha+\beta=2,
\end{align}
which again implies the potential vanishes...
Note that both $\alpha=0,1$ are actually allowed, in the first case no $F_y,F_{yy}$ are present at all, while in the latter we can avoid division by $F_yy$.