%--------------------------------------
%=======================================
\section{Notations, conventions \& cosmology}
%=======================================
%---------------------------------------

We mainly follow the standard notation from \cite{MaBertschinger_CosmologicalPerturbationTheorySynchronousConformalNewtonianGauges_1995,Baumann_Cosmology_2022}.
The metric has signature $-+++$.
The conformal time is denoted by $\tau$ and differentiation with respect to it is denoted by $\dot{f}$.
The proper time of a comoving observer is denoted by $t$, related to conformal time by $dt=a \,d\tau$, and differentiation with respect to it is denoted by $f'$.

We parametrize the general perturbation to a homogeneous isotropic FLRW metric as
{\small\begin{equation} \label{eq:metric_general_form_svt}
-\mathrm{d} s^{2}:=g_{\mu\nu}dx^\mu dx^\nu=a^{2}(\tau)\left[-(1+2 A) \,\mathrm{d} \tau^{2}+2 B_{i} \mathrm{~d} x^{i} \mathrm{~d} \tau+\left((1+2C)\delta_{ij}+2E_{ij}\right) \mathrm{d} x^{i} \,\mathrm{~d} x^{j}\right],
\end{equation}}
with SVT decomposition
{\small\begin{equation*}
\underbrace{A,\, C}_{\text {scalar }}, \quad
B_{i}=\underbrace{\partial_{i} {B}}_{\text {scalar }}+\underbrace{\hat{B}_{i}}_{\text {vector }},\quad E_{i j}=\underbrace{\partial_{\langle i} \partial_{j\rangle} {E}}_{\text {scalar }}+\underbrace{\partial_{(i} \hat{E}_{j)}}_{\text {vector }}+\underbrace{\tilde{E}_{i j}}_{\text {tensor }},
\end{equation*}}
where we use the shorthand
{\small\begin{align}\label{eq:convention_svt_decomposition}
\partial_{\langle i} \partial_{j\rangle} {E} \equiv\left(\partial_{i} \partial_{j}-\frac{1}{3} \delta_{i j} \nabla^{2}\right) \tilde{E},\qquad
\partial_{(i} \hat{E}_{j)}\equiv \frac{1}{2}\left(\partial_{i} \hat{E}_{j}+\partial_{j} \hat{E}_{i}\right).
\end{align}}
%
The energy-momentum tensor of a perturbed perfect fluid is
\begin{align}\label{eq:em_tensor_fluid_perturbed}
    T^\mu_\nu= u^\mu u_\nu (\rho+p) + p \delta^\mu_\nu + \Sigma^\mu_\nu
\end{align}
The 4-velocity $u^\mu$ and the peculiar velocity $v^i$, split into scalar and vector part, are\footnote{Fourier transform convention $f(k)=\int dx e^{-ikx}f(x)$.}
{\small\begin{align}\label{eq:velocity_decomposition}
\begin{aligned}
    u^\mu=\frac{1}{a}\left(1-A,\,v^i\right),\quad u_\mu=a\left(-(1+A),\,v_i\right),\quad v^i=\delta^{ij}v_j,\\[2pt]
    v_i=\partial_i v + \hat{v}_i, \quad \theta= \partial_i v^i = ik_i v^i = -k^2 v, \quad \hat{v}^i=(\nabla\times \omega)^i
\end{aligned}\end{align}}
The (symmetric) anisotropic stress $\Sigma^\mu_\nu$ is traceless and orthogonal to the 4-velocity, and the remaining DoFs are again decomposed into scalar, vector and tensor part
{\small\begin{align}\label{eq:anisotropic_stress_decomposition}
\begin{aligned}
    &\Sigma^\mu_\mu=0,\quad \Sigma^\mu_\nu u_\mu=\Sigma^\mu_\nu u^\nu=0, \quad \Rightarrow\quad \Sigma^0_0=\Sigma^0_i=\Sigma^i_0=0, \quad \Sigma^i_i=0, \,,\\[3pt]
    &\Sigma^i_j=\Sigma^i_j =\Sigma_{ij},\quad
    \Sigma_{i j}=\underbrace{\partial_{\langle i} \partial_{j\rangle} \Sigma}_{\text {scalar }}+\underbrace{\partial_{(i} \hat{\Sigma}_{j)}}_{\text {vector }}+\underbrace{\tilde{\Sigma}_{i j}}_{\text {tensor }}.
\end{aligned}
\end{align}}
The energy-momentum tensor is finally decomposed as
\begin{align}\label{eq:energy_momentum_tensor_decomposition}
\begin{aligned}
    T_{0}^{0} & =-(\bar{\rho}+\delta \rho), \\
    T_{i}^{0} & =(\bar{\rho}+\bar{p}) \,v_{i}=-T_{\,0}^{i}, \\
    T_{j}^{i} & =(\bar{p}+\delta p) \delta_{j}^{i}+\Sigma_{j}^{i}, \quad \Sigma_{i}^{i}=0.
\end{aligned}
\end{align}


%=================================================
\subsection{Gauge invariant variables \& comoving curvature perturbations}
%=================================================
Under a transformation $x^\mu \mapsto x^\mu +\xi^\mu$ with $\xi^\mu=(T,\,\partial^i L+\hat{L}^i)$, the pertubations transform as
\begin{align}
\begin{aligned}
    \begin{array}{ll}
    A \mapsto A-\dot{T}-\mathcal{H} T, & \\
    B \mapsto B+T-\dot{L}, & \hat{B}_{i} \mapsto \hat{B}_{i}-\dot{\hat{L}}_{i}, \\
    C \mapsto C-\mathcal{H} T-\frac{1}{3} \nabla^{2} L, & \\
    E \mapsto E-L, & \hat{E}_{i} \mapsto \hat{E}_{i}-\hat{L}_{i}, \quad \tilde{E}_{i j} \mapsto \tilde{E}_{i j} .
    \end{array}
\end{aligned}
\end{align}
and fluid perturbations transform as
\begin{align}
\begin{aligned}
    \delta \rho & \mapsto \delta \rho-\dot{\bar{\rho}}\, T, \\
    \delta P & \mapsto \delta P-\dot{\bar{P}}\, T, \\
    v_{i} & \mapsto v_{i}+\dot{L}_{i}, \\
    \Sigma_{i j} & \mapsto \Sigma_{i j}
\end{aligned}
\end{align}
The gauge-invariant Bardeen potentials are
\begin{align}
\begin{aligned}
    \Psi & \equiv A+\mathcal{H}\left(B-\dot{E}\right)+\left(\dot{B}-\ddot{E}\right), \quad \hat{\Phi}_{i} \equiv \hat{B}_{i}-\dot{\hat{E}}_{i}, \\
    \Phi & \equiv-C+\frac{1}{3} \nabla^{2} E-\mathcal{H}\left(B-\dot{E}\right), \qquad \hat{E}_{i j}.
\end{aligned}
\end{align}
The gauge-invariant density perturbation are
\begin{align}
\begin{aligned}
    \bar{\rho} \Delta \equiv \delta \rho+\dot{\bar{\rho}}(v+B)
\end{aligned}
\end{align}
The gauge-invariant comoving curvature perturbation $\mathcal{R}$, i.e. the geometric 3-curvature of \emph{comoving} spatial slices, is defined by
\begin{align}\label{eq:comoving_curvature_perturbations_gauge_inv_and_newtonian}
\begin{aligned}
    \mathcal{R}\equiv -C+\tfrac13\nabla^2E-\mathcal{H}(B+v) = \Phi -\mathcal{H}\big(v+\dot{E}\big).
\end{aligned}
\end{align}
Beware another gauge invariant quantity $\zeta$ is often used in place of the comoving curvature $\mathcal{R}$, namely\footnote{Beware different signs for $\zeta$ and $\mathcal{R}$ are used in the literature, as well as swapping $\zeta\leftrightarrow\mathcal{R}$.}
{\small\begin{align}
\begin{aligned}
    \zeta & =-C+\frac{1}{3} \nabla^{2} E+\mathcal{H} \frac{\delta \rho}{\dot{\bar{\rho}}} = \Phi +\mathcal{H}(B-\dot{E})+ \mathcal{H}\frac{\delta \rho}{\dot{\bar{\rho}}} = \mathcal{R}+\frac{\mathcal{H}}{\dot{\bar{\rho}}} \bar{\rho} \Delta.
\end{aligned}
\end{align}}
Only two of $\Delta, \mathcal{R}, \zeta$ are thus independent.
Moreover on large scales $\zeta \simeq \mathcal{R}$ since from the gauge-invariant Poisson equation \eqref{eq:poisson_gauge_invariant} and the 1st Friedman equation we have\footnote{The second equality holds for total species and for separately conserved $\nabla_\mu T^\mu_0=$ species.}
{\small
\begin{align}
\begin{aligned}
    \zeta-\mathcal{R}=\mathcal{H}\frac{\bar{\rho}\, \Delta }{\dot{\bar{\rho}}}
    = -\tfrac13 \frac{\bar{\rho}\Delta}{\bar{\rho}+\bar{p}}
    =\frac{k^2}{\mathcal{H}^2}\frac{\Phi}{\tfrac92 \Big(1+\frac{\bar{p}}{\bar{\rho}}\Big)}
    \,\to 0\quad\text{as}\,\,\frac{k}{\mathcal{H}}\to0.
\end{aligned}
\end{align}}
A distinctive feature of \emph{adiabatic} perturbations is that they are conserved in the superhorizon limit.




%=====================================================================
\subsection{Conformal Newtonian Gauge}
%====================================================================

In this gauge, vector and tensor modes are excluded from the beginning $\hat{B}=\hat{E}=\bar{E}=0$, and the residual gauge freedom is used to kill the non-diagonal scalar terms $\tilde{B}=\tilde{E}=0$, leaving \emph{no residual gauge freedom}.
The metric is conventionally\footnote{Beware of different conventions in the literature, swapping $\phi \leftrightarrow \psi$ or their signs. Here we adopt the convention of Ma and BertschingerBaumann MaBertschinger, Baumann \cite{Baumann_Cosmology_2022} and\texttt{CLASS} \cite{Lesgourgues_CosmicLinearAnisotropySolvingSystemCLASSIOverview_2011}, so that $\psi\approx \phi$ in the absence of anisotropic stress.
Dodelson \cite{DodelsonSchmidt_ModernCosmology_2025} uses $\phi\mapsto -\phi$. Weinberg \cite{Weinberg_AdiabaticModesCosmology_2003,Weinberg_Cosmology_2008} and Pajer swap $\phi\leftrightarrow \psi$.} written by calling $A=\psi$, $C=-\phi$
\begin{align}\label{eq:newtonian_gauge_metric}
\begin{aligned}
    &g_{\mu\nu}=a^2\begin{pmatrix}
        -(1+2\psi) & 0 \\
        0 & (1-2\phi)\delta_{ij}
    \end{pmatrix}\equiv a^2(\eta_{\mu\nu}+h_{\mu\nu})\\[5pt]
    &\quad \Rightarrow \quad h^{00}=h_{00}=-2\psi,\quad h^{ii}=h_{ii}=-6\phi\,.
\end{aligned}
\end{align}
%
In the Newtonian limit $\psi$ plays the role of the gravitational potential, while $\phi$ is a local perturbation to the scale factor.
In the absence of anisotropic stress, Einstein equations imply  $\psi\simeq \phi$.
%
The Christoffel symbols in this gauge
\begin{align}\label{eq:christoffel_newtonian_gauge}
    \begin{aligned}[t]
    \Gamma_{00}^{0} & =\mathcal{H}+\dot\psi,  \\
    \Gamma_{0 i}^{0} & =\partial_{i} \psi,  \\
    \Gamma_{00}^{i} & = \partial_{i} \psi,
    \end{aligned}
    \qquad
    \begin{aligned}[t]
    \Gamma_{i j}^{0} & =\mathcal{H} \delta_{i j}-\left[\dot\phi+2 \mathcal{H}(\phi+\psi)\right] \delta_{i j}, \\
    \Gamma_{j 0}^{i} & =\mathcal{H} \delta_{ij}-\dot\phi \delta_{ij}, \\
    \Gamma_{j k}^{i} & = -\partial_j \phi\, \delta_{ki} - \partial_k \phi\,\delta_{ji} + \partial_i \phi\,\delta_{jk}.
    \end{aligned}
\end{align}
The perturbed Einstein equations in this gauge, with (i,j)-component already split int trace and traceless part and $s$ denotes different species, are
\begin{align}
\label{eq:EE_00_poisson}
k^{2} \phi+3 \mathcal{H}(\dot{\phi}+\mathcal{H} \psi)&=-4 \pi G a^{2} \sum_{s} \bar{\rho}_{s} \,\delta_{s},
\\
%
k^{2}(\dot{\phi}+\mathcal{H} \psi)&=4 \pi G a^{2} \sum_{s}\left(\bar{\rho}_{s}+\bar{P}_{s}\right) \theta_{s},  \label{eq:EE_0j}
\\
%
\ddot{\phi}+\mathcal{H}(2 \dot{\phi}+\dot{\psi})+\left(2 \frac{\ddot{a}}{a}-\mathcal{H}^{2}\right) \psi+\frac{k^{2}}{3}(\phi-\psi)&=4 \pi G a^{2} \sum_{s} \delta P_{s}, \label{eq:EE_ij_diagonal}
\\
%
k^{2}(\phi-\psi)&=12 \pi G a^{2} \sum_{s}\left(\hat{k}^{j} \hat{k}_{i}-\frac{1}{3}\delta_{i}^{j}\right) \Sigma_{s\,j}^{i}. \label{eq:EE_ij_off_diagonal}
\end{align}
A linear combination of the 00 and 0i Einstein equation gives the gauge-invariant Poisson equation
\begin{align}\label{eq:poisson_gauge_invariant}
    \nabla^2\Phi= 4\pi G a^2 \bar{\rho}\Delta,  \text{for}\quad \Delta:=\delta +\frac{\dot{\bar{\rho}}}{\bar{\rho}}(v+B),
\end{align}
where $\Phi$ is the gauge-invariant Bardeen potential, equal to $\phi$ in Newtonian gauge, and $\Delta$ is the gauge-invariant density contrast, where again we used $B=0$ in Newtonian gauge. 
% 
The continuity and Euler equations $\nabla_\mu T^\mu_\nu=0$ for a \emph{separately conserved} species are\footnote{For now we keep {\color{blue}scalar anisotropic stress} and highlight in blue.}
{\small\begin{align}
    &\dot{\bar{\rho}} +3\mathcal{H}(\bar{\rho}+\bar{p})=0\,, \label{eq:backround_continuity}\\
    &\dot{\delta\rho}= -3\mathcal{H}(\delta\rho+\delta P) + (\bar{\rho}+\bar{P})\big(3\dot{\phi}-\partial_j v^j \big)\,,\label{eq:perturbed_continuity_drho}\\
    &\dot{v}_j+ \mathcal{H}v_j-3\mathcal{H}\frac{\dot{\bar{p}}}{\dot{\bar{\rho}}}v_j+\frac{\partial_j\delta p}{(\bar{\rho}+\bar{p})}+\partial_j\psi{\color{blue}+\frac{\partial_\ell \Sigma^{\ell}_{j}}{(\bar{\rho}+\bar{p})}}=0\,,\label{eq:euler_vector_form}
\end{align}}
Introducing the density contrast $\delta=\delta\rho/\bar{\rho}$, the velocity divergence \eqref{eq:velocity_decomposition} and the scalar anisotropic stress \eqref{eq:anisotropic_stress_decomposition}, and taking the divergence of the Euler equation, we get
\begin{align}
    &\dot{\delta}= 3\mathcal{H}\left(\frac{\bar{p}}{\bar{\rho}}-\frac{\delta p}{\delta\rho}\right)\delta + \left(1+\frac{\bar{p}}{\bar{\rho}}\right)3\dot{\phi} - \left(1+\frac{\bar{p}}{\bar{\rho}}\right)\theta\, \label{eq:continuity_overdensity_form}\\
    &\dot{\theta}+ \mathcal{H}\theta-3\mathcal{H}\frac{\dot{\bar{p}}}{\dot{\bar{\rho}}}\theta+\frac{\nabla^2\delta P}{(\bar{\rho}+\bar{P})}+\nabla^2\psi\,\,{\color{blue}+\frac{2}{3}\frac{\nabla^4 \Sigma}{(\bar{\rho}+\bar{p})}}=0\,.\label{eq:euler_divergence_form}
\end{align}
These are combined into a 2nd order equation for overdensities\todotag{Check 2nd order eq are correct}
\begin{align}
\begin{aligned}
    \ddot{\delta\rho}=&-3\big(\dot{\mathcal{H}}+4\mathcal{H}^2\big)\left(1+\frac{\delta p}{\delta\rho}\right)\delta\rho
    -\mathcal{H}(7\dot{\delta\rho}+3\dot{\delta p})+\nabla^2\delta P +(\bar{\rho}+\bar{P})\nabla^2\psi\\
    &+3(\bar{\rho}+\bar{p})\left[\left(\mathcal{H}+\frac{\dot{\bar{p}}}{\bar{\rho}+\bar{p}}\right)\dot{\phi}+\ddot{\phi}\right] {\color{blue}+ \frac{2}{3}\nabla^4\Sigma}\,.
\end{aligned}
\end{align}
Equivalently, in terms of the overdensity $\delta$ we have\todotag{Check 2nd order eq are correct}
\begin{align}
\begin{aligned}
    \ddot{\delta}=&-\dot{\delta}\mathcal{H}\left(1-6\frac{\bar{p}}{\bar{\rho}}\right)
    +\delta\left[3(\dot{\mathcal{H}}+4\mathcal{H}^2)\left(\frac{\bar{p}}{\bar{\rho}}-\frac{\delta p}{\delta\rho}\right)+3\mathcal{H}\frac{\dot{\bar{p}}}{\bar{\rho}}\right]
    +\frac{\nabla^2\delta p}{\bar{\rho}}\\
    & +\left(1+\frac{\bar{p}}{\bar{\rho}}\right)\nabla^2\psi
    +3\left(1+\frac{\bar{p}}{\bar{\rho}}\right)\left[\left(\mathcal{H}+\frac{\dot{\bar{p}}}{\bar{\rho}+\bar{p}}\right)\dot{\phi}+\ddot{\phi}\right] {\color{blue}+ \frac{2}{3\bar{\rho}}\nabla^4\Sigma}\,.
\end{aligned}
\end{align}



%=======================================================================
\subsection{Other gauges}
%====================================================================

%-------------------------------------------------------
\subsubsection*{Synchronous gauge}
%-------------------------------------------------------
If ever needed we collect here the relevant expressions for the synchronous gauge\dots
In particular recall in this gauge there is some residual gauge freedom that can be fixed by demanding the peculiar velocity of a \emph{cold} ($w\simeq 0$) species to vanish.\questiontag{Finish this if needed eventually cf. \cite{Weinberg_Cosmology_2008}}


%----------------------------------------------------------------
\subsubsection*{Comoving gauge}
%--------------------------------------------------------------------
To simplify some of the expressions in the fluid EFT, it will be useful to work in a gauge where the peculiar velocity $v=0$ or the total mometum density $v+B=0$ vanish.\questiontag{Finish this if needed eventually}


%--------------------------------------------------------------------
\subsubsection{ADM decomposition}
%---------------------------------------------------------------------

In order to make the comoving curvature perturbation $\mathcal{R}$ appear explicitly in the spatial metric (and later match it to phonon variables), it is useful to work in the ADM decomposition
\begin{align}
\begin{aligned}
    ds^2
    =
    a^2(\tau)\Big[
    - N^2(\tau,\mathbf{x})\,d\tau^2
    +
    h_{ij}(\tau,\mathbf{x})\bigl(dx^i+N^i(\tau,\mathbf{x})\,d\tau\bigr)\bigl(dx^j+N^j(\tau,\mathbf{x})\,d\tau\bigr)
    \Big].
\end{aligned}
\end{align}
Here $N$ is the lapse, $N^i$ the shift, and $h_{ij}$ the induced metric on $\tau=\mathrm{const}$ hypersurfaces.

Since $\mathcal{R}$ is defined as the intrinsic 3-curvature perturbation of \emph{comoving} hypersurfaces, we can use the 2 scalar DoFs $T,L$ of the gauge freedom to fix the comoving condition and remove the scalar shear so that
\begin{align}
\begin{aligned}
    B+v=0, \,\, E=0\quad \Rightarrow\quad \mathcal{R}=-C+\frac{1}{3}\nabla^2 E-\mathcal{H}(B+v)=-C.
\end{aligned}
\end{align}
According to the SVT decomposition \eqref{eq:metric_general_form_svt}, in this ``comoving-type'' gauge we can then parametrize the spatial metric nonlinearly as
\begin{align}
\begin{aligned}
    h_{ij}(\tau,\mathbf{x})
    =
    e^{2\mathcal{R}(\tau,\mathbf{x})}\,\big[\exp(\gamma(\tau,\mathbf{x}))\big]_{ij} \simeq\, \delta_{ij} (1+2\mathcal{R})+ \gamma_{ij}+O(2)\,,
    \qquad
    \gamma^i{}_i=0,
    \qquad
    \partial_i\gamma^i{}_j=0,
\end{aligned}
\end{align}
where $\gamma_{ij}$ contains the transverse-traceless tensor modes, and the matrix exponential is in the Taylor sense.
This form is convenient because $\det\!\big(\exp(\gamma)\big)=1$, so the scalar volume perturbation is entirely carried by $e^{2\mathcal{R}}$.
In particular we can compute $\mathcal{R}$ in any gauge and then interpret it geometrically through the exponential factor in $h_{ij}$ when working on comoving hypersurfaces.






