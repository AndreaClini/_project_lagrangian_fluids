%-------------------------------------------------------------------------
%======================================================================
\section{Superhorizon conservation of curvature perturbations from phonon EoM (in progress)}
%======================================================================
%---------------------------------------------------------------------------
A distinctive feature of \emph{adiabatic} perturbations is that they are conserved in the superhorizon limit.
We would like to show this explicitly from the EoM of phonons.

From \eqref{eq:peculiar_velocity_fluid_vs_phonons} we identify the comoving curvature perturbations \eqref{eq:comoving_curvature_perturbations_gauge_inv_and_newtonian} in terms of metric newtonian potentials and phonon potential \eqref{eq:phonon_potential} as
\begin{equation}
    \mathcal{R} \overset{\text{Newt.}}{=} \phi -\mathcal{H}v = \phi + \mathcal{H}\,\dot{\pi}_L\,.
\end{equation}
%
In Newtonian gauge \eqref{eq:newtonian_gauge_metric}, the quadratic lagrangian becomes 
{\small
\begin{align}
\begin{aligned}
    S^{(2)}=
    &\int d^4x\,\, a^4\,\bigg\{
    \tfrac12 (yF_y-bF_b) (\dot{\pi}^i)^2+ \tfrac12 b^2 F_{bb} (\partial_j \pi^j)^2\bigg\} -\tfrac12 a^4 y^2F_{yy} (\dot{\pi}^0)^2\\
    &\quad + a^4\,\,\bigg\{-3\phi(\partial_j \pi^j)\left(\bar{b}F_b - \bar{b}^2 F_{bb}\right) 
    -\psi(\partial_j \pi^j)\!\left(-2\bar{b}F_b + \bar{b}\bar{y}F_{by}\right) \bigg\}\,,
\end{aligned}
\end{align}}
where
{\small
\begin{align}
\begin{aligned}
    -\tfrac12 a^4y^2F_{yy}\big(\dot{\pi}^0\big)^2
    &=-\tfrac{1}{2}\tfrac{a^4}{y^2F_{yy}}\bigg[(byF_{by}-yF_y) (\partial_j\pi^j)+\bigg(\tfrac{1}{a^4}\!\!\int(\partial_j \pi^j)\tfrac{d}{d\tau}(a^4yF_y)\bigg)\bigg]^2\\
    &\quad+\tfrac{a^4}{y^2F_{yy}}\Big[\big(2{y}F_y - byF_{by}\big) 6\phi+ \left({y}^2 F_{yy} - {y}F_y\right) 2\psi\Big]\\
    &\qquad\qquad\cdot \bigg[(byF_{by}-yF_y) (\partial_j\pi^j)+\bigg(\tfrac{1}{a^4}\!\!\int(\partial_j \pi^j)\tfrac{d}{d\tau}(a^4yF_y)\bigg)\bigg]\,.
\end{aligned}
\end{align}}
Taking $\psi=\phi$ at \emph{late times} and in the large scale limit, and simplifying terms we get\todotag{write}
\begin{align}
    ...
\end{align}
The equation of motion for phonons will be something like
\begin{align}
    \ddot{\pi}_L=-\underbrace{c_s^2 k^2\pi_L}_{\to0}+\text{friction}
\end{align}
giving that $\ddot{\pi}_L\to 0$ as $k\to0$.
For the comoving curvature then
\begin{align}
    \dot{\mathcal{R}}=\dot{\phi}+\dot{\mathcal{H}}\,\dot{\pi}_L+\mathcal{H}\,\ddot{\pi}_L
\end{align}
and, upon using the 0i Einstein equation for $\dot{\phi}+\mathcal{H}\phi=...$, the vanishing of $\ddot{\pi}$ from the EoM of phonons might be the fastest way to predict that $\mathcal{R}$ is conserved on superhorizon scales.\todotag{Finish this argument}
