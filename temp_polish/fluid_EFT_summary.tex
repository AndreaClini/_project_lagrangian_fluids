%---------------------------------------------------------------------
%====================================================================
\section{Summary of Fluid EFT}
%====================================================================
%-----------------------------------------------------------------
%
In this section we summarize the conventions and formulas of the fluid EFT for our convenience while drafting, regardless of what will actually appear in the main text.\notetag{This sec will be shortened at the end.}
First we present the fomrmulas valid at any order and then the perturbative expansion.


%====================================================================
\subsection{Effective larangian \& thermodynamic interpretation}
%====================================================================
%
The mapping from the spacetime manifold $\mathcal{M}^{(4)}$ to the internal fluid space $\mathcal{F}^{(3)}$ is given by four scalar fields\footnote{Note we use $\varphi$ for fluid coordinates, not to confuse them with Bardeen or Newtonian potentials.}  
{\small\begin{align}\label{eq:internal_coord_fluid}
    \varphi^{\alpha}:(\mathcal{M},g)\to \mathbb{R}\times \mathcal{F},\quad \alpha=0,1,2,3.
\end{align}}
Note we are free to choose the coordinates in the 3-dimensional internal space and to shift $\varphi^0$ by a constant without using any gauge freedom of the spacetime manifold itself.
In particular, we might choose $\varphi^\alpha(x)=x^\alpha$ at some fixed initial time $x^0=\tau_{in}$.

The 4-velocity of the fluid is identified by requiring the spatial coordinates of the fluid be indeed comoving with it, that is $\partial_\mu \varphi^i=0$ for $i=1,2,3$.
After normalizing the 4-velocity we get
\begin{align}\label{eq:definition_4_velocity}
    u^\mu=& \frac{1}{\sqrt{-g}\, 3!\,b} \epsilon^{\mu \alpha \beta \gamma} \epsilon_{ijk} \, \partial_\alpha \varphi^i \partial_\beta \varphi^j \partial_\gamma \varphi^k\, \propto\,\,  \star(d\varphi^1\wedge d\varphi^2\wedge d \varphi^3), \quad u_\mu u^\mu=-1\,.
\end{align}
We can identify 2 adimensional quantities
\begin{align}\label{eq:definition_b_y}
    b=\sqrt{\det B}\quad \text{for}\quad
    B^{ij}= g^{\mu\nu} \partial_\mu \varphi^i \partial_\nu \varphi^j,\quad y= u^\mu \partial_\mu \varphi^0, \qquad [b]=[y]=[u^\mu]=E^0,
\end{align}
that respect Lorentz invariance in the spacetime manifold and the internal symmetries of the fluid, namely 
{\small\begin{align}\label{eq:symmetries_fluid_internal}
    &\varphi^i \mapsto \tilde{\varphi}^i\big(\{\varphi^j\}_j\big)\,\,\text{with}\,\, \det\Big(\frac{\partial \tilde{\varphi}^i}{\partial\varphi^j}\Big)=1\,\,\quad \text{(volume preserving 3d diffeo invariance)},\\
    &\varphi^0 \mapsto \varphi^0 + c\quad\qquad\quad\qquad\qquad\qquad\qquad \text{(shift symmetry in $\varphi^0$)}.
\end{align}}
Up to a \emph{dimensional} scale $\Lambda$ these variables are naturally identified with the number density $b\equiv n$ and the temperature $y\equiv T$ in the fluid EFT as explained in the next section.

The lagrangian is a generic function of the invariants
\begin{align}\label{eq:eft_action_exact}
    S=\int d^4x \sqrt{-g} F(b,y)\,.
\end{align}
The conserved Noether currents are the number density current (from volume-preserving diffeo invariance in the $\varphi^i$) and the entropy current\footnote{\label{fn:thermodynamic_interpretation}See next subsection for the explanation of this thermodynamic interpretation.} (from shift symmetry in $\varphi^0$)
\begin{align}\label{eq:noether_currents_nonbaro}
    J_n^\mu = b\, u^\mu\,,\qquad J_s^\mu = F_y \,u^\mu\,.
\end{align}
Varying the action wrt the metric, the energy-momentum tensor takes the perfect fluid form\footnote{Subscripts mean $X_b=\partial_b X_{\mid y}$ and $X_y=\partial_y X_{\mid b}$ i.e. main variables $(b,y)$, unless otherwise stated.}
\begin{align}\label{eq:em_tensor_nonbaro}
    T_{\mu\nu} = u_\mu u_\nu (yF_y - bF_b) + g_{\mu\nu} (F-bF_b) = u_\mu u_\nu(\rho+p) + g_{\mu\nu}\, p\,.
\end{align}
Energy density, pressure and equation of state parameter are then identified as
\begin{align}\label{eq:thermo_rho_p_nonbaro}
    \rho=-F+yF_y,\quad p=F-bF_b,\quad w=\frac{F-bF_b}{yF_y-F}.
\end{align}
The entropy density $s$, comoving entropy $\sigma$, temperature $T$, chemical potential $\mu$ and number density $n$ are identified as\footnotemark[\getrefnumber{fn:thermodynamic_interpretation}]
\begin{align}\label{eq:thermo_other_variables_nonbaro}
    s=F_y,\quad \sigma:=\frac{s}{n}=\frac{F_y}{b},\quad T=y,\quad \mu:=F_b, \quad n=b, \quad F=-\rho+Ts=-\mathfrak{a}.
\end{align}
Notably this identifies $F(y,b)$ with minus the Helmoltz free-energy density {\small $\mathfrak{a}=\frac{U-TS}{V}$.}
Adiabatic sound speed $c_a^2$ and non-adiabatic pressure coefficient $\Gamma$ are then by definition
{\small\begin{align}\label{eq:adiabatic_cs_nonadiabatic_pressure}
\begin{aligned}
    &c_a^2=\frac{\partial p}{\partial \rho}_{|\sigma}=\frac{-y^2F_{yy}b^2F_{bb}+(yF_y-bF_b)^2}{y^2F_{yy}(yF_y-bF_b)}, \\
    &\Gamma:=\frac{\partial p}{\partial \sigma}_{\mid\rho}= \frac{b^2F_{bb}y^2F_{yy}-(yF_y-byF_{by})(bF_b-byF_{yb})}{\tfrac{1}{by}y^2F_{yy}(yF_y-bF_b)}.
\end{aligned}
\end{align}}
Using $F=-\mathfrak{a}$ and $s=F_y=-\partial_T\mathfrak{a}$, we rewrite the adiabatic sound speed as
\begin{align}
    c_a^2:=
    &=\frac{-b^2F_{bb} F_{yy}+(F_y-bF_{yb})^2}{F_{yy}\,(yF_y-bF_b)} = \frac{n^2\,\partial_n^2\mathfrak{a}\,\partial_Ts + (s-n\partial_ns)^2}{\partial_Ts\,\,(\rho+p)}\,,
\end{align}
which makes it manifest that $c_a^2\geq0$ provided no ghosts $\rho+P>0$, the second principle of thermodynamics $\partial_Ts\geq 0$ and thermodynamic stability $\partial_n^2\mathfrak{a}\geq 0$ hold.

The adiabatic sound speed \eqref{eq:adiabatic_cs_nonadiabatic_pressure} always coincides with the coefficient $c_s^2$ of the gradient term in the quadratic action for phonons, after integrating out the non-dynamical entropy mode $\pi^0$,
\begin{align}
    c_s^2\equiv c_a^2\,,
\end{align}
which shows that absence of ghosts, thermodynamic stability and gradient instabilities are closely related.
For flat spacetimes this is proved in \eqref{eq:cs_equiv_cb_flat_spacetime} below.\todotag{prove it in curved space}

Finally, in the barotropic case we can still define $y=\partial_\mu\varphi^0u^\mu$, but $F=F(b)$ does not depend on $y$ and the above formulas reduce to
{\small\begin{align}\label{eq:thermo_variables_barotropic}
    \rho=-F,\quad p=F-bF_b,\quad w=\frac{bF_b-F}{F},\quad n=b,\quad \mu=F_b,\quad T=y\,\quad s\equiv0,\quad \frac{d p}{d \rho}=c_a^2=\frac{b^2F_{bb}}{bF_b}.
\end{align}}



%-----------------------------------------------------------------------
\subsubsection{Justification of the thermodynamic interpretation}
%----------------------------------------------------------------------
In this section we justify the thermodynamic interpretation of the fluid EFT variables and formulas.
As argued above, the identification of the 4-velocity \eqref{eq:definition_4_velocity} follows unavoidably from the requirement the spatial coordinates $\varphi^i$ be indeed comoving with the fluid.

The next point is the unequivocal identification of $b$ with the number density $n$.
Indeed, in order for the mapping \eqref{eq:internal_coord_fluid} from spacetime to the internal fluid space to make sense, it is necessary that the restriction $\varphi^{I}:\Sigma_\tau^{(3)}\to \mathcal{F}^{(3)}$ is a diffeomorphism for any spacelike hypersurface $\Sigma_\tau^{(3)}\subset \mathcal{M}^{(4)}$.
The fluid coordinates $\{\varphi^i\}_{i=1,2,3}$ can thus be used as coordinates for the hypersurface $\Sigma_\tau^{(3)}$, and $g^{\mu\nu}\partial_\nu\varphi^i$ are 3 independent vectors spanning its tangent space.
In these coordinates, the metric $g_{|\Sigma}$ induced on spatial slices reads
\begin{align}
    (g^\varphi_{\Sigma})_{ij} = \partial_\mu \varphi^i\partial_\nu \varphi^j g^{\mu\nu}\equiv B^{ij}.
\end{align}
The volume factor $\sqrt{g^\varphi_\Sigma}=\sqrt{\det B}=b$, i.e. the jacobian of the transformation from spatial coordinates to internal fluid coordinates, is then precisely the density of the fluid elements wrt the original spacetime coordinates $b\equiv n$.

To complete the thermodynamic interpretation we need some final imput, which comes from matching the EFT energy-momentum tensor to that of a perfect fluid \eqref{eq:em_tensor_nonbaro}, thereby identifying the energy density and pressure \eqref{eq:thermo_rho_p_nonbaro}.
Finally we simply impose the first law of thermodynamics
\begin{align}\label{eq:first_law_thermo}
    T s+ \mu n = \rho+p \overset{!}{=} yF_y-bF_b.
\end{align}
Since we have already shown $n=b$, we are lead to identify $\mu=-F_b$, and $s=F_y$ and $y=T$.
This is consistent with all the thermodynamical relations, for example we find back $\mu$ imposing entropy is conserved
\begin{align}\label{eq:chemical_potential_and_entropy_consistency}
    \begin{array}{c}
     0\overset{!}{=}ds = F_{yb} db + F_{yy}dy\\[3pt]
    \displaystyle\Rightarrow\,\, \frac{dy}{db}_{|s}=-\frac{F_{yb}}{F_{yy}}
    \end{array}
    \quad\Rightarrow\quad \mu:=\frac{\partial\rho}{\partial n}_{|s}= \underbrace{\frac{\partial\rho}{\partial b}_{|y}}_{=-F_b+yF_{by}}+ \underbrace{\frac{\partial\rho}{\partial y}_{|b}}_{=yF_{yy}}\,\,\frac{dy}{db}_{|s}=-F_b.
\end{align}
In fact, after enforcing $n=b$ and the expressions for $\rho$ and $p$, equating $s=F_y,\,T=y$ is the \emph{only} consistent identification.
Indeed taking $n=b$ and $y=s$ predicts wrong conjugate variables and fails the 1st law of thermodynamics \eqref{eq:first_law_thermo} since
\begin{align}\begin{aligned}
    &n=b,\,\,s= y \,\,\Rightarrow\,\, \begin{cases}
        \mu:=\frac{\partial\rho}{\partial n}_{|s}= \frac{\partial\rho}{\partial b}_{|y}=-F_b+yF_{by}\\
         T:=\frac{\partial\rho}{\partial s}_{|n}= \frac{\partial\rho}{\partial y}_{|b}=yF_{yy}
    \end{cases}\\[4pt]
    &\,\,\Rightarrow\,\, \mu n +T s= -bF_b+byF_{by}+y^2F_{yy}\neq yF_y-bF_b=\rho+p.
\end{aligned}\end{align}




%==============================================================
\subsection{Quadratic expansion \& matching phonons to fluid perturbations}
%============================================================
%
The above formulas are exact to any order in the fluid coordinates.o
We now expand to quadratic order around unperturbed\footnote{Beware this choice of fluid coordinates does not use any gauge freedom of the spacetime manifold!} fluid coordinates $\bar{\varphi}^\alpha\equiv x^\alpha$ and FLRW spacetime.
The general perturbed FLRW metric is written wrt conformal time $\tau$ as
{\small \begin{align}\label{eq:general_perturbed_FLRW_metric}
\begin{aligned}
    &g_{\mu\nu}=a^2\Big(\eta_{\mu\nu}+h_{\mu\nu}\Big)\,,\quad g^{\mu\nu}=a^{-2}\Big(\eta^{\mu\nu}-h^{\mu\nu}\Big)\quad \text{with} \quad h^{\mu\nu}=\eta^{\mu\alpha}\eta^{\nu\beta}h_{\alpha\beta},\\
    &\quad \sqrt{|g|}= a^4\big(1+\tfrac12 h^{\mu\nu}\eta_{\mu\nu} +O(h^2)\big)\ = a^4\Big[1+\tfrac12 (h^{ii}-h^{00}) +O(h^2)\Big]\,.
\end{aligned}
\end{align}}
The fluid variables are written for simplicity with rescaling factors $\gamma$ and $\lambda$ that will be used to reabsord unphysical constants in the compuations
{\small\begin{align}\label{eq:perturbed_fluid_coordinates}
    \varphi^0=\gamma\Big(\tau+\pi^0\Big), \quad \varphi^i=\lambda\Big(x^i+\pi^i\Big)\,, \quad [\varphi^\alpha]=[x^\alpha]=[\pi^\alpha]=E^{-1},\quad [\gamma]=[\lambda]=E^0.
\end{align}}
We have the expansions for the invariants $b,y$ and the 4-velocity
{\small\begin{align}\label{eq:expansions_b_y_u}
\begin{aligned}
    &b= \bar{b}(1+\delta b),\quad \bar{b}= \frac{\lambda^3}{a^3},\quad
    \delta b^{(1)} = -\tfrac12 h^{ii} +(\partial_j \pi^j),\\[3pt]
    &\delta b^{(2)} = - h^{0 i} \dot{\pi}^i - \tfrac12  h^{ii} (\partial_j \pi^j) - \tfrac12 (\dot{\pi}^i)^2 + \tfrac12 (\partial_j \pi^j)^2 -\tfrac12 \partial_i\pi^j\partial_j\pi^i\,,
    \\[3pt]
    &y= \bar{y}(1+\delta y),\quad \bar{y}= \frac{\gamma}{a},\quad \delta y^{(1)} = \tfrac12 h^{00}+ \dot{\pi}^0,\\[3pt]
    &\delta y^{(2)} = \tfrac12 h^{00}\dot{\pi}^0+ h^{0 i} \dot{\pi}^i  + \tfrac12 (\dot{\pi}^i)^2 -\partial_j\pi^0\dot{\pi}^j\,\\[3pt]
    &au^\mu=\delta_0^\mu \bigg(1 \!+\! \Big[\tfrac12 h^{00} -(\partial_j \pi^j)\Big]\! +\! \Big[ h^{0 i} \dot{\pi}^i -\tfrac12 h^{00} (\partial_j \pi^j) + \tfrac12 (\dot{\pi}^i)^2 + \tfrac12 (\partial_j \pi^j)^2 +\tfrac12 \partial_i\pi^j\partial_j\pi^i\Big]\bigg)
    \\
    &\quad\qquad+\tfrac{1}{2}\left[1 + \tfrac12 h^{00}-(\partial_\ell\pi^\ell)\right]
     \epsilon_{i j k}\,\epsilon^{\mu \alpha j k}\,\partial_\alpha\pi^i 
     +\frac{1}{2} \epsilon_{ijk}\, \epsilon^{\mu i \beta\gamma}\partial_\beta\pi^j\,\partial_\gamma\pi^k
    + O(\pi^3, h^2)
\end{aligned}\end{align}}
%
In turn we recast the fluctuations in density $\rho=-F+yF_y$ and pressure $p=F-bF_b$ as
\begin{align}\label{eq:rho_p_fluctuations_via_b_y}
\begin{aligned}
    \delta\rho &= y^2F_{yy}\delta y + (-bF_b+byF_{by})\delta b
    \\
    &= (y^2F_{yy})\left(\tfrac12 h^{00}+ \dot{\pi}^0\right) + (-bF_b+byF_{by})\left(-\tfrac12 h^{ii} +(\partial_j \pi^j)\right) + O(2)
    \\[3pt]
    \delta p &= (yF_y-byF_{by})\delta y - (b^2 F_{bb})\delta b
    \\
    &= (yF_y-byF_{by})\left(\tfrac12 h^{00}+ \dot{\pi}^0\right) - (b^2 F_{bb})\left(-\tfrac12 h^{ii} +(\partial_j \pi^j)\right)+ O(2)\,.
\end{aligned}
\end{align}
We also introduce the phonon potential to rewrite longitudinal (scalar) phonons
\begin{align}\label{eq:phonon_potential}
    \pi_L=\nabla^{-2}\partial_\ell\pi^\ell, \quad \pi^\ell_{\text{||}}=\partial_\ell \pi_L,\quad  [\pi_L]=E^{-2}.
\end{align}
Matching the velocity expansion \eqref{eq:expansions_b_y_u} in the fluid EFT to the general expression \eqref{eq:velocity_decomposition} in pertrubed FLRW spacetime $au^\mu =a(1+.., v^\ell)$, we identify the peculiar velocity and the velocity potential/divergence in terms of phonons as
\begin{align}\label{eq:peculiar_velocity_fluid_vs_phonons}
    v^\ell = \frac{1}{2}\epsilon_{i j k}\,\epsilon^{\ell \alpha j k}\,\partial_\alpha\pi^i = -\dot{\pi}^\ell\quad \Rightarrow \quad \theta=-\partial_\ell\dot{\pi}^\ell= -\nabla^2\dot{\pi}_L,\,\, \,\,v = -\dot{\pi}_L\,.
\end{align}



%==========================================================================
%\subsection{Quadratic action for phonons}
%==========================================================================
%
%=======================================================================
\subsection{Quadratic action for phonons}
%======================================================================
%
In this section we use the expansions \eqref{eq:expansions_b_y_u} to expand the action \eqref{eq:eft_action_exact} up to second order in phonon fields $\pi^\alpha \pi^\beta$ and mixed order in metric perturbations $h_{\mu\nu}\,\pi^\alpha$.
To get some intuition we first present the flat-spacetime case, since formulas are cleaner and  ultimately exact in the limit $\rho/ M_{Pl}^{4}\to0$.

%-----------------------------------------------------------------
\subsubsection{Quadratic action in flat spacetime}
%-------------------------------------------------------
%
We refer to the full computations in curved spacetime below not to repeat the same steps.
In flat spacetime $a=1$ and $h_{\mu\nu}=0$, only the first terms in the action \eqref{eq:final_quadratic_action_curved_spacetime} survive, and we can integrate by parts in both space and time the term $\partial_j \pi^0\dot{\pi}^j = -\dot{\pi}^0(\partial_j \pi^j)$ to get
{\small\begin{align}
\begin{aligned}
    S^{(2)}&= \int d^4x \frac{(yF_y-bF_b)}{2}\bigg\{(\dot{\pi}^i)^2 - (\partial_j\pi^j)^2\Big[-\frac{b^2F_{bb}}{(yF_y-bF_b)}\Big]\\
    &\qquad\qquad\qquad\qquad\qquad+(\dot{\pi}^0)^2 \frac{y^2F_{yy}}{(yF_y-bF_b)}
    - 2\dot{\pi}^0 (\partial_j \pi^j) \frac{(yF_y-byF_{by})}{(yF_y-bF_b)}
    \bigg\}\\
    &=\int d^4x \frac{(\rho+p)}{2}\bigg\{(\dot{\pi}^i)^2 - \mathcal{C} (\partial_j\pi^j)^2 
    + (\dot{\pi}^0)^2 \mathcal{A}
    - 2\mathcal{M}\dot{\pi}^0 (\partial_j \pi^j) 
    \bigg\}
\end{aligned}
\end{align}}
for $\rho= -F + y F_y$, $p= F - b F_b$ and 
{\small\begin{align}
\begin{aligned}
    &\mathcal{C} = -\frac{b^2F_{bb}}{(yF_y-bF_b)},\quad 
    \mathcal{A} = \frac{y^2F_{yy}}{(yF_y-bF_b)} = \frac{T\, \frac{\partial\rho }{\partial T}}{\rho+p},\quad
    \mathcal{M} = \frac{(yF_y-byF_{by})}{(yF_y-bF_b)}.
\end{aligned}
\end{align}}
The entropy field $\pi^0$ is thus non-dynamical and the resulting constraint equation gives
\begin{align}
    \dot{\pi}^0 = \frac{\mathcal{M}}{\mathcal{A}} (\partial_j \pi^j) + \text{const.}
\end{align}
The constant is not physical and is reabsorbed in the factor $\gamma$ of the original definition $\varphi^0:=\gamma(t+\pi^0)$.
After solving the constraint, the action for the coordinate $\pi^i$ becomes
{\small
\begin{align}
    S^{(2)}&= \int d^4x \frac{(\rho+p)}{2}\Big[(\dot{\pi}^i)^2 - c_s^2 (\partial_j\pi^j)^2 \Big], \quad \text{for}\quad c_s^2 = \mathcal{C} + \frac{\mathcal{M}^2}{\mathcal{A}}.
\end{align}
}
We note the entropy contribution to the speed of sound is always positive in \emph{physical} situations.
Indeed $\mathcal{M}^2$ is a true square and, recalling $y\equiv T$, we have
\begin{equation}
    \mathcal{A}= \frac{T\, \frac{\partial\rho }{\partial T}}{\rho+p}\geq 0 \quad\text{provided}\quad
    \begin{cases}
        \rho+p\geq 0 \quad(\text{no ghosts}),\\
        \frac{\partial\rho }{\partial T} \geq 0 \quad(\text{positive heat capacity}).
    \end{cases} 
\end{equation}
Finally we confirm the speed of sound $c_s^2$, defined as the coefficient of the gradient term for the fluctuations above, always matches the thermodynamical definition of \emph{adiabatic} sound speed \eqref{eq:adiabatic_cs_nonadiabatic_pressure}, namely
\begin{align}\label{eq:cs_equiv_ca_flat_spacetime}
    c_a^2:= \frac{\partial p}{\partial \rho}_{\mid \sigma} \overset{\text{proved before}}{=}\frac{-b^2F_{bb} F_{yy}+(F_y-bF_{yb})^2}{F_{yy}\,(yF_y-bF_b)} \overset{\text{easy check}}{\equiv} c_s^2.
\end{align}
This is expected: longitudinal phonons, i.e. sound waves, do propagate with the speed of sound!





%-----------------------------------------------------------------
\subsubsection{Quadratic action in curved spacetime {\color{red}(in progress)}}
%--------------------------------------------------------------
%
{\color{red}[I am reporting the full computation here since I am not yet satisfied with the final form. It will eventually be moved to the computation appendix.]}\todotag{Read \& feel the struggle}

\noindent
We now consider a general perturbed FRW metric written as in \eqref{eq:general_perturbed_FLRW_metric} and expand the action up to quadratic order  $\pi^\alpha \pi^\beta,\,h_{\mu\nu}\,\pi^\alpha$ for fluid perturbations
{\small\begin{align}
\begin{aligned}
    S&=\int d^4x \sqrt{-g} F(b,y) = \int d^4x\,a^4 \big[1+\tfrac12 (h^{ii}-h^{00}) \big] F(\bar{b}(1+\delta b), \bar{y}(1+\delta y))
    \\
    &= \int \!\!\!d^4\!x\, a^4 \!\big[1+\tfrac12 (h^{ii}-h^{00}) \big]
    \Bigg[ F(\bar{b}, \bar{y}) + (\bar{b}F_b)  (\delta b^{(1)} + \delta b^{(2)}) + (\bar{y}F_y) (\delta y^{(1)} + \delta y^{(2)})\\
    &\,\,\qquad\qquad\qquad\qquad\qquad\qquad+ \tfrac12 (\bar{b}^2 F_{bb})(\delta b^{(1)})^2 + \tfrac12 (\bar{y}^2 F_{yy}) (\delta y^{(1)})^2
    + (\bar{b}\bar{y} F_{by}) \delta b^{(1)} \delta y^{(1)} \Bigg]
\end{aligned}
\end{align}}
Inserting the expansions \eqref{eq:expansions_b_y_u} for $b$ and $y$, the quadratic part of the action is then
{\small\begin{align}
\begin{aligned}
    S^{(2)}&=\int d^4x\,\, a^4 \bigg\{
        (\bar{b}F_b) (h^{ii}-h^{00}) \Big[(\partial_j \pi^j) - \tfrac12 h^{ii}\Big]
        + (\bar{y}F_y) (h^{ii}-h^{00}) \Big[\tfrac12 h^{00}+ \dot{\pi}^0\Big]\\
    &\quad + (\bar{b}F_b) \Big[ - h^{0 i} \dot{\pi}^i - \tfrac12  h^{ii} (\partial_j \pi^j) - \tfrac12 (\dot{\pi}^i)^2 + \tfrac12 (\partial_j \pi^j)^2 -\tfrac12 \partial_i\pi^j\partial_j\pi^i\Big]
    \\
    &\quad + (\bar{y}F_y) \Big[ \tfrac12 h^{00}\dot{\pi}^0+ h^{0 i} \dot{\pi}^i + \tfrac12 (\dot{\pi}^i)^2 -\partial_j\pi^0\dot{\pi}^j\Big]
    + \tfrac12 (\bar{b}^2 F_{bb})\Big[ (\partial_j \pi^j) - \tfrac12 h^{ii}\Big]^2
    \\
    &\quad + \tfrac12 (\bar{y}^2 F_{yy})\Big[\tfrac12 h^{00}+ \dot{\pi}^0\Big]^2
    + (\bar{b}\bar{y} F_{by}) \Big[(\partial_j \pi^j) - \tfrac12 h^{ii}\Big]\Big[\tfrac12 h^{00}+ \dot{\pi}^0\Big]
        \bigg\}.
\end{aligned}
\end{align}}
Dropping  further second orders in metric perturbations and {\color{blue} integrating by parts in space}\footnote{\label{fn:integrating_parts_space_backround}Space derivatives yield no further terms since the background isspatially homogenous.}, we get
{\small\begin{align}
\begin{aligned}
    S^{(2)}=
    \int d^4x\,\, a^4 \bigg\{&
        (\bar{b}F_b) (h^{ii}-h^{00}) (\partial_j \pi^j)
        + (\bar{y}F_y) (h^{ii}-h^{00}) \dot{\pi}^0 \\
    &\quad + (\bar{b}F_b) \Big[ - h^{0 i} \dot{\pi}^i - \tfrac12  h^{ii} (\partial_j \pi^j) - \tfrac12 (\dot{\pi}^i)^2 + \bluecancel{\tfrac12 (\partial_j \pi^j)^2} -\bluecancel{\tfrac12 \partial_i\pi^j\partial_j\pi^i}\Big]
    \\
    &\quad + (\bar{y}F_y) \Big[ \tfrac12 h^{00}\dot{\pi}^0+ h^{0 i} \dot{\pi}^i + \tfrac12 (\dot{\pi}^i)^2 -\partial_j\pi^0\dot{\pi}^j\Big]
    \\
    &\quad + \tfrac12 (\bar{b}^2 F_{bb})\Big[ (\partial_j \pi^j)^2 - h^{ii}(\partial_j \pi^j)\Big]
    + \tfrac12 (\bar{y}^2 F_{yy})\Big[h^{00} \dot{\pi}^0+ (\dot{\pi}^0)^2\Big]\\
    &\quad+ (\bar{b}\bar{y} F_{by}) \Big[ \dot{\pi}^0(\partial_j \pi^j) - \tfrac12 h^{ii}\dot{\pi}^0 + \tfrac12 h^{00} (\partial_j\pi^j)\Big]
        \bigg\}.
\end{aligned}
\end{align}}
Collecting terms $\pi^\alpha\pi^\beta$ and $\pi^\alpha h_{\mu\nu}$, the final expression for the quadratic action is then
{\small\begin{align}\label{eq:final_quadratic_action_curved_spacetime}
\begin{aligned}
    S^{(2)}=
    &\int d^4x\,\,a^4\,\,\bigg\{
    \tfrac12 (yF_y-bF_b) (\dot{\pi}^i)^2+\tfrac12 b^2 F_{bb} (\partial_j \pi^j)^2
    +\tfrac12 y^2 F_{yy} (\dot{\pi}^0)^2 \\
    &\qquad\qquad\qquad+ (b y F_{by}) (\dot{\pi}^0 \partial_j \pi^j)-(yF_y) (\partial_j \pi^0\dot{\pi}^j)
    \bigg\}\\
    &\quad + a^4\,\,\bigg\{
    \tfrac12\!\left(\bar{b}F_b - \bar{b}^2 F_{bb}\right) h^{ii}(\partial_j \pi^j)
    + \tfrac12\!\left(-2\bar{b}F_b + \bar{b}\bar{y}F_{by}\right) h^{00}(\partial_j \pi^j)\\
    &\qquad\qquad+ \left(\bar{y}F_y - \bar{b}F_b\right) h^{0j}\,\dot{\pi}^j + \tfrac12\!\left(2\bar{y}F_y - \bar{b}\bar{y}F_{by}\right) h^{ii}\dot{\pi}^0
    + \tfrac12\!\left(\bar{y}^2 F_{yy} - \bar{y}F_y\right) h^{00}\dot{\pi}^0
    \bigg\}\,.
\end{aligned}
\end{align}}
Note that only \emph{longitudinal} modes $\pi^j_{||}\propto\, \partial_j\pi^j$ are dynamical.
Transverse modes $\pi^j_{\perp}$ simply mix with vector metric perturbations $h^{0j}$ with resulting equation of motion
{\small\begin{align}\label{eq:eom_transverse_phonons}
    a^4\,\big(\bar{\rho}+\bar{p}\big)\,\big(\dot{\pi}_\perp^j + h^{0j}\,\big)=\text{const}\,.
\end{align}}
This is expected since vorticity in the fluid sources vector modes in the metric and viceversa.

In the following we thus restrict to longitudinal modes $\pi_{||}^j$.
Gathering $\rho+p=yF_y-bF_b$ outside the brackets to highlight ghosts, we rewrite the action \eqref{eq:final_quadratic_action_curved_spacetime} as\todotag{Probably should wait and integrate bp first}
{\small\begin{align}
\begin{aligned}
    S^{(2)}= &\int d^4x\,\,a^4\, (\rho+p)\,\bigg\{
    \tfrac12 (\dot{\pi}^i)^2-\tfrac12 \underbrace{\tfrac{-b^2 F_{bb}}{(yF_y-bF_b)}}_{=\mathcal{C}} (\partial_j \pi^j)^2
    +\tfrac12 \underbrace{\tfrac{y^2 F_{yy}}{(yF_y-bF_b)}}_{=\mathcal{A}} (\dot{\pi}^0)^2 \\
    &\quad\qquad\qquad\qquad\qquad+ \tfrac{(b y F_{by})}{(yF_y-bF_b)} (\dot{\pi}^0 \partial_j \pi^j)-\tfrac{(yF_y)}{(yF_y-bF_b)} (\partial_j \pi^0\dot{\pi}^j)
    \bigg\}\\
    & \,\,+ a^4\, (\rho+p)\, \bigg\{
    \tfrac12\!\left(\tfrac{\bar{b}F_b - \bar{b}^2 F_{bb}}{(yF_y-bF_b)}\right) h^{ii}(\partial_j \pi^j)
    + \tfrac12\!\left(\tfrac{-2\bar{b}F_b + \bar{b}\bar{y}F_{by}}{(yF_y-bF_b)}\right) h^{00}(\partial_j \pi^j)\\
    &\quad\qquad\qquad+ \left(\tfrac{\bar{y}F_y - \bar{b}F_b}{(yF_y-bF_b)}\right) h^{0 j}\,\dot{\pi}^j + \tfrac12\!\left(\tfrac{2\bar{y}F_y - \bar{b}\bar{y}F_{by}}{(yF_y-bF_b)}\right) h^{ii}\dot{\pi}^0
    + \tfrac12\!\left(\tfrac{\bar{y}^2 F_{yy} - \bar{y}F_y}{(yF_y-bF_b)}\right) h^{00}\dot{\pi}^0
    \bigg\}.
\end{aligned}
\end{align}}
%
Integrating by parts in both time and space the term $\partial_j \pi^0\dot{\pi}^j$ we get\footnotemark[\getrefnumber{fn:integrating_parts_space_backround}]
{\small
\begin{align}
\begin{aligned}
    S^{(2)}&=\int d^4x\, -(\partial_j \pi^j\, \pi^{0})\tfrac{d}{d\tau}(a^4yF_y)\\
    &+a^4\,\bigg\{
    \tfrac12 (yF_y-bF_b) (\dot{\pi}^i)^2+ \tfrac12 b^2 F_{bb} (\partial_j \pi^j)^2 +\tfrac12 y^2 F_{yy} (\dot{\pi}^0)^2+(b y F_{by}-yF_y) (\dot{\pi}^0 \partial_j \pi^j)\bigg\}\\
    &+ a^4\,\,\bigg\{
    \tfrac12\!\left(\bar{b}F_b - \bar{b}^2 F_{bb}\right) h^{ii}(\partial_j \pi^j)
    + \tfrac12\!\left(-2\bar{b}F_b + \bar{b}\bar{y}F_{by}\right) h^{00}(\partial_j \pi^j)+\left(\bar{y}F_y - \bar{b}F_b\right) h^{0 j}\,\dot{\pi}^j\\
    &\qquad\qquad +\tfrac12\!\left(2\bar{y}F_y - \bar{b}\bar{y}F_{by}\right) h^{ii}\dot{\pi}^0
    + \tfrac12\!\left(\bar{y}^2 F_{yy} - \bar{y}F_y\right) h^{00}\dot{\pi}^0
    \bigg\}.
\end{aligned}
\end{align}}
%
The constraint equation for $\pi^0$ now reads
{\small\begin{align}
\begin{aligned}
    \frac{d}{d\tau}&a^4\bigg[y^2F_{yy}\dot{\pi}^0+(byF_{by}-yF_y) (\partial_j\pi^j)+\tfrac12\!\big(2{y}F_y - byF_{by}\big) h^{ii}
    + \tfrac12\!\left({y}^2 F_{yy} - {y}F_y\right) h^{00}\bigg]
    \\
    &=-(\partial_j \pi^j)\tfrac{d}{d\tau}(a^4yF_y)\,.
\end{aligned}
\end{align}}
Resolving the constraint and plugging it back into the action we get
{\small\begin{align}
\begin{aligned}
    S^{(2)}&=
    \int d^4x\,\, a^4\,\bigg\{
    \tfrac12 (yF_y-bF_b) (\dot{\pi}^i)^2+ \tfrac12 b^2 F_{bb} (\partial_j \pi^j)^2\bigg\} -\tfrac12 a^4 y^2F_{yy} (\dot{\pi}^0)^2\\
    & \,\,\,+ a^4\bigg\{\tfrac12\!\left(\bar{b}F_b - \bar{b}^2 F_{bb}\right) h^{ii}(\partial_j \pi^j)
    + \tfrac12\!\left(\bar{b}\bar{y}F_{by}-2\bar{b}F_b\right) h^{00}(\partial_j \pi^j)+\left(\bar{y}F_y - \bar{b}F_b\right) h^{0j}\,\dot{\pi}^j \bigg\},
\end{aligned}
\end{align}}
where now\todotag{2nd form likely better}
{\small
\begin{align}
\begin{aligned}
    a^4y^2F_{yy}\dot{\pi}^0
    =&-a^4\bigg[(byF_{by}-yF_y) (\partial_j\pi^j)+\tfrac12\!\big(2{y}F_y - byF_{by}\big) h^{ii}
    + \tfrac12\!\left({y}^2 F_{yy} - {y}F_y\right) h^{00}\bigg]\\
    &-\int d\tau\, (\partial_j \pi^j)\tfrac{d}{d\tau}(a^4yF_y)\\
    &{\color{red}=-a^4\bigg[(byF_{by}-yF_y) (\partial_j\pi^j)+\bigg(\tfrac{1}{a^4}\!\!\int d\tau(\partial_j \pi^j)\tfrac{d}{d\tau}(a^4yF_y)\bigg)\bigg]}\\
    &{\color{red}\qquad-a^4\tfrac12\Big[\big(2{y}F_y - byF_{by}\big) h^{ii}+ \left({y}^2 F_{yy} - {y}F_y\right) h^{00}\Big]}.
\end{aligned}
\end{align}}
That is
{\small
\begin{align}
\begin{aligned}
    -\tfrac12&a^4y^2F_{yy}\big(\dot{\pi}^0\big)^2\\
    &=-\tfrac12 a^4 \tfrac{(byF_{by}-yF_y)^2}{y^2F_{yy}} (\partial_j\pi^j)^2\\
    &\quad-\tfrac{(byF_{by}-yF_y)}{y^2F_{yy}} (\partial_j\pi^j)\bigg(\int(\partial_\ell\pi^\ell)\tfrac{d}{d\tau}(a^4yF_y) \bigg)
    -\tfrac12 \tfrac{a^{-4}}{y^2F_{yy}}\bigg(\int(\partial_\ell\pi^\ell)\tfrac{d}{d\tau}(a^4yF_y)\bigg)^2
    \\
    &\quad-\tfrac12 a^4\tfrac{\big(2{y}F_y-byF_{by}\big)(byF_{by}-yF_y)}{y^2F_{yy}}h^{ii}(\partial_j\pi^j)
    -\tfrac12 a^4\tfrac{\left({y}^2 F_{yy} - {y}F_y\right)(byF_{by}-yF_y)}{y^2F_{yy}}h^{00}(\partial_j\pi^j)
    \\
    &\quad -\tfrac12 \tfrac{\big(2{y}F_y-byF_{by}\big)}{y^2F_{yy}}h^{ii}\bigg(\int\!d\tau(\partial_\ell\pi^\ell)\tfrac{d}{d\tau}(a^4yF_y)\bigg)
    -\tfrac12 \tfrac{\left({y}^2 F_{yy} - {y}F_y\right)}{y^2F_{yy}}h^{00}\bigg(\int\! d\tau(\partial_\ell\pi^\ell)\tfrac{d}{d\tau}(a^4yF_y)\bigg).
\end{aligned}
\end{align}}
Or equivalently keeping therms $\propto \, \nabla \pi+\int d\tau \nabla\pi$ together
{\small\color{red}
\begin{align}
\begin{aligned}
    -\tfrac12 a^4y^2F_{yy}\big(\dot{\pi}^0\big)^2
    &=-\tfrac{1}{2}\tfrac{a^4}{y^2F_{yy}}\bigg[(byF_{by}-yF_y) (\partial_j\pi^j)+\bigg(\tfrac{1}{a^4}\!\!\int\! d\tau\,(\partial_j \pi^j)\tfrac{d}{d\tau}(a^4yF_y)\bigg)\bigg]^2\\
    &\quad-\tfrac{a^4}{y^2F_{yy}}\Big[\big(2{y}F_y - byF_{by}\big) h^{ii}+ \left({y}^2 F_{yy} - {y}F_y\right) h^{00}\Big]\\
    &\qquad\qquad\cdot \bigg[(byF_{by}-yF_y) (\partial_j\pi^j)+\bigg(\tfrac{1}{a^4}\!\!\int\! d\tau\,(\partial_j \pi^j)\tfrac{d}{d\tau}(a^4yF_y)\bigg)\bigg]\,.
\end{aligned}
\end{align}}
%
Substituting back in the action we finally get\todotag{Rewrite w red exp, keeping $\nabla\pi+\int$ together}
{\small
\begin{align}
\begin{aligned}
    S^{(2)}=&\int d^4x\,\, a^4\,\frac{\rho+p}{2}\Bigg\{
    (\dot{\pi}^i)^2-\bigg[-\tfrac{b^2 F_{bb}}{(yF_y-bF_b)} +\tfrac{(byF_{by}-yF_y)^2}{y^2F_{yy}(yF_y-bF_b)}\bigg] (\partial_j \pi^j)^2 \\
    &\quad-2\tfrac{(byF_{by}-yF_y)}{y^2F_{yy}(yF_y-bF_b)} (\partial_j\pi^j)\bigg(\tfrac{1}{a^4}\int\!d\tau(\partial_\ell\pi^\ell)\tfrac{d}{d\tau}(a^4yF_y) \bigg)
    -\tfrac{1}{y^2F_{yy}(yF_y-bF_b)}\bigg(\tfrac{1}{a^4}\int(\partial_\ell\pi^\ell)\tfrac{d}{d\tau}(a^4yF_y)\bigg)^2\\
    &\quad +\left(\tfrac{\bar{b}F_b-\bar{b}^2 F_{bb}}{yF_y-bF_b} - \tfrac{\big(2{y}F_y-byF_{by}\big)(byF_{by}-yF_y)}{(yF_y-bF_b)y^2F_{yy}}\right) h^{ii}(\partial_j \pi^j)\\
    &\qquad\qquad\qquad+\!\left(\tfrac{\bar{b}\bar{y}F_{by}-2\bar{b}F_b}{yF_y-bF_b} - \tfrac{\left({y}^2 F_{yy} - {y}F_y\right)(byF_{by}-yF_y)}{(yF_y-bF_b)y^2F_{yy}}\right) h^{00}(\partial_j \pi^j)
    +2h^{0 j}\,\dot{\pi}^j\\
    &-\tfrac{\big(2{y}F_y-byF_{by}\big)}{y^2F_{yy}(yF_y-bF_b)}h^{ii}\bigg(\tfrac{1}{a^4}\int(\partial_\ell\pi^\ell)\tfrac{d}{d\tau}(a^4yF_y)\bigg)
    -\tfrac{\left({y}^2 F_{yy} - {y}F_y\right)}{y^2F_{yy}(yF_y-bF_b)}h^{00}\bigg(\tfrac{1}{a^4}\int(\partial_\ell\pi^\ell)\tfrac{d}{d\tau}(a^4yF_y)\bigg)\Bigg\}.
\end{aligned}
\end{align}}







%================================================================================
\subsection{Related computations (to be removed at the end)}
%================================================================================

Dear collaborators, I know you just want to see the results, but I need a safe place to keep my computations for several reasons: they might (often) be wrong, I want to improve or simplify them, we might disagree on the results, they remind me how to do the math the fastest way in case I need to redo them, etc.
I promise I will keep them hidden here without polutting the main text and will remove them at the end.


%-------------------------------------------------------------------------
\subsubsection{Thermo formulas adiabatic sound speed \& non-adiabatic pressure}
%--------------------------------------------------------------------------

The entropy per particle is $\sigma=F_y/b$, imposing it be preserved gives
\begin{align}
    0=d\sigma = \frac{F_{yy}}{b}dy+\Big(\frac{bF_{by}-F_y}{b^2}\Big)db\quad \Rightarrow\quad \frac{dy}{db}_{\mid\sigma}= \frac{F_y-b F_{yb}}{bF_{yy}}=\frac{y}{b}\frac{p_y}{\rho_y}.
\end{align}
The adiabatic sound speed is then by definition\notetag{Polish redundant formulas}
\begin{align}
    c_b^2:=\frac{dp}{d\rho}_{\mid\sigma}&=\frac{p_b + p_y\, \tfrac{dy}{db}_{\mid\sigma}}{\rho_b + \rho_y\, \tfrac{dy}{db}_{\mid\sigma}} 
    =\frac{-b^2F_{bb} F_{yy}+(F_y-bF_{yb})^2}{F_{yy}\,(yF_y-bF_b)}\\[5pt]
    &= \frac{b p_b\, y\rho_y+(yp_y)^2}{b\rho_b\,y\rho_y+y\rho_y\,yp_y} = \frac{b\,p_b \,y\rho_y + (y\,p_y)^2}{y \rho_y (\rho+p)}\,.
\end{align}
The rewritings in the second line might come handy when trying to show positivity or other relations.
Using $F=-\mathfrak{a}$ and $s=F_y=-\partial_T\mathfrak{a}$, we rewrite the adiabatic sound speed as
\begin{align}
    c_b^2:=
    &=\frac{-b^2F_{bb} F_{yy}+(F_y-bF_{yb})^2}{F_{yy}\,(yF_y-bF_b)} = \frac{n^2\,\partial_n^2\mathfrak{a}\,\partial_Ts + (s-n\partial_ns)^2}{\partial_Ts\,\,(\rho+p)}\,,
\end{align}
which makes it manifest that $c_b^2\geq0$ provided no ghosts $\rho+P>0$, the second principle of thermodynamics $\partial_Ts\geq 0$ and thermodynamic stability $\partial_n^2\mathfrak{a}\geq 0$ hold.
The equivalence $c_a^2\equiv c_s^2$ is proved in \eqref{eq:cs_equiv_ca_flat_spacetime} for flat spacetime.\todotag{check curved spacetime}

Similarly we compute the nonadiabatic pressure coefficient $\frac{\partial P}{\partial \sigma}_{|\rho}$.
We proceed as before enforcing 
\begin{equation}
    0=d\rho=yF_{yy}dy+(yF_{yb}-F_b)db\quad \Rightarrow\quad \frac{dy}{db}_{|\rho}=\frac{F_b-yF_{yb}}{yF_{yy}}
\end{equation}
As above we get, with analgous possible thermoodynamical rewriting in terms of the free energy density $\mathfrak{a}$,  
{\small
\begin{align}\label{eq:nonadiabatic_pressure_coefficient_gamma}
\begin{aligned}
    \Gamma:=\frac{\partial p}{\partial \sigma}_{\mid\rho}&=\frac{p_b + p_y\, \tfrac{dy}{db}_{\mid\rho}}{\sigma_b + \sigma_y\, \tfrac{dy}{db}_{\mid\rho}}=\frac{-bF_{bb}+(F_y-bF_{by})\tfrac{F_b-yF_{yb}}{yF_{yy}}}{\Big(\tfrac{bF_{by}-F_y}{b^2}\Big)+\tfrac{F_{yy}}{b}\frac{F_b-yF_{yb}}{yF_{yy}}}
    \\
    &=\frac{-b^2F_{bb}y^2F_{yy}+(yF_y-byF_{by})(bF_b-byF_{yb})}{y^2F_{yy}\Big(\tfrac{bF_{by}-F_y}{b}\Big)+yF_{yy}(F_b-yF_{yb})}
    \\
    &=\frac{b^2F_{bb}y^2F_{yy}-(yF_y-byF_{by})(bF_b-byF_{yb})}{\tfrac{1}{by}y^2F_{yy}(yF_y-bF_b)}.
\end{aligned}\end{align}}

